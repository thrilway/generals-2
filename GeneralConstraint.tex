%
%arara: pdflatex
%arara: biber
%arara: pdflatex
%arara: pdflatex

\documentclass[GPFinal]{subfiles}
\begin{document}
In the previous section, I argued that the restriction on indefinite SC subjects was due to a requirement that SC subjects contain but not be CT marked constituents.
In this section I will argue that, rather than being a parochial constraint, this requirement can be derived from a more general constraint on CT-Foc structure.
The general constraint I will propose, is that in addition to contrastive/new information, as marked by CT and Foc intonation, CT-Foc structured utterances require given/presupposed material.
\ex. \textbf{A General constraint on CT-Foc Structure}\\
In addition to the new/contrastive material marked by CT and Foc, a CT-Foc structured utterance must contain discourse given/presupposed content.

When we consider the types of utterances and discourses that CT-Foc theory was designed to account for, we can see that none violate the constraint.
Take, for example, the main example from \textcite{jackendoff1972Ssemantics} reproduced below.
\ex. (What about FRED? What did HE eat?)\\
FRED$_{CT}$ ate [the BEANS]$_F$.
\a.
\a.[\textbf{Focus:} ] the beans
\b.[\textbf{CT:} ] Fred
\c.[\textbf{Given/presupposed:} ] x ate y
\z.
\b.
\begin{forest}
  tree defaults
  [Who ate what?
    [What did Fred eat?
      [Did Fred eat the beans?
	[Yes]
      ]
      [\ldots]
    ]
    [\ldots]
  ]
\end{forest}
\z.

In this case we can see that, even though both argument constituents are entirely new/contrastive, the relation defined by the verb is given/presupposed.
The same holds of every example of licit CT-Foc structured uterrances.
In fact we can even have utterances in which the entire sentence as uttered is marked as either CT or Foc so long as there is some implicit given/presupposed material.
Consider the example in \Next.
\ex.\label{ex:FredDo} (What about FRED? what did HE do?)\\
FRED$_{CT}$ [ate the beans]$_F$.
\a.
\a.[\textbf{Focus:} ] ate the beans
\b.[\textbf{CT:} ] Fred
\c.[\textbf{Given/presupposed:} ] $\exists x\exists e[\text{Agent}(x,e)]$
\z.
\b.
\begin{forest}
  tree defaults
  [Who did what?
    [What did Fred do?
      [Did Fred eat the beans?
	[Yes]
      ]
      [\ldots]
    ]
    [\ldots]
  ]
\end{forest}
\z.

So, in \Last, even though it is not pronounced, the given/presupposed infomation that allows the CT-Foc structure is the fact that there is some event with an agent.
In fact, most current conceptions of clause structure assume a phonologically null head, \textit{v}/\textit{v*}/Voice, that assigns the agent thematic role, so the given material is encoded directly in the syntax as shown in the tree below.
\ex.
\begin{forest}
  tree defaults
  [VoiceP 
  [Fred$_{CT}$,draw] 
    [,en 
      [Voice] 
      [VP$_F$,draw 
	[ate the beans,triangle]
      ]
    ]
  ]
\end{forest}

Given the data used to propose CT-Foc structure, it should come as no surprise that there has been no explicit mention of any constraint on CT-Foc structure of the nature I propose here.
Copular structures however are distinct from utterances with ordinary verbs in their absence of thematic structure, since copular structures only seem to encode the fact that some predicate holds of some individual.
As shown below, ruled-out SCs would only presuppose a predicate-argument relation, which is not enough to be considered given/presupposed
\ex.* A doctor is Mary
\a.
\a.[\textbf{Focus:} ] Mary
\b.[\textbf{CT:} ] A doctor
\c.[\textbf{given/presupposed:} ] $\exists P\exists x[P(x)]$
\z.
\b.
\begin{forest}
  tree defaults
  [??
    [Who is a doctor?
      [Is Mary a doctor?
	[Yes]
      ]
      [\ldots]
    ]
    [\ldots]
  ]
\end{forest}
\z.

A possible counterexample to the the general constraint is given below in \Next
\ex. JOHN$_{CT}$ is ALTRUISTIC$_F$.

If \textit{John} and \textit{altruistic} are new/contrastive, then we would be left with only the copula as given/presupposed.
This, however would be associated with a particular discourse context, one in which \Last could not be uttered felicitously.
\ex. (What about JOHN? What is HE?)\\
\#? JOHN$_{CT}$ is ALTRUISTIC$_F$.
\a.[\textbf{Focus:} ] Altruistic
\b.[\textbf{CT:} ] John
\c.[\textbf{given/presupposed:} ] $\exists P\exists x[P(x)]$

The more natural discourse that \LLast could be uttered in would focus the polarity of the utterance and leave the predicate given/presupposed as shown below in \Next.
\ex. (What about JOHN? Is HE altruistic?)\\
JOHN$_{CT}$ is ALTRUISTIC$_F$.
\a.[\textbf{Focus:} ] Polarity=Positive 
\b.[\textbf{CT:} ] John
\c.[\textbf{given/presupposed:} ] $\exists x[\text{Altruistic}(x) \in \left\{ 0,1 \right\}]$

This type of CT-Foc structure, though allowed by the general constraint, is unavailable in SCs because the subject, where CT must be marked, and the copula, where polarity is encoded, do not form a syntactic constituent.

Another type of predicational clause that seems problematic for this general constraint is demonstrated below in \Next.
\ex.\label{ex:MaryDoctorPred} MARY$_{CT}$ is [a doctor]$_F$.

If we assume that these clauses are uninverted versions of SCs, we would expect the following type of information structure from them.
\ex.
\a.
\a.[\textbf{Focus:} ] A doctor
\b.[\textbf{CT:} ] Mary
\c.[\textbf{given/presupposed:} ] $\emptyset$
\z.
\b.
\begin{forest}
  tree defaults
  [??
    [What is Mary?
      [Is Mary a doctor?
	[Mary is a doctor]
      ]
      [\ldots]
    ]
    [\ldots]
  ]
\end{forest}
\z.

If this were the information structure of \LLast, it would violate the general constraint I propose here.
Of course, it isn't obvious that \Last represents the information structure of \LLast when we consider how \LLast is most naturally used and interpreted.
Consider the mini-discourse in \Next.
\ex. 
\a.[A:] What about MARY? What does SHE do?
\b.[B:] MARY$_{CT}$ is [a doctor]$_F$.

Notice the parallel between \Last and \ref{ex:FredDo}, which was another apparent counterexample until we saw that the subject's agency was discourse given.
Likewise, in the discourse in \Last, the question is \textit{What does she do?}, so the focus value of the answer would be things that can be \textit{done}, that is, activities.
This gives us a different information structure,shown below in \Next, one that does not violate the general constraint.
\ex.
\a.
\a.[A:] What about MARY? What does SHE do?
\b.[B:] MARY$_{CT}$ is [a doctor]$_F$.
\z.
\b.
\a.[\textbf{Focus:} ] A doctor
\b.[\textbf{CT:} ] Mary
\c.[\textbf{given/presupposed:} ] x does y
\z.
\b.
\begin{forest}
  tree defaults
  [Who does what?
    [What does Mary do?
      [Is Mary a doctor?
	[Mary is a doctor.]
      ]
      [\ldots]
    ]
    [\ldots]
  ]
\end{forest}
\z.

This analysis of the CT-Foc structure of \ref{ex:MaryDoctorPred}, seems to predict that its syntactic structure is distinct from ordinary predicational copular clauses.
Specifically, it predicts that the syntactic structure of \ref{ex:MaryDoctorPred} contains an agent-assigning Voice head\footnote{
  One might claim that \ref{ex:MaryDoctorPred} only expresses predication, and the link between it and the question it answers is some sort of Gricean inference.
  So, when A hears \textit{Mary is a doctor}, they interpret it as narrowly saying that the property of being a doctor holds of Mary, and infer that the answer to the question is that Mary does things that doctors do.
  This, however, assumes that Gricean reasoning is necessarily taking propositions built from narrowly interpreting sentences and returning distinct propositions, and this assumption is unfounded.
  Consider the following discourse:
  \ex. 
  \a.[A: ] Which man did the boy see?
  \b.[B: ] The boy saw the man with binoculars.

  Surely, Gricean reasoning would lead A to assume that B, being cooperative, intended to produce one structure associated with the uttered string and not the other possible structure.
  In this case Gricean reasoning allows A to choose between two syntactic structures rather than inferring one proposition from another.
  The claim that \ref{ex:MaryDoctorPred} is an example of Gricean reasoning, then, is not an argument against the syntactic ambiguity argued for here.
}.
Since an explication of the particulars of syntactic strucures is beyond the scope of this paper, I will leave this as a prediction to be tested in later work.
However, assuming it to be the case that \ref{ex:MaryDoctorPred} encodes the agentivity of the subject in its syntactic structure, the sentence will differ from copular structures in that it has thematic structure and thus would not invert as easily.
This explains why the corresponding apparent SC is not licit even when agentivity is presupposed.

The general constraint on CT-Foc structure that I have proposed here seems to hold, but why it should hold is not immediately clear.
There are a few options that we can rule out, though.
The constraint is unlikely to be phonological in nature, as it is about the discourse status of the semantic content of an utterance.
It is also not syntactic, since the distribution of new and given material cannot be stated in any sensible syntactic terms.
The given/presupposed content can be its own constituent, it can form a constituent with either the CT-marked or Foc-marked material or both, or it can be relatively freely distributed throughout the clause.

This leaves semantics and pragmatics as the possible locus of the constraint.
Choosing whether the constraint is semantic or pragmatic requires a definition of the domains of semantic vs pragmatic constraints.
Since there is no such thing as an uncontroversial distinction between semantics and pragmatics, I will leave the question for later research.

\end{document}
