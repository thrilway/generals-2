%
%arara: pdflatex
%arara: biber
%arara: pdflatex
%arara: pdflatex

\documentclass[GPFinal]{subfiles}
\begin{document}
In the previous section, I argued that the restriction on indefinite SC subjects was due to a requirement that SC subjects contain but not be CT marked constituents.
In this section I will argue that, rather than being a parochial constraint, this requirement can be derived from a more general constraint on CT-Foc structure.
The general constraint I will propose, is that in addition to contrastive/new information, as marked by CT and Foc intonation, CT-Foc structured utterances require given/presupposed material.
\ex. \textbf{A General constraint on CT-Foc Structure}\\
In addition to the the new/contrastive material marked by CT and Foc, a CT-Foc structured utterence must contain discourse given/presupposed content.

When we consider the types of utterances and discourses that CT-Foc theory was designed to account for, we can see that none violate the constraint.
Take, for example, the main example from \textcite{jackendoff1972Ssemantics} reproduced below.
\ex. (What about FRED? What did HE eat?)\\
FRED$_{CT}$ ate [the BEANS]$_F$.
\a.
\a.[\textbf{Focus:} ] the beans
\b.[\textbf{CT:} ] Fred
\c.[\textbf{Given/presupposed:} ] x ate y
\z.
\b.
\begin{forest}
  tree defaults
  [Who ate what?
    [What did Fred eat?
      [Did Fred eat the beans?
	[Yes]
      ]
      [\ldots]
    ]
    [\ldots]
  ]
\end{forest}
\z.

In this case we can see that, even though both argument constituents are entirely new/contrastive, the relation defined by the verb is given/presupposed.
The same holds of every example of licit CT-Foc structured uterrances.
In fact we can even have utterences in which the entire sentence as uttered is marked as either CT or Foc so long as there is some implicit given/presupposed material.
Consider the example in \Next.
\ex. (What about FRED? what did HE do?)\\
FRED$_{CT}$ [ate the beans]$_F$.
\a.
\a.[\textbf{Focus:} ] ate the beans
\b.[\textbf{CT:} ] Fred
\c.[\textbf{Given/presupposed:} ] $\exists x\exists e[\text{Agent}(x,e)]$
\z.
\b.
\begin{forest}
  tree defaults
  [Who did what?
    [What did Fred do?
      [Did Fred eat the beans?
	[Yes]
      ]
      [\ldots]
    ]
    [\ldots]
  ]
\end{forest}
\z.

So, in \Last, even though it is not pronounced, the given/presupposed infomation that allows the CT-Foc structure is the fact that there is some event with an agent.
Given the data used to propose CT-Foc structure, it should come as no surprise that there has been no explicit mention of any constrain on CT-Foc structure of the nature I propose here.
Copular structures however are distinct from utterences with ordinary verbs in their absence of thematic structure, since copular structures only seem to encode the fact that some predicate holds of some individual.
As shown below, ruled-out SCs would only presuppose a predicate-argument relation, which is not enough to be considered given/presupposed
\ex.* A doctor is Mary
\a.
\a.[\textbf{Focus:} ] Mary
\b.[\textbf{CT:} ] A doctor
\c.[\textbf{given/presupposed:} ] $\exists P\exists x[P(x)]$
\z.
\b.
\begin{forest}
  tree defaults
  [??
    [Who is a doctor?
      [Is Mary a doctor?
	[Yes]
      ]
      [\ldots]
    ]
    [\ldots]
  ]
\end{forest}
\z.

A possible counterexample to the the general constraint is given below in \Next
\ex. JOHN$_{CT}$ is ALTRUISTIC$_F$.

If \textit{John} and \textit{altruistic} are new/contrastive, then we would be left with only the copula as given/presupposed.
This, however would be associated with a particular discourse context, one in which \Last could not be uttered felicitously.
\ex. (What about JOHN? What is HE?)\\
\#? JOHN$_{CT}$ is ALTRUISTIC$_F$.
\a.[\textbf{Focus:} ] Altruistic
\b.[\textbf{CT:} ] John
\c.[\textbf{given/presupposed:} ] $\exists P\exists x[P(x)]$

The more natural discourse that \LLast could be uttered in would focus the polarity of the utterence and leave the predicate given/presupposed as shown below in \Next.
\ex. (What about JOHN? Is HE altruistic?)\\
JOHN$_{CT}$ is ALTRUISTIC$_F$.
\a.[\textbf{Focus:} ] Polarity=Positive 
\b.[\textbf{CT:} ] John
\c.[\textbf{given/presupposed:} ] $\exists x[\text{Altruistic}(x) \in \left\{ 0,1 \right\}]$

This type of CT-Foc structure, though allowed by the general constraint, is unavailable in SCs because the subject, where CT must be marked, and the copula, where polarity is encoded, do not form a syntactic constituent.

The general constraint on CT-Foc structure that I have proposed here seems to hold, but why it should hold is not immediately clear.
There are a few options that we can rule out, though.
The constraint is unlikely to be phonological in nature, as it is about the discourse status of the semantic content of an utterence.
It is also not syntactic, since the distribution of new and given material cannot be stated in any sensible syntactic terms.
The given/presupposed content can be its own constituent, it can form a constituent with either the CT-marked or Foc-Marked material or both, or it can be relatively freely distributed throughout the clause.

This leaves semantics and pragmatics as the possible locus of the constraint.
Choosing whether the constraint is semantic or pragmatic requires a definition of the domains of semantic vs pragmatic constraints.
Since there is no such thing as an uncontroversial distinction between semantics and pragmatics, I will leave the question for later research.

\end{document}
