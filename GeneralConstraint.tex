%
%arara: pdflatex
%arara: biber
%arara: pdflatex
%arara: pdflatex

\documentclass[GPFinal]{subfiles}
\begin{document}
In the previous section, I argued that the restriction on indefinite SC subjects was due to a requirement that SC subjects contain but not be CT marked constituents.
In this section I will argue that, rather than being a parochial constraint, this requirement can be derived from a more general constraint on CT-Foc structure.
The general constraint I will propose, is that in addition to contrastive/new information, as marked by CT and Foc intonation, CT-Foc structured utterances require given/presupposed material.

\subsection{The Empirical Base of CT-Foc Theory}
When we consider the types of utterances and discourses that CT-Foc theory was designed to account for, we can see that none violate the constraint.
Take, for example, the main example from \textcite{jackendoff1972Ssemantics} reproduced below.
\ex. (What about FRED? What did HE eat?)\\
FRED$_{CT}$ ate [the BEANS]$_F$.
\a.[Focus: ] the beans
\b.[CT: ] Fred
\c.[given/presupposed: ] x ate y

In this case we can see that, even though both argument constituents are entirely new/contrastive, the relation defined by the verb is given/presupposed.
The same holds of every example of licit CT-Foc structured uterrances.
In fact we can even have utterences in which the entire sentence as uttered is marked as either CT or Foc so long as there is some implicit given/presupposed material.
Consider the example in \Next.
\ex. (What about FRED? what did HE do?)\\
FRED$_{CT}$ [ate the beans]$_F$.
\a.[Focus: ] ate the beans
\b.[CT: ] Fred
\c.[given/presupposed: ] $\exists x\exists e[\text{Agent}(x,e)]$

So, in \Last, even though it is not pronounced, the given/presupposed infomation that allows the CT-Foc structure is the fact that there is some event with an agent.

\end{document}
