%
% arara: pdflatex: {options: "-draftmode"}
% arara: biber
% arara: pdflatex: {options: "-draftmode"}
% arara: pdflatex: {options: "-file-line-error-style"}
\documentclass[GPFinal]{subfiles}
\begin{document}
\subsection{Felicity vs Well-formedness}
Two factors in the (un)acceptability of utterances have been shown in this section: felicity and well-formedness.
The two factors are distinguishable from each other in how they interact with context.
The felicity of an utterance is its appropriateness in a given discourse, so it is completely context dependant.
Well-formedness, on the other hand, depends only on the structures projected by an utterance, meaning that it is context independant.
Put another way, a well-formed utterance used in the wrong discourse context is infelicitous, but an ill-formed utterance is unacceptable in any context.

So, is the indefinite restriction based on felicity or well-formedness?
\textcite{mikkelsen2004specifying} suggests that, because the restriction is due to the fact that SC subjects must be topics, it is pragmatic in nature and therefore based on felicity.
\textcite{heycockkroch1999pseudocleft,heycock2012specification}, however argue that the restriction is semantic in nature and therefore based on well-formedness.
It seems, based on the above discussion however, that the indefinite restriction is pragmatic in nature \textit{and} based on well-formedness.
Since my analysis of the indefinite restriction starts with the proposal that SC subjects must contain a CT constituent and the notion of CT is pragmatic in nature, the indefinite restriction is essentially pragmatic according to my analysis.
SCs with simple indefinite subjects, however, are ruled out because there is no possible CT-F structure associated with them that projects a well-formed d-tree, so the indefinite restriction must be based on well-formedness considerations.

In the above discussion of CT-F structures and the discourse pragmatics of SCs, we have seen clear instances of infelicity and ill-formedness, and one instance in which the source of unacceptability is in some way mixed.
The assertion in \ref{ex:HilBagelInfel}, reproduced below in \Next, projects a well-formed d-tree that is incongruent with it's context, making the utterance infelicitous.
\ex. 
\a.
\a.[A:] Who ate bagels?
\b.[B:] \#[Hilary]$_{CT}$ ate [bagels]$_F$.
\z.
\b. $\llbracket$[Hilary]$_{CT}$ ate [bagels]$_F\rrbracket^{ct}$\\
\begin{forest}
  tree defaults
  [What did who eat?
    [What did Robin eat?]
    [What did Hilary eat?
      [\textbf{Hilary ate bagels}]
    ]
  ]
\end{forest}
\z.

The SCs with simple indefinite subjects like those in \ref{ex:BadSCs} are unacceptable because the d-trees constructed from them are ill-formed.
Since they are ill-formed \textit{per se}, considerations of felicity do not enter into their assessment.

A more complex case is that of \ref{ex:Beatles}, reproduced below in \Next.
\ex.
\a. Paul was a vocalist in the Beatles.
\b. \#A [guitarist]$_{CT}$ in the Beatles was [John]$_F$.

I have marked this SC as infelicitous, but it is presented as evidence of the well-formedness condition on d-trees.
While there is a d-tree that \Last[b] could project that would be felicitous in this context (reproduced in \Next below), that d-tree is ill-formed.
\ex. 
\begin{forest}
  tree defaults
  [Who was what in the in the Beatles?
    [Who was a guitarist \dots
      [\dots was John]
      [\dots was George]
    ]
    [Who was a vocalist \dots
      [\dots was Paul]
      [\dots was John]
      [\dots]
    ]
    [\dots]
  ]
\end{forest}

The well-formed d-tree projected by \LLast[b] would not include the question \textit{Who was a vocalist in The Beatles?}, rendering the utterance infelicitous.

We are left with the following picture of utterances, d-trees and felicity.
Well-formedness judgements are the result of a function from d-trees to truth-values.
Felicity judgements can be considered a function from well-formed d-trees to contexts to truth-values.
So the felicity of an ill-formed utterance is undefined in all contexts.\footnote{
  Felicity can indeed be recast as a well-formedness condition on discourse, but this would still be distinct from the well-formedness conditions discussed here which are conditions on individual utterances  
}
\end{document}
