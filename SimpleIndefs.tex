%
% arara: pdflatex: {options: "-draftmode"}
% arara: biber
% arara: pdflatex: {options: "-draftmode"}
% arara: pdflatex: {options: "-file-line-error-style"}
\documentclass[GPFinal]{subfiles}
\begin{document}
Given the surface generalization (\textit{SC subjects must contain but not be a CT-Marked constituent}) proposed above, the constraint on simple indefinites as SC subjects is fairly obvious.
In the examples below in \Next, the only candidate for CT marking would be the nominal.
CT-marking on the indefinite article would render the these strings acceptable, and DP2 is inherently focused. 
\ex.\label{ex:BadSCs}
\a.* A doctor is Derek.
\b.* A building is Robarts Library.
\c.* A linguist is Eric Lenneberg.
\z.

If we assume, following \textcite{heim1982semantics}, the indefinite article in English if semantically vacuous, CT-marking on the nominal is equivalent to CT-Marking on the entire subject.
Even without this assumption, we can see that CT-marking on the nominal of a simple indefinite has odd consequences.
If the nominal is our CT constituent, then what is the discourse/information-structure function of the determiner?
Is it the aboutness topic?
Unlikely, because what does it mean for a sentence to be about \textit{a(n)}?
Is it the Background?
Again, unlikely, because what does it mean for an assertion to presuppose \textit{a(n)}?
The (deaccented) indefinite article does not seem to serve any discourse role on its own, so CT-marking on the nominal of a simple indefinite is equivalent to CT-Marking on the entire DP.
Since simple indefinite SC subjects would be, rather than contain, CT-marked constituents, they violate the surface generalization and are ruled out. 

Simple indefinites also allow us to investigate why the surface generalization should hold.
What's wrong with an SC subject being, rather than containing, a CT-marked constituent?
Let's consider how \Last[a] would be interpreted.
\ex.
\a.
\a. $\llbracket$A doctor is Derek$\rrbracket = Doctor(\mathbf{d})$
\b. $\llbracket$A doctor is Derek$\rrbracket^f = \{x \in Alt(\mathbf{d}) | Doctor(x)\}$
\b. $\llbracket$A doctor is Derek$\rrbracket^{ct} = \{P \in Alt(Doctor)|\{x \in Alt(\mathbf{d}) | P(x)\}\}$
\z.
\b.
\begin{forest}
  tree defaults
  [Who is what?\\What about who?,align=center
    [Who is a doctor?
      [Is Derek a doctor?
	[Yes]
      ]
      [Who is another doctor?
	[\ldots,triangle]
      ]
    ]
    [Who is what else?\\What else about who?,align=center
      [\ldots,triangle]
    ]
  ]
\end{forest}
\z.

If we also recall that CT-marking triggers an exhaustivity implicature, then \textit{A doctor is Derek} would trigger an implicature that Derek is a doctor and no alternative property holds of Derek.
This leads to the question: What are the excluded alternative properties?

In his original formulation of aternative semantics, \textcite{rooth1985diss,rooth1992theory} considered a constituent's alternative set to be the domain of items of the same semantic type, so the alternatives of \textit{doctor} would be the domain of properties of individuals.
More recent work \parencite[][among others]{wagner2006givenness,katzir2007structurally,katzir2013note,fox2011characterization} has argued that for a given item, its alternative set is much more restricted.
%The source of this restriction varies depending on the proposal, but
%\textcite{wagner2006givenness} argues that the alternative set for a focused element X that has a sister Y is the set of elements that contrast with X as sisters of Y.
%So, in \Next, \textit{BLUE convertibles} has as its alternatives only \textit{RED convertibles}.
%\ex.
%\a. John only owns BLUE convertibles.
%\b. \textbf{Alternatives:}
%\a. John owns red convertibles.
%\b.* John owns expensive convertibles.
%\b.* John owns cheap convertibles.
%\z.
%\z.
%
%Applying this to \textit{*A doctor is Derek} yields no alternatives.
%Suppose the unrestrictive alternatives to \textit{A doctor} when its sister is \textit{is Derek} is the set of professions.
%Professions do not contrast with one another, as no two professions are mutually exclusive.
%So, \textit{A doctor} yields no contrastive alternatives, and therefore, since the function of CT-marking is alternative generation, \textit{A doctor} cannot be a contrastive topic.
%
%\textcite{katzir2013note}, however points out that under an intensional verb such as \textit{collect}, the the alternative set no longer seems to depend on contrast given local environment.
%\ex.
%\a. John only collects BLUE convertibles.
%\b. \textbf{Alternatives:}
%\a. John collects red convertibles.
%\b. John collects expensive convertibles.
%\b. John collects cheap convertibles.
%\z.
%\z.
%
%
%<+Alternatives+>
%
The source of this restriction varies depending on the proposal, but common to all the approaches is the assumption of some larger domain of alternatives which is restricted in a particular use by syntactic or contextual factors.
For the purposes of this paper, I will assume, following \textcite{rooth1985diss,rooth1992theory}, that the base alternative set is the the set of all items of the same semantic type.
Given that the (un)acceptability of the pair of strings in \Next is context-independent, I will assume that contextual factors are irrelevant here.
\ex. 
\a. A MALE doctor is Derek.
\b.* A doctor is Derek.

So, how would the syntactic environment pare down the alternatives?
If we make the assumption that the given/non-contrastive material in a sentence restricts the alternatives generated by CT (and Foc), then we are forced to say syntactic factors cannot restrict the CT alternatives in \textit{*A doctor is Derek}.
Recall that, in SCs, DP2 is always the focused.
This means that the material DP2 is not given, and therefore cannot help to restrict the CT alternative set.
So we are left with the copula, and therefore tense, aspect, modality, \textit{etc.}, as given material.
It is not clear that the material encoded in the copula could meaningfully restrict the CT alternatives in \textit{*A doctor is Derek}, so I will assume that it does not.
With nothing to restrict it, we are left with the unrestricted alternative set.

When combined, the exhaustivity implicature that CTs trigger and the unrestricted alternative set generated by simple indefinite SC subjects lead to the conclusion that \textit{*A doctor is Derek} implies that the individual Derek has the property of being a doctor and no other property. 
This implicature is implausible and is ruled out.
Given their structure, every SC with a simple indefinite subject triggers this implausible implicature and is therefore ruled out.

\end{document}
