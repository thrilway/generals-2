%        File: lf.tex
%     Created: Tue Apr 14 11:00 AM 2015 E
% Last Change: Tue Apr 14 11:00 AM 2015 E
%
\documentclass[letterpaper,12pt]{article}

\usepackage[margin=1in]{geometry}
%\usepackage{natbib}
\usepackage{linguex}

\usepackage[]{amsmath}
\usepackage{stmaryrd}

\usepackage{forest}

\forestset{tree defaults/.style={for tree={parent anchor=south, child anchor=north},every tree node/.style={align=center,anchor=north},level/.style={sibling distance=50mm/#1},baseline}}

\usepackage[backend=bibtex,style=authoryear]{biblatex}

\bibliography{GP2}

\begin{document}
\begin{itemize}
  \item A good doctor if you have high blood pressure is Tanya.
  \item $[\forall s: HBP(s)][\exists x: good(k_s)(x) \wedge doctor(x)(s)][Tanya(x)]$
    \begin{itemize}
      \item $HBP(s)$ = $you_{\textsc{gen}}$ have high blood pressure in s.
      \item $k_s$ = the scale, relative to the situation s, on which goodness is judged. \parencite{kennedy2007vagueness}
    \end{itemize}
  \item A problem with universal quantification over situations:
    \begin{itemize}
      \item Suppose there are two German restaurants in our town: \textit{Irene's} and \textit{Angelika's}.
	\textit{Irene's} serves traditional German cuisine, while \textit{Angelika's} serves vegetarian German cuisine.
	\textit{Irene's} is considered better.
      \item A good restaurant if you like German food is \textit{Irene's}
      \item A good restaurant if you like German food and are vegetarian is \textit{Angelika's}
    \end{itemize}
  \item Existential quantification seems too weak.
  \item Perhaps quantifying over \textit{minimal} situations \parencite{elbourne2005situations} plus an implicature.
  \item $[\exists s,x: min(s) \wedge HBP(s) \wedge good(k_s)(x) \wedge doctor(x)(s)][Tanya(x)]$
\end{itemize}
\printbibliography
\end{document}
