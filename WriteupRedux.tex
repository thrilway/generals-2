%        File: WriteupRedux.tex
%     Created: Fri May 15 04:00 PM 2015 E
% Last Change: Fri May 15 04:00 PM 2015 E
%
% arara: pdflatex
% arara: bibtex
% arara: pdflatex
% arara: pdflatex
\documentclass[letterpaper]{article}

\usepackage[margin=1in]{geometry}
\usepackage[backend=bibtex]{biblatex}


\usepackage[]{amsmath}
\usepackage{stmaryrd}

\usepackage{linguex}

\bibliography{GP2}

\usepackage{forest}
\forestset{tree defaults/.style={for tree={parent anchor=south, child anchor=north},every tree node/.style={align=center,anchor=north},level/.style={sibling distance=50mm/#1},baseline}}

\begin{document}

Consider the increasing acceptability of the following specificational clauses:
\ex.
\a.\label{ex:bad}\# A figure is Eric Lenneberg.
\b.\label{ex:so-so}\#? A figure in the history of Generative Grammar is Eric Lenneberg.
\b.\label{ex:good} An underrated figure in the history of Generative Grammar is Eric Lenneberg.
\z.

What properties of the indefinite DP in \ref{ex:good} makes it an acceptable subject?
It is not being interpreted as a definite description, as it lacks a presupposition of uniqueness.
\ex.
\a. An underrated figure in the history of Generative Grammar is Eric Lenneberg.\\
Another one is Tanya Reinhart.
\b. The most underrated figure in the history of Generative Grammar is Eric Lenneberg.\\
\#Another one is Tanya Reinhart.

As \textcite{mikkelsen2004specifying} argues, specificational subjects are interpreted as topics, so whatever makes the indefinite in \ref{ex:good} a good topic is what allows it to be a subject.
Following \textcite{buring1999topic,roberts2012information}, I treat the notions of topic and focus in relation to questions under discussion, formalized as discourse trees.
In a given discourse, the QUD is the root node, and foci of assertions are terminal nodes.
\begin{figure}[h]
  \centering
  \begin{forest}
    tree defaults
    [QUD 
      [SubQ1
	[SubSubQ1
	  [Ans1]
	  [Ans2]
	  [Ans3]
	]
	[SubSubQ2
	  [Ans4]
	  [Ans5]
	]
      ]
      [SubQ23
	[Ans6]
	[Ans7]
	[Ans8]
      ]
    ]
  \end{forest}
  \caption{A sample discourse tree}
  \label{fig:dtree}
\end{figure}

I propose that in order for an indefinite DP to serve as the subject of a specificational clause, it must have as a constituent, a \textit{contrastive topic} (CT).
Consider the licit clause in \ref{ex:good}.
The modifier, \textit{underrated}, expresses the CT, while the remainder of the DP expresses the QUD.
\ex. An (underrated)$_\text{CT}$(figure in the history of Generative Grammar)$_\text{QUD}$

The sentence in \ref{ex:so-so} is marginal because the subject has no constituent that can express a CT.
To show why this is consider the most likely division between CT and QUD for \ref{ex:so-so} below.
\ex.\# A (figure)$_\text{CT}$(in the history of Generative Grammar)$_\text{QUD}$

Why is \textit{figure} not a licit CT, while \textit{underrated} is?
I propose that for a constituent to be a CT it must carry an implication that its opposite could be a a CT.
\textcite{buring1999topic} gives a  CT-value formation algorithm, reproduced below, which can demonstrate the contrast between \textit{figure} and \textit{underrated}.
\ex. CT-value formation:\\
\begin{tabular}[t]{lp{0.6\textwidth}}
  Step 1: & Replace the focus with a \textit{wh}-word and front the latter\\
  Step 2: & Form a set of questions from the result of step 1 by replacing the contrastive topic with some alternative to it.
\end{tabular}

To demonstrate, consider the application of CT-value formation on \ref{ex:good}.
\ex. CT-value formation:\\
\begin{tabular}[t]{ll}
  Step 1: & Who is an underrated figure in the history of GG?\\
  Step 2: & Who is an underrated figure in the history of GG?\\
  & Who is an accurately rated figure in the history of GG?\\
  & Who is an overrated figure in the history of GG?\\
\end{tabular}

Compare this to the output of CT-value formation on \ref{ex:so-so}.
\ex. CT-value formation:\\
\begin{tabular}[t]{ll}
  Step 1: & Who is a figure in the history of GG?\\
  Step 2: & Who is a figure in the history of GG?\\
  & \# Who is an event in the history of GG?\\
  & \# Who is a circumstance in the history of GG?\\
  & \ldots
\end{tabular}

Note that the alternatives given by the algorithm are ill-formed questions due to the mismatch between the \textit{wh}-word and the object of the predicate.
The failure of the algorithm to generate any alternative CTs, makes \textit{figure} unsuited to be a CT.
The CT-values for \ref{ex:so-so} and \ref{ex:good} can be expressed as trees as below:
\ex. $\llbracket\ref{ex:so-so}\rrbracket^{CT}$\\
\begin{forest}
  tree defaults
  [What is the history of GG?
    [Who is a figure in the history of GG?
      [Eric Lenneberg]
      [Noam Chomsky]
      [Morris Halle]
      [\ldots]
    ]
  ]
\end{forest}

\ex. $\llbracket\ref{ex:good}\rrbracket^{CT}$\\
\begin{forest}
  tree defaults
  [Who is a figure in the history of GG?
    [Who is an underrated figure?
      [Eric Lenneberg]
      [Tanya Reinhart]
      [\ldots]
    ]
    [Who is an accurately rated figure?
      [Noam Chomsky]
      [Morris Halle]
      [\ldots]
    ]
    [\ldots]
  ]
\end{forest}

\end{document}


