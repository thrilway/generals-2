%        File: 2June.tex
%     Created: Mon Jun 01 10:00 PM 2015 E
% Last Change: Mon Jun 01 10:00 PM 2015 E
%
\documentclass[a4paper]{article}
\usepackage[margin=1in]{geometry}
\usepackage[backend=bibtex]{biblatex}

\usepackage{linguex}

\usepackage[]{amsmath}
\usepackage{stmaryrd}

\begin{document}
\ex.\label{bad} To be explained:
\a.\# A figure is Eric Lenneberg.
\b.\# A doctor is Mary.

\ex.\label{ct} $\llbracket$A doctor is Mary$\rrbracket^{CT}=$\\
$\left\{ P | \left\{ x | P(x) \right\}  \right\}$

\ex.\label{qud} QUD(A doctor is Mary) = Who is what?

\begin{itemize}
	\item From last time: \ref{bad} is infelicitous because it implies \ref{qud}, which is an illicit question.
		\begin{itemize}
			\item Why is it illicit?
		\end{itemize}
	\item It seems to put everything under discussion.
	\item No discourse exists in a vacuum, though.
	\item Some answers to \ref{qud} will be included in the context.
\end{itemize}
\ex. Informal hypothesis:\\
A question \textit{q} is felicitous in a discourses only if there is no assertion \textit{a} such that \textit{a} is a partial answer to \textit{q}

\ex. Consider:\\
A: What did the pop-stars wear?\\
B: The \textsc{female} pop-stars wore \textsc{caftans}.\\
A: \# But, What did the pop-stars wear?

\begin{itemize}
	\item Roberts models QUD as a stack.
	  \begin{itemize}
	    \item Q's are pushed in when they are asked.
	    \item Q's are popped off when they are answered or determined to be unanswerable.
	  \end{itemize}
	\item How to CTs work?
\end{itemize}
\ex. 
	\a. Ingredients:
		\a. \textit{q} = What did the pop-stars wear?
		\b. \textit{sq1} = What did the female pop-stars wear?
		\c. \textit{sq2} = What did the male pop-stars wear?
		\d. \textit{sq3} = What did the non-female pop-stars wear?
		\z.
	\b. \textit{m} =\\
	$\llbracket$The \textsc{female} pop-stars wore \textsc{caftans}$\rrbracket^{CT}=$ 
		\a. push$_{QUD}$(\textit{sq1}) \hfill (by CT implication)
		\b. pop$_{QUD}$(\textit{sq1}) \hfill (ans(\textit{sq1}) = caftans)
		\c. END\\
		\textit{or}\\
		push$_{QUD}$(\textit{sq2})\\
		\textit{or}\\
		push$_{QUD}$(\textit{sq3})
		\z.
	\z.

\begin{itemize}
  \item B\"uring (2014) proposes conditions on CTs:
\end{itemize}
\ex. CT-Interpretation Rule (CIR):\\
For a sentence S$^\text{CT+F}$ to be felicitous, there must be at least one question meaning in S$^\text{CT+F}$’s CT-value which is
\a. currently pertinent,
\b. logically independent of the ordinary interpretation of S
\c. identifiable.
\z.

\begin{itemize}
  \item Pertinence seems to require that questions be answerable.
  \item Independence seems to require a non-singleton set of questions.
  \item Identifiability seems to require the set of questions be finite or closed.
\end{itemize}
\ex.When are you guys' birthdays?
\a. \textsc{Marcy}$_\text{CT}$’s birthday is on the 12th of \textsc{September}$_\text{F}$, and \dots
\b. \# On the 12th of \textsc{September}$_\text{CT}$, it's \textsc{Marcy}$_\text{F}$'s birthday, and \dots \hfill *Identifiability
\z.

\begin{itemize}
  \item Given a contextually determined set of addressees we can identify an answerable question \textit{When is x's birthday?} but not \textit{who's birthday is on d?} where \textit{x} is an individual and \textit{d} is a date.
  \item The sentences in \ref{bad} seem to fail here. We cannot identify an answerable question \textit{Who is a y?}
  \item Consider:
\end{itemize}
\ex.
\a. \# A red wine is merlot.
\b. \# A stout is Guinness.
\z.

\begin{itemize}
  \item Surely any person familiar with wine or beer can identify an answerable alternative in the CT interpretation of these.
  \item The relations between red and white wine, stout and lager, and doctors and lawyers are certainly different from those between you and me, John and Mary, and a lager I like and one I don't.

\end{itemize}

\end{document}


