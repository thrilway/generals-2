%        File: MainArgument.tex
%     Created: Wed Sep 09 12:00 PM 2015 E
% Last Change: Wed Sep 09 12:00 PM 2015 E
%
% arara: pdflatex: {options: "-draftmode"}
% arara: biber
% arara: pdflatex: {options: "-draftmode"}
% arara: pdflatex: {options: "-file-line-error-style"}
\documentclass[GPFinal]{subfiles}

\begin{document}
I am now prepared to modify Mikkelsen's (\citeyear{mikkelsen2004specifying}) analysis of SCs so that is properly captures the indefinite restriction.
Recall that Mikkelsen argued that SCs have a fixed information structure, with the postcopular DP being focus and the subject being topic, as shown in \Next below, and that for Mikkelsen, topicality requires discourse familiarity.
\ex. [My favourite singer]$_\text{Top}$ is [Ian]$_F$.

I propose that SC subjects must \textit{contain} (but not be) a \textit{contrastive} topic, in the sense of \textcite{buring2003d,buringforthcomingtopic}.
I will show, in the remainder of this section, that this addition to Mikkelsen's analysis effectively captures the indefinite restriction.
Specifically, requiring SC subjects to contain a CT will account for the fact that more complex/heavy indefinites (such as those in \ref{ex:MikkPhilosopher}-\ref{ex:MikkBarcan}) are more likely to be acceptable SC subjects as well as the fact that simple indefinites are almost never allowed as SC subjects.

\subsection{When indefinites can be SC subjects}
Consider the following SCs with their natural CT-F structure marked.
\ex.\label{ex:GoodSCs}
	\a. A [newly minted]$_{CT}$ doctor is [Derek$_F$].
	\b. A building on campus [no one]$_{CT}$ likes is [Robart's Library]$_F$.
	\c. An [underrated]$_{CT}$ figure in the history of Generative Grammar is [Eric Lenneberg]$_F$.
	\z.

Using \posscite{buring2003d} algorithm for CT-F interpretation we can compute the alternatives for each of these SCs and determine types of contexts they could be felicitously uttered in.
Consider the CT value of \Last[b].
To derive that we first replace the focus with \textit{what} and form a wh-question (\textit{What is a building on campus [no one]$_{CT}$ likes?}), then we replace the CT with \textit{who} to get the implied QUD (\textit{What is a building on campus who likes?}). 
This is formalized below in \Next.
\ex.\label{ex:GoodSCAnalysis}
	\a. 
		\a. $\llbracket\LLast[b]\rrbracket^f = \left\{ BuildingOnCampus(x) \wedge likes(x)(\emptyset) | x \in D_e \right\}$
		\b. $\llbracket\LLast[b]\rrbracket^{CT} = \left\{\left\{ BuildingOnCampus(x) \wedge likes(x)(y) | x \in D_e \right\} | y \in D_e\right\}$
		\z.
	\b. 
	\begin{forest}
	  tree defaults
	  [What is a building on campus who likes?
	    [What is \dots Joe likes?
	      [\dots]
	    ]
	    [\dots,triangle]
	    [What is \dots no one likes?
	      [\dots,triangle] 
	      [A building on campus\\no one likes\\is Robart's Library,align=center]
	    ]
	  ]	    
	\end{forest}
	\z.

This correctly predicts that \LLast[b], for instance, should only be felicitous in discourse contexts where buildings on campus and who likes them is under discussion.
If, as shown in \Next, things on campus that no one likes is under discussion, \LLast[b] is infelicitous, but a version of the same string with a modified CT-F structure is felicitous
\ex. No one likes Philosopher's walk but,\footnote{Philosopher's walk is a foot-path that most people at UofT do like}
\a.\# A building on campus [no one]$_{CT}$ likes is Robart's Library.
\b. A [building]$_{CT}$ on campus no one likes is Robart's Library.
\z.

The context in \Last can be expressed as the QUD \textit{What is what-kind-of-thing on campus no one likes?} which is not available in the d-tree constructed in \LLast from \Last[a].

In addition to NP internal costituents, some indefinite determiner elements can act as CTs.
In particular, \textit{one} and \textit{another}, are able to bear the stress that marks CT.
When these do serve as CTs, the remaining material in the SC subject is treated as discourse-given.
For concrete examples consider the examples in \Next.
\ex.
\a.\label{ex:One} [One]$_{CT}$ building on campus no-one likes is Robart's library.
\b.\label{ex:Another}[Another]$_{CT}$ newly minted doctor is Ail\'is.

In \ref{ex:One}, buildings on campus that no one likes must be discourse-given and Robart's library being a member of that set is discourse-new.
The function of marking \textit{one} as CT is not immediately clear, though.



\subsection{Simple indefinites}
Having only shown how indefinites can be licit SC subjects, we are no further to a full explanation of the indefinite restriction than Mikkelsen, who also could explain these facts.
We are left with the problem of explaining why simple indefinites seem to be so ill-suited as SC subjects.
In other words, we have a hypothesis that includes the examples in \ref{ex:GoodSCs}, we need to show that it also excludes the strings in \ref{ex:BadSCs}, below.
\ex.\label{ex:BadSCs}
\a.* A doctor is Derek.
\b.* A building is Robarts Library.
\c.* A figure is Eric Lenneberg.
\z.

So, why are the strings in \Last unacceptable?
The hypothesis that I have been pursuing in this paper provides an initial answer: Simple indefinite SC subjects cannot be CTs.
This of course is not a satisfactory answer, because it simply reframes the question as: Why can't simple indefinite SC subjects be CTs?
The theoretical framework developed by \textcite{rooth1992theory,roberts2012information,buring2003d,buringforthcomingtopic}, however, makes the reframed question tractable.

Consider how \Last[a] would be interpreted.
The postcopular DP would be focused, and, since the indefinite article cannot ordinarily\footnote{<++>} bear CT stress, \textit{doctor} would bear CT Stress.
To compute the CT interpretation, we use \posscite{buring2003d} algorithm.
The focus value would be equivalent to \textit{Who is a doctor?} and the CT value would be approximately equivalent to \textit{Who is a what?}, as formalized below.
\ex.\label{ex:BadSCAnalysis}
	\a.
		\a. $\llbracket\LLast[a]\rrbracket^f = \left\{ Doctor(x) | x \in D_e \right\}$
		\b. $\llbracket\LLast[a]\rrbracket^{CT} = \left\{ \left\{ P(x) | x \in D_e \right\} | P \in D_{et} \right\}$
		\z.
	\b. 
	\begin{forest}
	  tree defaults
	  [Who is a what?
	    [Who is a doctor
	      [A doctor is Derek]
	      [A doctor is Mary]
	      [\dots]
	    ]
	    [Who is a lawyer?]
	    [Who is a socialist
	      [A socialist is Mary]
	    ]
	    [\dots]
	  ]
	\end{forest}

Compare the QUD implied here, to that of \ref{ex:GoodSCAnalysis}.
The QUD derived in \Last is exceedingly broad, essentially asking what properties hold of which individuals.
There seems to be no restriction on what properties and which individuals for the domains of this QUD.
Perhaps, the set of alternative properties is restricted to those that could possibly hold of human individuals, but even that is not very much of a restriction.
The QUD derived in \ref{ex:GoodSCAnalysis} is quite restricted by comparison.
The domain of the focus alternatives is restricted to buildings on (some salient) campus, while the domain of the CT-alternatives is restricted (implicitly) to individuals who might hold some opinion of some buildings on campus.

A result of the unrestricted nature of the QUD derived in \Last, is that a given individual can be in multiple terminal nodes in the d-tree.
This, I argue, is an unwanted result, which is at the centre of why simple indefinite SC subjects make poor CTs.
As evidence of this claim consider the following example of an SC, which is infelicitous for this reason.
\ex. \label{ex:Beatles} \textbf{Context:} The Beatles were a band with four members John, Paul, George, and Ringo. Each member played an instrument and sang.
\a. Paul was a vocalist in the Beatles.
\b. \#A [guitarist]$_{CT}$ in the Beatles was [John]$_F$.
\z.

Despite the fact that there is no reason in principle for the indefinite in \Last[b] to be an illicit SC subject, the context makes it illicit.
Consider the d-tree that would be constructed from \Last[b].
\ex.
\begin{forest}
  tree defaults
  [Who was what in the in the Beatles?
    [Who was a guitarist \dots
      [\dots was John]
      [\dots was George]
    ]
    [Who was a vocalist \dots
      [\dots was Paul]
      [\dots was John]
      [\dots]
    ]
    [\dots]
  ]
\end{forest}

The fact that \textit{John} is referenced in two terminal nodes renders \LLast[b] infelicitous.
Note, however, that the SC \LLast[b] can be made more felicitous in a context that doesn't explicitly mention vocalists.
This is because the exclusion of the role of vocalist, allows the domain of the CT alternatives to be restricted to instrumentalist roles.
\ex.
\a.
\a. Paul was the bassist in the Beatles,
\b. A [guitarist]$_{CT}$ in the Beatles was John.
\z.
\b. $\llbracket\Last[a-i]\rrbracket^{ct}$\\
\begin{forest}
  tree defaults
  [Who was $\langle instrumentalist\rangle$ in the Beatles?
    [Who was a bassist in \ldots?
      [\ldots was Paul]
    ]
    [Who was a drummer in \ldots?
      [\ldots was Ringo]
    ]
    [who was a guitarist in \ldots?
      [\ldots was George]
      [\ldots was John]
    ]
  ]
\end{forest}

Similarly, there are examples of simple indefinites which are licit SC subjects in a given context because their nominal has an inherently restricted alternative set.
Two such nominals are the antonyms \textit{hero} and \textit{villain}.
Consider the following examples.
\ex.
\a. In \textit{Othello},\\
a villain is Iago.
\b. In \textit{War \& Peace},\\
a hero is Pierre.

<++>

It seems that d-trees (at least, those generated by CT+F structures) require that no one individual whose properties are under discussion be referred to in multiple terminal nodes.
Rather than merely stipulating this as a condition on d-trees, we ought to build it into how we construct those trees.
Instead of constructing a tree from alternatives in which each alternative is a node, I propose that d-trees be binary branching.
The three values of a CT-F utterance ($\llbracket\cdot\rrbracket^\mathcal{O}$,$\llbracket\cdot\rrbracket^{f}$,$\llbracket\cdot\rrbracket^{ct}$), are assigned to nodes with $\llbracket\cdot\rrbracket^{ct}$ in the highest node (a branching node), which immediately dominates $\llbracket\cdot\rrbracket^{f}$ (a branching node), which in turn immediately dominates $\llbracket\cdot\rrbracket^\mathcal{O}$ (a teminal node) as in \Next.
\ex. 
\begin{forest}
  tree defaults
  [{$\llbracket\cdot\rrbracket^{ct}$}
    [{$\alpha$}]
    [{$\llbracket\cdot\rrbracket^{f}$}
      [{$\beta$}]
      [{$\llbracket\cdot\rrbracket^\mathcal{O}$}]
    ]
  ]
\end{forest}

This first step leaves empty nodes $\alpha$ and $\beta$, which are filled by sets containing the alternatives such each alternative is a member of the immediately dominating node, and no member is identical to the sister node.
That is to say branching represents a two member partition of the parent node.
The d-tree in \Next demonstrates how the empty nodes are filled.
\ex. 
\begin{forest}
  tree defaults
  [{$\llbracket\cdot\rrbracket^{ct}$}
    [{$\llbracket\cdot\rrbracket^{ct}\setminus\left\{\llbracket\cdot\rrbracket^{f}\right\}$}]
    [{$\llbracket\cdot\rrbracket^{f}$}
      [{$\llbracket\cdot\rrbracket^{f}\setminus\left\{ \llbracket\cdot\rrbracket^\mathcal{O} \right\}$}]
      [{$\llbracket\cdot\rrbracket^\mathcal{O}$}]
    ]
  ]
\end{forest}

This binary branching, partitioning d-tree can rule out SCs such as \textit{A doctor is Mary} given a few extra, though not outlandish, assumptions.
I first assume (as does \textcite{roberts2012information}), following <+stalnaker1978+>, that the goal of discourse is the communal discovery of how the world is.
This goal, of course is much too ambitious for real world discourse, which instead can only realistically address parts of the world (\textit{i.e.}, situations).
I also assume \posscite{kratzer1989investigation} ontology of situations, which states that situations are partially ordered into semilattices joined at worlds with individuals as lower bounds.
<+MaybeMoreHere+>

\singlespacing \textit{\textbf{Note:} I may end up just stipulating that discourse moves are must define a partition. I'm not sure if I can come up with a coherent explanation as to why this would be.}
\doublespacing

The restriction on simple indefinite SC subjects may also be ruled out by what \textcite{groenendijkstokhof1996questions} call \textit{Hamblin's postulates} reproduced below.
\ex. \textbf{Hamblin's postulates} \parencite[][emphasis mine]{groenendijkstokhof1996questions}
\a.[i] An answer to a question is a sentence, or statement.
\b.[ii] The possible answers to a question form an \textit{exhaustive set of mutually exclusive posibilities}.
\b.[iii] To know the meaning of a question, is to know what counts as an answer to that question.
\z.

Hamblin's second postulate, as \textcite{groenendijkstokhof1996questions} elaborate on it, highlights what is so problematic about the examples in \ref{ex:BadSCs}.
For \citeauthor{groenendijkstokhof1996questions}, questions divide possible worlds into equivalence classes, so, \textit{Did John answer?} divides the possible world into those in which John answered and those in which he didn't.
The question implied by those unacceptable SCs in \ref{ex:BadSCs}, \textit{Who/what does what property hold of?}, would need to define mutually exclusive classes of properties such that the property named by the subject (\textit{Doctor}, \textit{building}, \textit{figure}) defines a class that excludes not those individuals for whom it doesn't hold, but those individuals for whom a different property holds.

To take a concrete example, \textit{*A doctor is Mary} picks out the set of worlds in which Mary is a doctor and no other property holds of Mary.
Worlds in which Mary holds (or does not hold) any number of political or aesthetic beliefs, for example, cannot be included in the set of worlds picked out by the SC, as are worlds in which Mary has (or does not have) hair of a certain colour, length, or type.
This is, fairly obviously, an absurdity.

So, the task of discovering the way the world is involves discovering which properties hold of which individuals.

If propositions are situation denoting, and questions are sets of propositions, then questions are sets of situations.
Since every situation is ultimately composed of smaller situations and ultimately individuals

The partitions represented by branching d-trees, then, require a partitioning of the set of individuals.
So, \textit{A doctor is Mary} can be ruled out because it essentially asserts that the property of being a doctor holds of Mary to the exclusion of all other properties.
On the other hand, if the domain of properties is sufficiently restricted, a similar assertion can be made.
The sentence \textit{A doctor you SHOULD see is Mary}, for instance, would assert (in one context) that Mary is a doctor you should talk to and is not a doctor you should not talk to.
Similarly, the sentence \textit{In Othello, a VILLAIN is Iago} asserts that Iago is a villain and is not a hero in \textit{Othello}.

This is not to say that there is an absolute restriction on which lexical items can serve as CTs.
Capacity nominals such as \textit{doctor}, though they cannot be CTs in simple indefinites, can be CTs so long as their alternative sets can be restricted in such a way that a partition of individuals is possible.
Consider for instance the discourse in \Next.
\ex.
\a.[A:] I'm having the worst back pain. Do you know any doctors who could cure it?
\b.[B:] Well, a [chiropractor]$_{CT}$ you should see is Mary.\\

In this discourse, the context, that A needs a someone to treat her back pain, restricts the alternatives generated by the CT to professions that would treat back pain.
So, while \textit{doctor}, \textit{chiropractor}, and \textit{physiotherapist} would be alternatives in this context, \textit{semanticist}, \textit{man}, and \textit{American} would not.



\subsection{Felicity vs Well-formedness}
Two factors in the (un)acceptability of utterances have been shown in this section: felicity and well-formedness.
The two factors are distinguishable from each other in how they interact with context.
The felicity of an utterance is its appropriateness in a given discourse, so it is completely context dependant.
Well-formedness, on the other hand, depends only on the structures projected by an utterance, meaning that it is context independant.
Put another way, a well-formed utterance used in the wrong discourse context is infelicitous, but an ill-formed utterance is unacceptable in any context.

So, is the indefinite restriction based on felicity or well-formedness?
\textcite{mikkelsen2004specifying} suggests that, because the restriction is due to the fact that SC subjects must be topics, it is pragmatic in nature and therefore based on felicity.
\textcite{heycockkroch1999pseudocleft,heycock2012specification}, however argue that the restriction is semantic in nature and therefore based on well-formedness.
It seems, based on the above discussion however, that the indefinite restriction is pragmatic in nature \textit{and} based on well-formedness.
Since my analysis of the indefinite restriction starts with the proposal that SC subjects must contain a CT constituent and the notion of CT is pragmatic in nature, the indefinite restriction is essentially pragmatic according to my analysis.
SCs with simple indefinite subjects, however, are ruled out because there is no possible CT-F structure associated with them that projects a well-formed d-tree, so the indefinite restriction must be based on well-formedness considerations.

In the above discussion of CT-F structures and the discourse pragmatics of SCs, we have seen clear instances of infelicity and ill-formedness, and one instance in which the source of unacceptability is in some way mixed.
The assertion in \ref{ex:HilBagelInfel}, reproduced below in \Next, projects a well-formed d-tree that is incongruent with it's context, making the utterance infelicitous.
\ex. 
\a.
\a.[A:] Who ate bagels?
\b.[B:] \#[Hilary]$_{CT}$ ate [bagels]$_F$.
\z.
\b. $\llbracket$[Hilary]$_{CT}$ ate [bagels]$_F\rrbracket^{ct}$\\
\begin{forest}
  tree defaults
  [What did who eat?
    [What did Robin eat?]
    [What did Hilary eat?
      [\textbf{Hilary ate bagels}]
    ]
  ]
\end{forest}
\z.

The SCs with simple indefinite subjects like those in \ref{ex:BadSCs} are unacceptable because the d-trees constructed from them are ill-formed.
Since they are ill-formed \textit{per se}, considerations of felicity do not enter into their assessment.

A more complex case is that of \ref{ex:Beatles}, reproduced below in \Next.
\ex.
\a. Paul was a vocalist in the Beatles.
\b. \#A [guitarist]$_{CT}$ in the Beatles was [John]$_F$.

I have marked this SC as infelicitous, but it is presented as evidence of the well-formedness condition on d-trees.
While there is a d-tree that \Last[b] could project that would be felicitous in this context (reproduced in \Next below), that d-tree is ill-formed.
\ex. 
\begin{forest}
  tree defaults
  [Who was what in the in the Beatles?
    [Who was a guitarist \dots
      [\dots was John]
      [\dots was George]
    ]
    [Who was a vocalist \dots
      [\dots was Paul]
      [\dots was John]
      [\dots]
    ]
    [\dots]
  ]
\end{forest}

The well-formed d-tree projected by \LLast[b] would not include the question \textit{Who was a vocalist in The Beatles?}, rendering the utterance infelicitous.

We are left with the following picture of utterances, d-trees and felicity.
Well-formedness judgements are the result of a function from d-trees to truth-values.
Felicity judgements can be considered a function from well-formed d-trees to contexts to truth-values.
So the felicity of an ill-formed utterance is undefined in all contexts.\footnote{
  Felicity can indeed be recast as a well-formedness condition on discourse, but this would still be distinct from the well-formedness conditions discussed here which are conditions on individual utterances  
}

\section{Summary}
In this section I argued that the indefinite restriction, specifically the restriction on simple indefinite SC subjects, is the result of well-formedness constraints on d-trees.
Given the observation that SC subjects must contain CT constituents, and the alternative semantics of CT-F structures represented by \posscite{buring2003d} d-trees, we can see that the CT-F structure required by simple indefinite SC subjects generates a d-tree with individuals represented in multiple terminal nodes.
These structures, I argued, seem to be ruled out regardless of context.
In the next section I will address some apparent counterevidence to this claim.
\end{document}
