%        File: MainArgument.tex
%     Created: Wed Sep 09 12:00 PM 2015 E
% Last Change: Wed Sep 09 12:00 PM 2015 E
%
% arara: pdflatex: {options: "-draftmode"}
% arara: biber
% arara: pdflatex: {options: "-draftmode"}
% arara: pdflatex: {options: "-file-line-error-style"}
\documentclass[GPFinal]{subfiles}

\begin{document}
I am now prepared to modify Mikkelsen's (\citeyear{mikkelsen2004specifying}) analysis of SCs so that is properly captures the indefinite restriction.
Recall that Mikkelsen argued that SCs have a fixed information structure, with the postcopular DP being focus and the subject being topic, as shown in \Next below, and that for Mikkelsen, topicality requires discourse familiarity.
\ex. [My favourite singer]$_\text{Top}$ is [Ian]$_F$.

I propose that SC subjects must be \textit{contrastive} topic, in the sense of \textcite{constant2014diss}, but must not be wholly F-marked.
I will show, in the remainder of this section, that this addition to Mikkelsen's analysis effectively captures the indefinite restriction.
Specifically, requiring SC subjects to properly contain an F-marked constituent will account for the fact that more complex/heavy indefinites (such as those in \ref{ex:MikkPhilosopher}-\ref{ex:MikkBarcan}) are more likely to be acceptable SC subjects as well as the fact that simple indefinites are almost never allowed as SC subjects.
\ex. \textbf{The Contrastive Topic requirement on Specificational Clauses}\\
A clause of the form $X$ \textsc{be} $Y$ is a licit specificational clause iff
\a. $\llbracket X\rrbracket(\llbracket Y\rrbracket)$ is defined,
\b. $Y$ is an exhaustive focus, \parencite{mikkelsen2004specifying}
\b. $X$ is a contrastive topic, and
\b. A F-marked constituent is properly contained by $X$.

In the above definition, \Last[a] restricts the requirement to possible SCs, and \Last[b] incorporates Mikkelsen's observation of the fixed information structure of SCs.
The final two parts of the requirement, \Last[c] and \Last[d] are what I will argue for in the following two sections.

I have framed this proposal as a condition on SCs in general rather than one on indefinite subects of SCs for reasons of parsimony.
While indefinite subjects play an important role in the discussion that follows, I intersperse SCs with definite subjects for ease of demonstration.

\subfile{SCSubjsContainCTs}
%\subsection{When indefinites can be SC subjects}

\end{document}
