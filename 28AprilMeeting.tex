%        File: 28AprilMeeting.tex
%     Created: Tue Apr 28 10:00 AM 2015 E
% Last Change: Tue Apr 28 10:00 AM 2015 E
%
\documentclass[letterpaper]{article}
\usepackage[margin=1in]{geometry}
\usepackage[backend=bibtex]{biblatex}

\usepackage{linguex}

\usepackage[]{amsmath}
\usepackage{stmaryrd}

\begin{document}
\begin{center}
  {\Large Meeting notes}\\
  \today
\end{center}
\ex.
  \a.\# A building on campus is Robart's.
  \b.\# A building that is on St George St. is Robart's
  \b.? A building that exemplifies Brutalism is Robart's.
  \b.
    \a.\#? A building that no one likes is Robart's.
    \b.\label{ex:emph} A building that NO ONE likes is Robart's.
    \z.
  \z.

\ex. 
  \a. A person behind the success of Generative Grammar is Morris Halle.
    \a. An underrated figure in the development of Generative Grammar is Eric Lenneberg.
    \z.
  \b. A city that was built below sea level is Amsterdam.
    \a. A city built below sea level is Amsterdam.
    \z.
  \b. A person running for president is Hillary Clinton.
    \a. A Democrat running for president is Hillary Clinton.
    \z.
  \z.

\begin{itemize}
  \item There seem to be two repair strategies for these sentences:
    \begin{itemize}
      \item Interpret \textit{a} as \textit{one}.
      \item Emphasize part of the subject.
    \end{itemize}
  \item Both impose a presupposition on the sentence:
    \begin{itemize}
      \item \textit{one} presupposes there are others.
      \item Emphasis on an element makes it a contrastive topic:
	\begin{itemize}
	  \item Example \ref{ex:emph} implies/presupposes that there are buildings that few but not no people like.
	\end{itemize}
    \end{itemize}
\end{itemize}
\end{document}
