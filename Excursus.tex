%        File: Excursus.tex
%     Created: Thu Sep 10 03:00 PM 2015 E
% Last Change: Thu Sep 10 03:00 PM 2015 E
%
% arara: pdflatex: {options: "-draftmode"}
% arara: biber
% arara: pdflatex: {options: "-draftmode"}
% arara: pdflatex: {options: "-file-line-error-style"}
\documentclass[GPFinal]{subfiles}

\begin{document}
Assuming the analysis I have given for the indefinite restriction, it is worth asking how this fits into the general cognitive architecture.

\textcite{fodor2001mind} argues that mental processes come in two varieties: local and global.
Local processes are computational in the sense that they can be carried out by a Turing Machine.
For a given local process, we can define a deterministic function from a structured symbolic input to some output.
An example of a local process, is a derivation of a phrase marker, or the compositional interpretation of a phrase marker.
These processes depend only of the the input's constituent symbols and their organization.

Global processes, on the other hand, involve many factors external to their input.
Take for example the interpretation of counterfactuals such as the classic example in \Next, below.
\ex. If Nixon had pushed the button, there would have been a nuclear holocaust.\hfill <++>

This is interpreted as meaning that the closest possible world to the actual world in which Nixon pushed the button is a world in which there is a nuclear holocaust.
The selection of the \textit{closest} posible world depends not only on the semantic content of \Last but also on our general understanding of how the world works.
Fodor argues that these global processes are not at present amenable to analysis using any known method in cognitive science.
Whether or not Fodor's pessimism is waranted is beyond the scope of this paper, because an analysis of any of these global processes is well byond the scope of this paper.

So, are the processes at work in the indefinite restriction local or global?
There are certainly some processes which can be considered local, but other important processes seem to be global in nature.
If we take the overall process that defines the indefinite restriction to be composed of several ordered subprocesses, we can see which are global and which are local.
\ex. Assessing CT felicity\\
Construct a d-tree $\rightarrow$ Populate d-tree $\rightarrow$ Assess d-tree well-formedness

The first step in this process is to construct an appropriate d-tree based on the CT-F structure of an utterance.
Since \textcite{buring2003d} already defines this process with a simple algorithm, we can safely classify it as a local process.
The final step is also a local process, as it is only concerned with properties of the d-tree.
The middle process, however, requires a great deal of encyclopedic and contextual knowledge which interacts in a complex way to generate the alternatives that populate the d-tree.

\ex.
\begin{tikzpicture}
  \node[draw,cloud,aspect=1.5,minimum width = 5cm] (knowledge) at (0,2) {Knowledge};
  \node[draw,rectangle,rounded corners] (population) at (0,0) {Populate d-tree};
  \node[draw,rectangle,rounded corners] (construct) at (-4,0) {Construct d-tree};
  \node[draw,rectangle,rounded corners] (assess) at (4,0) {Assess d-tree};
  \node (input) at (-7.5,0) {Input from $\llbracket\cdot\rrbracket^{CT}$};
  \node (output) at (6.5,0) {Output};
  \draw[->,thick] (knowledge.south west) arc[start angle=180,end angle=360,y radius=0.9cm,x radius=1.75cm];
  \draw[->,thick] (input.east)--(construct.west);
  \draw[->,thick] (construct.east)--(population.west);
  \draw[->,thick] (population.east) -- (assess.west);
  \draw[->,thick] (assess.east) -- (output.west); 
\end{tikzpicture}


\end{document}


