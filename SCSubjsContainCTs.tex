%
% arara: pdflatex: {options: "-draftmode"}
% arara: biber
% arara: pdflatex: {options: "-draftmode"}
% arara: pdflatex: {options: "-file-line-error-style"}
\documentclass[GPFinal]{subfiles}
\begin{document}
\subsection{SC subjects must contain contrastive topics}
The first claim of my proposal that must be justified is that \textit{contrastive} topichood, rather than givenness or aboutness topichood is the relevant notion for SC subjects.
This claim can be further divided into three claims.
First, CT-Foc structure is a licit information structure for SCs.
Second, SC subjects cannot be entirely discourse given.
Finally, SC subjects cannot be aboutness topics.
In the following subsection I will present evidence for each of these claims in turn.
Following that, I will address the second component claim of my proposal, that SC subjects cannot be wholly CT marked
\subsubsection{CT-Foc structure is compatible with SCs}
English SCs are most naturally uttered with intonational stress on some part of their subject as shown in \Next.
\ex.
\a. A building on campus no-one LIKES is Robarts.
\b. A building on campus NO-ONE likes is Robarts.
\b. A building on CAMPUS no-one likes is Robarts.
\b. A building ON campus no-one likes is Robarts.
\b. A BUILDING on campus no-one likes is Robarts.
\b.? A building on campus no-one likes is Robarts.
\z.

English intonational stress is associated with informational prominence , and since, as Mikkelsen shows, DP2 position in SCs is necessarily focused, the intonation stress in the subjects of \Last cannot be focus.

Pragmatically, CT-Foc structures are characterized by association with a complex discourse strategy of a question and subquestion.
SCs can indeed be associated with a question-subquestion strategy.
\ex.Not many people like the Athletic Centre.\\
A building on campus no one likes is Robarts.
\a. 
\a.[QUD: ] What is a building on campus who likes?
\b.[SubQ: ] What is a building on campus no one likes?
\z.
\b. 
\begin{forest}
  tree defaults
  [What is a building on campus who likes?
    [What is a building on campus no one likes?
      [Is Robarts a building on campus no one likes?
	[Yes]
      ]
      [\ldots]
    ]
    [\ldots]
  ]
\end{forest}
\z.

So, intonational stress in SC subjects is consistent with CT-Foc structure.
\subsubsection{SC subjects are not wholly givenness topics}
If \textcite{mikkelsen2004specifying} is correct, and SC subjects are necessarily givenness topics, we would expect that a maximally given DP is the ideal SC subject.
As \Next demonstrates, however, maximally given DPs are not good SC subjects, but SC subjects that are minimally contrastive are acceptable.
\ex. Many philosophers have written about the mind-body problem.
\a.\# A philosopher who has written about the mind-body problem is Chomsky.
\b. A modern philosopher who has written about the mind-body problem is Chomsky.

So, SC subjects are not givenness topics.
\subsubsection{SC subjects are not wholly aboutness topics}
\textcite{reinhart1981pragmatics} argues that the important notion associated with topichood is aboutness rather than givenness.
If we wish to retain \posscite{mikkelsen2004specifying} analysis, the natural move would be to claim that licit SC subjects are characterized by aboutness.
Aboutness is diagnosable by a paraphrasing test.
\ex. \textbf{Reinhart's test for aboutness}\\
If sentence S is about constituent X, then S is paraphrasable by the sentence \textit{They said about }X\textit{, that }S$^\prime$, where S$^\prime$ is derived by replacing X in S with a proform.

As \Next shows, when the entire SC subject is the aboutness topic, as diagnosed by Reinhart's test, it is interpreted \textit{de re}, rendering the copular clause equational rather than specificational.
Ccnversely, when the subject is not entirely the aboutness topic, it is interpreted \textit{de dicto} rendering the clause specificational.
\ex. \textbf{Background:} David Bowie is John's favourite singer.
\a. Mary said of John's favourite singer that \{he/?it\}'s Iggy Pop.(Identificational/*Specificational)\\
(=Mary said David Bowie is Iggy Pop)
\b. Mary said of singers that John's favourite (one) is Iggy Pop. (*Identificational/Specificational)\\
($\neq$Mary said David Bowie is Iggy Pop)
\c. Mary said of John that his favourite singer is Iggy Pop. (*Identificational/Specificational)\\
($\neq$Mary said David Bowie is Iggy Pop)
\d. Mary said of people's favourite singers that John's is Iggy Pop. (*Identificational/Specificational)\\
($\neq$Mary said David Bowie is Iggy Pop)
\z.

So, while some part of an SC subject can be an aboutness topic, the entire subject DP cannot be the aboutness topic.
\subsubsection{Summary}
Since SC subjects are compatible with CT marking and cannot be givenness or aboutness topics, it is reasonable to assume that the presence of CT is necessary for SC subjects.
\subsection{SC subjects cannot entirely be contrastive topics}
The second claim of my proposal is that SC subjects cannot be CT-marked constituents.
If SC subjects must minimally contain a CT marked constituent, it follows directly from the unacceptibility of simple indefinite SC subjects that SC subject DPs cannot be CT-marked.
Consider the unacceptable SC \textit{*A doctor is Mary}.
The subject \textit{a doctor} must contain a CT-marked constituent, in this case \textit{doctor}.
Since the indefinite article does not encode any particular information, CT marking on the nominal is equivalent to CT marking on the entire DP.

It is worth noting here that indefinite articles can be CT-marked when a definiteness contrast is relevant in a discourse.
In these cases, simple indefinites can be SC subjects.
\ex. Who is the guitarist?\\
$[$ej$]$ guitarist is John.

So, simple indefinites can be SC subjects if they contain but do not comprise a CT-marked constituent.
\subsection{Summary}
In this section I have presented evidence that the restriction on indefinite SC subjects comes from a requirement that SC subjects contain but not be CT marked constituents.
I first showed that \textit{contrastive} rather than aboutness or givenness topichood is the source of the restriction.
I then argued that the ban on simple indefinite SC subjects is neatly predicted if the SC subject is banned from being the CT marked constituent.
In the next section, I will argue that the indefinite restriction, in fact, can be derived from a more general constraint on CT-Foc structures that is implicit in the the literature on CTs.
\end{document}
