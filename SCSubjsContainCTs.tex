%
% arara: pdflatex: {options: "-draftmode"}
% arara: biber
% arara: pdflatex: {options: "-draftmode"}
% arara: pdflatex: {options: "-file-line-error-style"}
\documentclass[GPFinal]{subfiles}
\begin{document}
\subsection{SC subjects must contain contrastive topics}
The first claim of my proposal that must be justified is that \textit{contrastive} topichood, rather than givenness or aboutness topichood is the relevant notion for SC subjects.
This claim can be further divided into three claims.
First, CT-Foc structure is a licit information structure for SCs.
Second, SC subjects cannot be entirely discourse given.
Finally, SC subjects cannot be aboutness topics.
In the following subsection I will present evidence for each of these claims in turn.
Following that, I will address the second component claim of my proposal, that SC subjects cannot be wholly CT marked
\subsubsection{CT-Foc structure is compatible with SCs}
English SCs are most naturally uttered with intonational stress on some part of their subject as shown in \Next.
\ex.
\a. A building on campus no-one LIKES is Robarts.
\b. A building on campus NO-ONE likes is Robarts.
\b. A building on CAMPUS no-one likes is Robarts.
\b. A building ON campus no-one likes is Robarts.
\b. A BUILDING on campus no-one likes is Robarts.
\b.? A building on campus no-one likes is Robarts.
\z.

English intonational stress is associated with informational prominence, and since, as Mikkelsen shows, DP2 position in SCs is necessarily focused, the intonational stress in the subjects of \Last cannot be primary focus.

Pragmatically, CT-Foc structures are characterized by association with a complex discourse strategy of a question and subquestion.
SCs can indeed be associated with a question-subquestion strategy.
Consider the example in \Next.
\ex.(Not many people like the Athletic Centre.)\\
A building on campus NO ONE likes is Robarts.

If DP2 is Foc-Marked, and the stressed constituent \textit{no one} is CT-Marked, then we can use \posscite{buring2003d} CT-value formation procedure to construct the d-tree associated with it.

\ex. CT-value formation:
\a.[step 1: ] What's a building on campus no one likes?
\b.[step 2: ] $
\begin{Bmatrix}
  \text{What's a building on campus no one likes?}\\
  \text{What's a building on campus  someone likes?}\\
  \cdots\\
  \text{What's a building on campus  everyone likes?}
\end{Bmatrix}
$

\ex.
\begin{forest}
  tree defaults
  [What is a building on campus who likes?
    [What is a building on campus no one likes?
      [Is Robarts a building on campus no one likes?
	[A building on campus no one likes is Robarts.]
      ]
      [\ldots]
    ]
    [\ldots]
  ]
\end{forest}
\z.

Similarly, we can see that the felicity conditions on the accent placement in SC subjects match the those of the canonical CT-Foc structures demonstrated in \ref{ex:HilBagelInfel} and \ref{ex:MonChiroInfel}.
So, the SCs in question need to imply a question and subquestion to which they provide a (partial) answer, and this question-subquestion-answer sequence must be congruent with the QUD.

\ex.
\a. Everyone likes Hart House\\
\# A BUIDING on campus no-one likes is Robarts.

\ex. 
\a.
\a.[A:] What's a building on campus no one knows?
\b.[B:]\# A buiding on campus [everyone LIKES] is Hart House].

So, intonational stress in SC subjects is consistent with CT-Foc structure.
\subsubsection{SC subjects are not wholly givenness topics}
If \textcite{mikkelsen2004specifying} is correct, and SC subjects are necessarily givenness topics, we would expect that a maximally given DP is the ideal SC subject.
As \ref{ex:PhilosInfel} demonstrates, however, maximally given DPs are not good SC subjects, but SC subjects that are minimally contrastive are acceptable.\footnote{
  The infelicity is not due to a constraint on repeating indefinites.
  Consider the following pair:
  \ex. Many philosophers have written about the mind-body problem.
  \a.\# A philosopher who has written about the mind-body problem is Chomsky.
  \b. A philosopher who has written about the mind-body problem came to dinner last night.
  \z.

}

\ex. Many philosophers have written about the mind-body problem.
\a.\label{ex:PhilosInfel}\# A philosopher who has written about the mind-body problem is Chomsky.
\b. A modern philosopher who has written about the mind-body problem is Chomsky.

So, SC subjects are not givenness topics.
\subsubsection{SC subjects are not wholly aboutness topics}
\textcite{reinhart1981pragmatics} argues that the important notion associated with topichood is aboutness rather than givenness.
If we wish to retain \posscite{mikkelsen2004specifying} analysis, the natural move would be to claim that licit SC subjects are characterized by aboutness.
Aboutness is diagnosable by a paraphrasing test.
\ex. \textbf{Reinhart's test for aboutness}\\
If sentence S is about constituent X, then S is paraphrasable by the sentence \textit{They said about }X\textit{, that }S$^\prime$, where S$^\prime$ is derived by replacing X in S with a proform.

As \Next shows, when the entire SC subject is the aboutness topic, as diagnosed by Reinhart's test, it is interpreted \textit{de re}, rendering the copular clause equational rather than specificational.
Conversely, when the subject is not entirely the aboutness topic, it is interpreted \textit{de dicto} rendering the clause specificational.
\ex. \textbf{Background:} David Bowie = John's favourite singer.\\
(Mary said that) John's favourite singer is Iggy Pop. (Identificational/Specificational)
\a. Mary said of John's favourite singer that \{he/?it\}'s Iggy Pop.(Identificational/*Specificational)\\
(=Mary said David Bowie is Iggy Pop)
\b. Mary said of singers that John's favourite (one) is Iggy Pop. (*Identificational/Specificational)\\
($\neq$Mary said David Bowie is Iggy Pop)
\c. Mary said of John that his favourite singer is Iggy Pop. (*Identificational/Specificational)\\
($\neq$Mary said David Bowie is Iggy Pop)
\d. Mary said of people's favourite singers that John's is Iggy Pop. (*Identificational/Specificational)\\
($\neq$Mary said David Bowie is Iggy Pop)
\z.

In the above examples, Mary's claim that John's favourite singer is Iggy Pop is invariably false, but varies in the exact claim being made.
In the case that \textit{John's favourite singer} is understood \textit{de re}, Mary is wrongly identifying David Bowie as Iggy Pop.
When \textit{John's favourite singer} is understood \textit{de dicto}, Mary is wrongly specifying the singer that John prefers above all other singers is Iggy Pop.

It has been suggested to me that it is the pronominal subject of \Last[a] that forces its identificational reading.
While I am not prepared to concede this point, even if it were true, we are left with \Last[b]--\Last[d] which cannot be captured by this claim.
If pronomial subjects forced Identificational readings, the reverse could not be true, as most SCs with full (definite) DP subjects are ambiguous with identificational readings.
If we were to apply this hypothesis to \Last[b]--\Last[d] it would be non-predictive, so we would need a further explanation for the fact that specificational readings are forced when only part of the subject is an aboutness topic as in \Last[b]--\Last[d].

So, absent any compelling argument otherwise, it seems that while some part of an SC subject can be an aboutness topic, the entire subject DP cannot be the aboutness topic.

\subsubsection{Summary}
Since SC subjects are compatible with CT marking and cannot be givenness or aboutness topics, it is reasonable to assume that the presence of CT is necessary for SC subjects.
\subsection{SC subjects cannot entirely be contrastive topics}
The second claim of my proposal is that SC subjects cannot be CT-marked constituents.
So, if the entirety of the SC subject is new/contrastive, the SC is unacceptable.

\ex.
\a.
\a.[A:] Tell me about your home university?
\b.[B:]\# A BUILDING on campus no-one likes is Robarts.
\z.

If SC subjects must minimally contain a CT marked constituent, it follows directly from the unacceptability of simple indefinite SC subjects that SC subject DPs cannot be CT-marked.
Consider the unacceptable SC \textit{*A doctor is Mary}.
The subject \textit{a doctor} must contain a CT-marked constituent, in this case \textit{doctor}.
Since the indefinite article does not encode any particular information, CT marking on the nominal is equivalent to CT marking on the entire DP.

It is worth noting here that indefinite articles can be CT-marked when a definiteness contrast is relevant in a discourse.
In these cases, simple indefinites can be SC subjects.
\ex. Who is the guitarist?\\
$[$ej$]$ guitarist is John.

So, simple indefinites can be SC subjects if they contain but do not comprise a CT-marked constituent.
\subsection{\textit{One} and \textit{another}}
As mentioned in above The determiner-like elements \textit{one} and \textit{another} can serve as CTs in SC subjects.
\ex.\label{ex:AONeAnother}
\a.* A doctor$_{CT}$ is Mary.
\b.\label{ex:OneCT} One$_{CT}$ doctor is Mary.
\b.\label{ex:AnotherCT} Another$_{CT}$ doctor is Mary.

In this section I argue that \textit{one} and \textit{another} can be CT marked, meaning they encode enough semantic material to generate alternatives.
Where possible I will attempt to sketch what is encoded by these items and what their alternatives might be.
Since \textit{one} and \textit{another} each warrant a dedicated research project, these sketches are decidedly preliminary.

Let's consider \textit{another} first.
Following \textcite{heim1991reciprocity}, I take the meaning of  \textit{other} to include two crucial parts: anaphoricity and distinctness.
Consider the sentence in \Next.
\ex. Alice met with another student.

This sentence presupposes that there is a previously mentioned student (anaphoricity) and asserts that the student Alice met with is distinct from the presupposed antecedent (distinctness). 
As we can see from \Next, the anaphoricity projects when embedded, but the distinctness does not.
\ex.
\a. Alice didn't meet with another student
\a.\# \dots she never met with any student.
\b. \dots it was the same student.
\z.
\b. If Alice met with another student, she would have told us.
\a.\# She didn't tell us because she hadn't met with a student previous to this one.
\b. She didn't tell us because it was the same student.
\z.
\b. Alice probably met with another student.
\a.\# but she might not have met with a student previous to this one.
\b. but it might have been the same student. 
\z.
\b. Johan thought that Alice met with another student.
\a.\# He was wrong. She hadn't met with a student previous to this one.
\b. He was wrong. It was the same student.
\z.
\z.

The SC in \ref{ex:AnotherCT}, then, is roughly paraphrasable as \textit{A doctor [OTHER than x] is Mary}, where the value of \textit{x} is resolved contextually.
Assuming that \textit{other} is CT marked in \ref{ex:AnotherCT}, and, following \textcite{heim1991reciprocity}, that  \textit{other} is a three-place predicate\footnote{
  \textcite{heim1991reciprocity}, discussing the reciprocals \textit{each other} and \textit{one another} give the following denotation for \textit{other}: \textit{z} is an atomic part of \textit{y}, a plural individual, and \textit{z} is distinct from \textit{x}.
  \ex. $\llbracket$other$\rrbracket = \lambda x\lambda y\lambda z(x \cdot\Pi y \wedge z \neq x)$

  If we were to translate this directly into the example under discussion (\textit{Another doctor is Mary.}), \textit{x} would be the contextually given doctor, \textit{y} would be the plural individual \textit{doctor} and \textit{z} would be \textit{Mary}.
  So the SC roughly means that \textit{x} is a doctor, Mary is not \textit{x}, and Mary is a doctor.
}, we can calculate the SC's CT-value.\footnote{
  There may be good reason to question the particulars of both of these assumptions.
  There is also good reason to believe that the particulars of these assumptions are irrelevant to the discussion at hand.
}
If we calculate the CT-value of \ref{ex:AnotherCT} given this understanding of its semantics, we can see that its acceptibility is expected under my proposal.

\ex. 
\a.
\a. $\llbracket$ANOTHER$_{CT}$ doctor is Mary$_F\rrbracket^f = \left\{ doctor(x) \wedge other(x)(\bigwedge doctor)(y) | x \in D_e \right\} (y \text{ is a doctor})$\\
(Who is another doctor?)
\b. $\llbracket$ANOTHER$_{CT}$ doctor is Mary$_F\rrbracket^{ct} = \left\{ \left\{ doctor(x) \wedge P(x)(y)(\bigwedge doctor) | x \in D_e \right\} | P \in D_{\langle e,\langle e, \langle e,t\rangle\rangle\rangle}\right\}$\\
($\approx$ Who is a doctor?)
\z.
\b. Molly$_i$ is a doctor.\\
Another$_i$ doctor is Mary.
\b.
\begin{forest}
  tree defaults
  [Who is a doctor?
    [\textbf{Molly$_i$ is a doctor.}]
    [Who is another$_i$ doctor?
      [\textbf{Another$_i$ doctor is Mary.}]
      [\ldots]
    ]
  ]
\end{forest}
\z.

So, \textit{ANOTHER doctor} contains both new/contrastive information, in \textit{other} and given/presupposed material in \textit{doctor}, thus it is a licit SC subject.

The SC in \ref{ex:OneCT} shows the inverse felicity conditions, it requires that doctors have been discussed but none have been named.
\ex.
\a. Let me tell you about doctors.\\
One doctor is Mary.
\b. Molly is a doctor.\\
\#One doctor is Mary.

If \textit{one} is merely the stressed pronunciation of \textit{a/an}, then the account I have proposed would likely require serious revision.
Fortunately, there are good reasons doubt that \textit{one} and \textit{a/an} are distinct lexical items.
First, it is unlikely that \textit{one} is the stressed version of \textit{a/an}, since \textit{a/an} has another stressed version pronounced [ej]/[\ae{}n], which usually marks a contrast of definiteness.
\ex.
\a.[A:] Are you the professor?
\b.[B:] I'm [ej] professor.

Also, \textcite{kayne2015one} presents several pieces of evidence that \textit{one} is lexically distinct from \textit{a/an}.
While \textit{a/an NP}  can be interpreted as generic, \textit{one NP} cannot
\ex.
\a. A spider has eight legs and many eyes. (generic/specific)
\b. One spider has eight legs and many eyes. (*generic/specific)\hfill\parencite{kayne2015one}

He also notes that the syntactic distribution of \textit{a/an} differs from \textit{one} as shown below.
\ex.
	\a. 
		\a. too long a book
		\b.* too long one book
		\z.
	\b.
		\a. a few books
		\b.* one few books
		\z.
	\b.
		\a.* They're selling a-drawer desks in the back of the store.
		\b. They're selling one-drawer desks in the back of the store.
		\z.\hfill\parencite{kayne2015one}

Kayne argues that \textit{one} is a complex determiner composed of \textit{a/an} and a \textit{singular classifier}, with the syntactic structure given below in \Next
Since the locus of CT marking is not the indefinite article, it must be the \textit{singular classifier}, which means that the classifier ought to be contentful enough to generate alternatives.
\ex.
\begin{forest}
  tree defaults
  [DP
    [D
      [Clf\\\textit{w-},align=center]
      [D\\\textit{an},align=center]
    ]
    [ClfP
      [$\langle \text{Clf} \rangle$]
      [NP]
    ]
  ]
\end{forest}

The licit SC \textit{One doctor is Mary} would, by hypothesis, have the following CT-Foc structure.
\ex.
\a.[\textbf{CT}: ] $\llbracket w-\rrbracket$
\b.[\textbf{Focus}: ] Mary
\b.[\textbf{given/presupposed}: ] doctor/doctors/a doctor

If this is correct, then the singular classifier must be able to generate alternatives.
The question is, what counts as an alternative to \textit{one}.
A proper answer to that question would require an in depth study of the semantics and pragmatics of \textit{one}, which is beyond the scope of this paper.
%In lieu of a proper analysis of the singular classifier component of \textit{one} I will stipulate that it encodes \posscite{carlson1977reference} REL operator, which defines the set of individuals that \textit{realize} its kind referring argument, with an additional singularity assertion.
%\ex. $\llbracket w-\rrbracket = \lambda k \in D_e . \lambda y \in D_e (REL(k)(y) \wedge Sg(y) )$
%
%So the alternatives to \textit{ONE doctor} would be non-singular realizations of the kind doctor, or perhaps all the other sets of individuals generable from the kind doctor.
%The focus-value for \textit{One doctor is Mary} would, then, be equivalent to the question \textit{Who is one doctor?}, leading to the following d-tree.
%\ex.
%\begin{forest}
%  tree defaults
%  [Who is a doctor?
%    [Who is one doctor?
%      [Is one doctor Mary?
%	[One doctor is Mary.]
%      ]
%      [\ldots]
%    ]
%    [\ldots]
%  ]
%\end{forest}
%
%This is, of course, a first attempt based on a stipulated meaning for \textit{one}.
%Future analyses of \textit{one} will likely suggest a different denotation and therefore a different alternative set.
%So long as it is not demonstrated that \textit{one}, in fact, does not generate alternatives, however, these future analyses will be consistent with the overall claim of this paper.

\subsection{Simple Definite SC Subjects}

\textcite{heycock2010variability} and \textcite{bejarkahnemuyipour2013agreement} discuss a particular reading of SCs with simple definite subjects, called ``the Poirot reading'' which is shown below in \Next.
\ex.And Poirot pointed at the Major and said ``For a long time now we have been trying to establish the identity of the murderer. But now I know\ldots\\
\ldots The murderer is you''

At first blush, this seems to be a counterexample to my proposal.
In this context, the existence and relevance \textit{the murderer} is entirely given/presupposed, while the fact that the identity of the murderer is Poirot's addressee seems to be new/contrastive. 
This would mean that no part of the subject is CT-marked, which should render the clause unacceptable.

If we consider the context carefully, we can see that this is not the entire story.
The sentence \textit{The murderer is you} would occur at the culmination of a murder mystery at which point many properties of \textit{the murderer} have been gleaned from the evidence.
The only relevant ``property'' left is \textit{the murderer}'s identity.
So, what is given is the existence, salience and uniqueness of  some murderer and several of \textit{the murderer}'s properties.
What is new/contrastive is the identity of \textit{the murderer}, and that that identity is Poirot's addressee.

Consider the following alternative discourse:
\ex. We already know the following: The murderer is 6 feet tall. The murder has dark hair. The murderer walks with a limp. From this I have deduced that \\
\ldots The murderer is you.

In the discourse leading up to \textit{The murderer is you}, we can see that \textit{the murderer} is only used referentially.
The culminating accusation shifts the usage of \textit{the murderer} to that of a predicate.
For the purposes of this paper, I will assume that shifting \textit{the murderer} from $e$ to $\langle e, t\rangle$ is accomplished by an \textsc{ident} operator \parencite[cf.][]{partee1987noun}.
The SC in \Last and \LLast, then, has the following CT-Foc structure:
\ex. The murderer is you.
\a.[\textbf{Focus: }] \textit{you}
\b.[\textbf{CT: }] \textsc{ident}
\c.[\textbf{Given/presupposed:} ] \textit{the murderer}
\z.

So, simple definite SC subjects can, in fact, be accounted for by the proposal in this paper, and therefore do not represent a counterexample.
\subsection{Summary}
In this section I have presented evidence that the restriction on indefinite SC subjects comes from a requirement that SC subjects contain but not be CT marked constituents.
I first showed that \textit{contrastive} rather than aboutness or givenness topichood is the source of the restriction.
I then argued that the ban on simple indefinite SC subjects is neatly predicted if the SC subject is banned from being the CT marked constituent.
In the next section, I will argue that the indefinite restriction, in fact, can be derived from a more general constraint on CT-Foc structures that is implicit in the the literature on CTs.
\end{document}
