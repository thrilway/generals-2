%
% arara: pdflatex: {options: "-draftmode"}
% arara: biber
% arara: pdflatex: {options: "-draftmode"}
% arara: pdflatex: {options: "-file-line-error-style"}
\documentclass[GPFinal]{subfiles}
\begin{document}
\subsection{SC subjects must contain contrastive topics}
The first claim of my proposal that must be justified is that \textit{contrastive} topichood, rather than givenness or aboutness topichood is the relevant notion for SC subjects.
This claim can be further divided into three claims.
First, CT-Foc structure is a licit information structure for SCs.
Second, SC subjects cannot be entirely discourse given.
Finally, SC subjects cannot be aboutness topics.
In the following subsection I will present evidence for each of these claims in turn.
Following that, I will address the second component claim of my proposal, that SC subjects cannot be wholly CT marked
\subsubsection{CT-Foc structure is compatible with SCs}
English SCs are most naturally uttered with intonational stress on some part of their subject as shown in \Next.
\ex.
\a. A building on campus no-one LIKES is Robarts.
\b. A building on campus NO-ONE likes is Robarts.
\b. A building on CAMPUS no-one likes is Robarts.
\b. A building ON campus no-one likes is Robarts.
\b. A BUILDING on campus no-one likes is Robarts.
\b.? A building on campus no-one likes is Robarts.
\z.

English intonational stress is associated with informational prominence, and since, as Mikkelsen shows, DP2 position in SCs is necessarily focused, the intonation stress in the subjects of \Last cannot be focus.

Pragmatically, CT-Foc structures are characterized by association with a complex discourse strategy of a question and subquestion.
SCs can indeed be associated with a question-subquestion strategy.
\ex.Not many people like the Athletic Centre.\\
A building on campus no one likes is Robarts.
\a. 
\a.[QUD: ] What is a building on campus who likes?
\b.[SubQ: ] What is a building on campus no one likes?
\z.
\b. 
\begin{forest}
  tree defaults
  [What is a building on campus who likes?
    [What is a building on campus no one likes?
      [Is Robarts a building on campus no one likes?
	[Yes]
      ]
      [\ldots]
    ]
    [\ldots]
  ]
\end{forest}
\z.

Similarly, we can see that the felicity conditions on the accent placement in SC subjects match the those of the canonical CT-Foc structures demonstrated in \ref{ex:HilBagelInfel} and \ref{ex:MonChiroInfel}.
So, the SCs in question need to imply a question and subquestion to which they provide a (partial) answer, and this question-subquestion-answer sequence must be congruent with the QUD.

\ex.
\a. Everyone likes Hart House\\
\# A BUIDING on campus no-one likes is Robarts.

\ex. 
\a.
\a.[A:] What's a building on campus no one knows?
\b.[B:]\# A buiding on campus [everyone LIKES] is Hart House].

So, intonational stress in SC subjects is consistent with CT-Foc structure.
\subsubsection{SC subjects are not wholly givenness topics}
If \textcite{mikkelsen2004specifying} is correct, and SC subjects are necessarily givenness topics, we would expect that a maximally given DP is the ideal SC subject.
As \Next demonstrates, however, maximally given DPs are not good SC subjects, but SC subjects that are minimally contrastive are acceptable.
\ex. Many philosophers have written about the mind-body problem.
\a.\# A philosopher who has written about the mind-body problem is Chomsky.
\b. A modern philosopher who has written about the mind-body problem is Chomsky.

So, SC subjects are not givenness topics.
\subsubsection{SC subjects are not wholly aboutness topics}
\textcite{reinhart1981pragmatics} argues that the important notion associated with topichood is aboutness rather than givenness.
If we wish to retain \posscite{mikkelsen2004specifying} analysis, the natural move would be to claim that licit SC subjects are characterized by aboutness.
Aboutness is diagnosable by a paraphrasing test.
\ex. \textbf{Reinhart's test for aboutness}\\
If sentence S is about constituent X, then S is paraphrasable by the sentence \textit{They said about }X\textit{, that }S$^\prime$, where S$^\prime$ is derived by replacing X in S with a proform.

As \Next shows, when the entire SC subject is the aboutness topic, as diagnosed by Reinhart's test, it is interpreted \textit{de re}, rendering the copular clause equational rather than specificational.
Conversely, when the subject is not entirely the aboutness topic, it is interpreted \textit{de dicto} rendering the clause specificational.
\ex. \textbf{Background:} David Bowie = John's favourite singer.\\
(Mary said that) John's favourite singer is Iggy Pop. (Identificational/Specificational)
\a. Mary said of John's favourite singer that \{he/?it\}'s Iggy Pop.(Identificational/*Specificational)\\
(=Mary said David Bowie is Iggy Pop)
\b. Mary said of singers that John's favourite (one) is Iggy Pop. (*Identificational/Specificational)\\
($\neq$Mary said David Bowie is Iggy Pop)
\c. Mary said of John that his favourite singer is Iggy Pop. (*Identificational/Specificational)\\
($\neq$Mary said David Bowie is Iggy Pop)
\d. Mary said of people's favourite singers that John's is Iggy Pop. (*Identificational/Specificational)\\
($\neq$Mary said David Bowie is Iggy Pop)
\z.

In the above examples, Mary's claim that John's favourite singer is Iggy Pop is invariably false, but varies in the exact claim being made.
In the case that \textit{John's favourite singer} is understood \textit{de re}, Mary is wrongly identifying David Bowie as Iggy Pop.
When \textit{John's favourite singer} is understood \textit{de dicto}, Mary is wrongly specifying the singer that John prefers above all other singers is Iggy Pop.

It has been suggested to me that it is the pronominal subject of \Last[a] that forces its identificational reading.
While I am not prepared to concede this point, even if it were true, we are left with \Last[b]--\Last[d] which cannot be captured by this claim.
If pronomial subjects forced Identificational readings, the reverse could not be true, as most SCs with full (definite) DP subjects are ambiguous with identificational readings.
If we were to apply this hypothesis to \Last[b]--\Last[d] it would be non-predictive, so we would need a further explanation for the fact that specificational readings are forced when only part of the subject is an aboutness topic as in \Last[b]--\Last[d].

So, absent any compelling argument otherwise, it seems that while some part of an SC subject can be an aboutness topic, the entire subject DP cannot be the aboutness topic.

\subsubsection{Summary}
Since SC subjects are compatible with CT marking and cannot be givenness or aboutness topics, it is reasonable to assume that the presence of CT is necessary for SC subjects.
\subsection{SC subjects cannot entirely be contrastive topics}
The second claim of my proposal is that SC subjects cannot be CT-marked constituents.
So, if the entirety of the SC subject is new/contrastive, the SC is unacceptable.

\ex.
\a.
\a.[A:] Tell me about your home university?
\b.[B:]\# A BUILDING on campus no-one likes is Robarts.
\z.

If SC subjects must minimally contain a CT marked constituent, it follows directly from the unacceptibility of simple indefinite SC subjects that SC subject DPs cannot be CT-marked.
Consider the unacceptable SC \textit{*A doctor is Mary}.
The subject \textit{a doctor} must contain a CT-marked constituent, in this case \textit{doctor}.
Since the indefinite article does not encode any particular information, CT marking on the nominal is equivalent to CT marking on the entire DP.

It is worth noting here that indefinite articles can be CT-marked when a definiteness contrast is relevant in a discourse.
In these cases, simple indefinites can be SC subjects.
\ex. Who is the guitarist?\\
$[$ej$]$ guitarist is John.

So, simple indefinites can be SC subjects if they contain but do not comprise a CT-marked constituent.
\subsection{\textit{One} and \textit{another}}
As mentioned in above The determiner-like elements \textit{one} and \textit{another} can serve as CTs in SC subjects.
\ex.
\a.* A doctor$_{CT}$ is Mary.
\b.\label{ex:OneCT} One$_{CT}$ doctor is Mary.
\b. Another$_{CT}$ doctor is Mary.

If \textit{one} and \textit{another} can be CT marked, they must be able to generate alternatives, but what are these alternatives?

Let's consider \textit{another} first.
Following \textcite{heim1991reciprocity}, I take the meaning of  \textit{other} to include two crucial parts: anaphoricity and distinctness.
Consider the sentence in \Next.
\ex. Alice met with another student.

This sentence presupposes that there is a previously mentioned student (anaphoricity) and asserts that the student Alice met with is distinct from the presupposed antecedent (distinctness). 
As we can see from \Next, the anaphoricity projects when embedded, but the distinctness does not.
\ex.
\a. Alice didn't meet with another student.\\
$\implies$
\b. If Alice met with another student, she would have told us.
\b. Alice probably met with another student.
\b. Johan thought that Alice met with another student.

The sentence in \Last[c] can be felicitously uttered only in discourse contexts in which other doctors have been discussed and identified.
\ex.
\a. Let me tell you about doctors.\\
\#Another doctor is Mary.
\b. Molly is a doctor.\\
Another doctor is Mary.



Given these judgements the CT-alternatives generated by \textit{Another doctor is Mary} seem to be equivalent to the question \textit{Who is a doctor?} as in the d-tree below in \Next.

\ex. 
\begin{forest}
  tree defaults
  [Who is a doctor?
    [\textbf{Molly is a doctor.}]
    [Who is another doctor?
      [\textbf{Another doctor is Mary.}]
      [\ldots]
    ]
  ]
\end{forest}

The SC in \ref{ex:OneCT} shows the inverse felicity conditions, it requires that doctors have been discussed but none have been named.
\ex.
\a. Let me tell you about doctors.\\
One doctor is Mary.
\b. Molly is a doctor.\\
\#One doctor is Mary.

If \textit{one} is merely the stressed pronunciation of \textit{a/an}, then the account I have proposed woul likely require serious revision.
Fortunately, there are good reasons doubt that \textit{one} and \textit{a/an} are distinct lexical items.
First, it is unlikely that \textit{one} is the stressed version of \textit{a/an}, since \textit{a/an} has another stressed version pronounced [ej]/[\ae{}n], which usually marks a contrast of definiteness.
\ex.
\a.[A:] Are you the professor?
\b.[B:] I'm [ej] professor.

Also, \textcite{kayne2015one} presents several pieces of evidence that \textit{one} is lexically distinct from \textit{a/an}.
While \textit{a/an NP}  can be interpreted as generic, \textit{one NP} cannot
\ex.
\a. A spider has eight legs and many eyes. (generic/specific)
\b. One spider has eight legs and many eyes. (*generic/specific)\hfill\parencite{kayne2015one}

He also notes that the syntactic distribution of \textit{a/an} differs from \textit{one} as shown below.
\ex.
	\a. 
		\a. too long a book
		\b.* too long one book
		\z.
	\b.
		\a. a few books
		\b.* one few books
		\z.
	\b.
		\a.* They're selling a-drawer desks in the back of the store.
		\b. They're selling one-drawer desks in the back of the store.
		\z.\hfill\parencite{kayne2015one}

While Kayne argues that \textit{one} is a complex determiner that is composed of \textit{a/an} and a singular classifier, the specific syntax/semantics of \textit{one} is beyond the scope of this paper.
What is important is that \textit{one} is semantically richer than \textit{a/an} so that it can be CT-marked.

For the purpose of illustration, I will assume \posscite{kayne2015one} analysis of \textit{one} is correct.
What does this predict about the CT interpretation of \textit{One doctor is Mary}, then?
The focus interpretation is the question \textit{Who is one doctor?}, and the CT interpretation would be \textit{Who is a doctor?}, as shown below in \Next.
\ex.
\begin{forest}
  tree defaults
  [Who is a doctor?
    [\(\cdots\)]
    [Who is one doctor?
      [\textbf{One doctor is Mary}]
      [\(\cdots\)]
    ]
  ]
\end{forest}


According to Kayne's analysis, the question \textit{Who is one doctor?} is only answerable with singular and therefore, its alternative questions ask for plural answers.

\ex.
\begin{forest}
  tree defaults
  [$\left\{ x \in D_e | doctor(x) \right\}$
    [$\left\{ x \in D_e | doctor(x) \wedge \neg \textsc{sing}(x) \right\}$]
    [$\left\{ x \in D_e | doctor(x) \wedge \textsc{sing}(x) \right\}$]
  ]
\end{forest}

%\ex.
%\begin{forest}
%  tree defaults
%  [{$
%    \begin{Bmatrix}
%      \text{Mary}\oplus\text{Molly}\oplus\text{John,}\\
%      \text{Mary}\oplus\text{Molly,}\\
%      \text{Molly}\oplus\text{John,}\\
%      \text{Mary}\oplus\text{Molly,}\\
%      \text{Mary, Molly, John}
%    \end{Bmatrix}
%  $}
%  [{$
%    \begin{Bmatrix}
%      \text{Mary}\oplus\text{Molly}\oplus\text{John,}\\
%      \text{Mary}\oplus\text{Molly,}\\
%      \text{Molly}\oplus\text{John,}\\
%      \text{Mary}\oplus\text{Molly}
%    \end{Bmatrix}
%  $}]
%  [{$
%    \begin{Bmatrix}
%      \text{Mary,}\\
%      \text{Molly,}\\
%      \text{John}\\
%    \end{Bmatrix}
%  $}]
%]
%\end{forest}
%
Again, whether this particular d-tree is correct is not vital for my proposal, just that some d-tree consistent with the one in \LLast is correct.

%It is often assumed that stressed \textit{one} is actually a covert partitive (\textit{one of the \ldots}), though it does not behave quite as such.
%For instance \textit{one of the Xs} presupposes that the set of X's is non-singleton, while \textit{one X} merely implies it.
%\ex.
%\a. ONE ugly building on campus is Robarts. In fact, it's the only one.
%\b. One of the ugly buildings on campus is Robarts. \# In fact, it's the only one.

\subsection{Summary}
In this section I have presented evidence that the restriction on indefinite SC subjects comes from a requirement that SC subjects contain but not be CT marked constituents.
I first showed that \textit{contrastive} rather than aboutness or givenness topichood is the source of the restriction.
I then argued that the ban on simple indefinite SC subjects is neatly predicted if the SC subject is banned from being the CT marked constituent.
In the next section, I will argue that the indefinite restriction, in fact, can be derived from a more general constraint on CT-Foc structures that is implicit in the the literature on CTs.
\end{document}
