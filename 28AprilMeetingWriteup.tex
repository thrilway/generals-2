%        File: 28AprilMeetingWriteup.tex
%     Created: Tue Apr 28 04:00 PM 2015 E
% Last Change: Tue Apr 28 04:00 PM 2015 E
%
% arara: pdflatex
% arara: bibtex
% arara: pdflatex
% arara: pdflatex
\documentclass[letterpaper]{article}

\usepackage[margin=1in]{geometry}
\usepackage[backend=bibtex,style=authoryear]{biblatex}

\usepackage{forest}

\usepackage{linguex}

\bibliography{GP2}

\begin{document}
Consider the increasing acceptability of the following specificational clauses:
\ex.
\a.\label{ex:bad}\# A figure is Eric Lenneberg.
\b.\label{ex:so-so}\#? A figure in the history of Generative Grammar is Eric Lenneberg.
\b.\label{ex:good} An underrated figure in the history of Generative Grammar is Eric Lenneberg.
\z.

What properties of the indefinite DP in \ref{ex:good} makes it an acceptable subject?
It is not being interpreted as a definite description, as it lacks a presupposition of uniqueness.
\ex.
\a. An underrated figure in the history of Generative Grammar is Eric Lenneberg.\\
Another one is Tanya Reinhart.
\b. The most underrated figure in the history of Generative Grammar is Eric Lenneberg.\\
\#Another one is Tanya Reinhart.

As \textcite{mikkelsen2004specifying} argues, specificational subjects are interpreted as topics, so whatever makes the indefinite in \ref{ex:good} a good topic is what allows it to be a subject.
Following \textcite{buring1999topic,roberts2012information}, I treat the notions of topic and focus in relation to questions under discussion, formalized as discourse trees.
In a given discourse, the QUD is the root node, and foci of assertions are terminal nodes.
\begin{figure}[h]
  \centering
  \begin{forest}
    [QUD 
      [SubQ1
	[SubSubQ1
	  [Ans1]
	  [Ans2]
	  [Ans3]
	]
	[SubSubQ2
	  [Ans4]
	  [Ans5]
	]
      ]
      [SubQ23
	[Ans6]
	[Ans7]
	[Ans8]
      ]
    ]
  \end{forest}
  \caption{A sample discourse tree}
  \label{fig:dtree}
\end{figure}

The most general context in which one could felicitously utter \ref{ex:good} is a discussion of the history of Generative Grammar, so I will take this to be the QUD.
There are two possible subquestions to the QUD \textit{What is the history of Generative Grammar}:
\ex.
\a.\label{ex:wrong-subq} What are the underrated parts of the history of Generative Grammar?
\b.\label{ex:subq} Who are the figures in the history of Generative Grammar?

Since figures (along with events, etc) are inherent parts of any history, \ref{ex:subq} is the most likely subquestion to the QUD\footnote{It also seems to be the case that \ref{ex:wrong-subq} is not a well-formed question. To answer it, it seems that one would have to compare the overrated-ness of figures to that of events.}.
This leaves us with the discourse tree below.
\begin{figure}[h]
  \centering
  \begin{forest}
    [What is the history of Generative Grammar?
      [figures?
	[underrated figures?
	  [Eric Lenneberg]
	  [Tanya Reinhart]
	  [\ldots]
	]
	[controversial figures?\\
	  \ldots,align=center]
	[\ldots]
      ]
      [events\\
	\ldots,align=center]
    ]
  \end{forest}
  \caption{The Discourse tree for \ref{ex:good}}
  \label{fig:good-dtree}
\end{figure}

\end{document}
