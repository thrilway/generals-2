%        File: 28AprilMeetingWriteup.tex
%     Created: Tue Apr 28 04:00 PM 2015 E
% Last Change: Tue Apr 28 04:00 PM 2015 E
%
% arara: pdflatex
% arara: bibtex
% arara: pdflatex
% arara: pdflatex
\documentclass[letterpaper]{article}

\usepackage[margin=1in]{geometry}
\usepackage[backend=bibtex,style=authoryear]{biblatex}

\usepackage[]{amsmath}
\usepackage{stmaryrd}

\usepackage{linguex}

\bibliography{GP2}

\usepackage{forest}
\forestset{tree defaults/.style={for tree={parent anchor=south, child anchor=north},every tree node/.style={align=center,anchor=north},level/.style={sibling distance=50mm/#1},baseline}}

\begin{document}
\begin{center}
  {\large Follow-up from 28 April meeting}\\
  Dan Milway\\
\end{center}
Consider the increasing acceptability of the following specificational clauses:
\ex.
\a.\label{ex:bad}\# A figure is Eric Lenneberg.
\b.\label{ex:so-so}\#? A figure in the history of Generative Grammar is Eric Lenneberg.
\b.\label{ex:good} An underrated figure in the history of Generative Grammar is Eric Lenneberg.
\z.

What properties of the indefinite DP in \ref{ex:good} makes it an acceptable subject?
It is not being interpreted as a definite description, as it lacks a presupposition of uniqueness.
\ex.
\a. An underrated figure in the history of Generative Grammar is Eric Lenneberg.\\
Another one is Tanya Reinhart.
\b. The most underrated figure in the history of Generative Grammar is Eric Lenneberg.\\
\#Another one is Tanya Reinhart.

As \textcite{mikkelsen2004specifying} argues, specificational subjects are interpreted as topics, so whatever makes the indefinite in \ref{ex:good} a good topic is what allows it to be a subject.
Following \textcite{buring1999topic,roberts2012information}, I treat the notions of topic and focus in relation to questions under discussion, formalized as discourse trees.
In a given discourse, the QUD is the root node, and foci of assertions are terminal nodes.
\begin{figure}[h]
  \centering
  \begin{forest}
    tree defaults
    [QUD 
      [SubQ1
	[SubSubQ1
	  [Ans1]
	  [Ans2]
	  [Ans3]
	]
	[SubSubQ2
	  [Ans4]
	  [Ans5]
	]
      ]
      [SubQ23
	[Ans6]
	[Ans7]
	[Ans8]
      ]
    ]
  \end{forest}
  \caption{A sample discourse tree}
  \label{fig:dtree}
\end{figure}

The most general context in which one could felicitously utter \ref{ex:good} is a discussion of the history of Generative Grammar, so I will take this to be the QUD.
There are two possible subquestions to the QUD \textit{What is the history of Generative Grammar}:
\ex.
\a.\label{ex:wrong-subq} What are the underrated parts of the history of Generative Grammar?
\b.\label{ex:subq} Who are the figures in the history of Generative Grammar?

Since figures (along with events, etc) are inherent parts of any history, \ref{ex:subq} is the most likely subquestion to the QUD\footnote{It also seems to be the case that \ref{ex:wrong-subq} is not a well-formed question. To answer it, it seems that one would have to compare the overrated-ness of figures to that of events.}.
This leaves us with the discourse tree below.
\begin{figure}[h]
  \centering
  \begin{forest}
    tree defaults
    [What is the history of Generative Grammar?
      [figures?
	[underrated figures?
	  [Eric Lenneberg]
	  [Tanya Reinhart]
	  [\ldots]
	]
	[controversial figures?\\
	  \ldots,align=center]
	[\ldots]
      ]
      [events?\\
	\ldots,align=center]
    ]
  \end{forest}
  \caption{The Discourse tree for \ref{ex:good}}
  \label{fig:good-dtree}
\end{figure}

So, to get from the QUD to an appropriate topic for assertion, we must make two ``implicit'' discourse moves: One restricting the question to \textit{figures} and another restricting it to \textit{underrated figures}.
Based on this I propose a preliminary condition on topics.
\ex. \textbf{Condition on Topics (to be revised):} Topics must contain two implicit discourse moves.

The fact that this condition requires counting moves makes it suspect.
To refine it, let's inspect the nature of these moves.

The first move (\textit{history} $\rightarrow$ \textit{figures in history}), is in fact a required move, because a complete answer to the QUD would inevitably involve a litany of the figures and events that are part of the history of GG.
The second move, however, is not a required move, as it is not necessary when listing the figures in a history to offer an opinion of which were underrated, or controversial, or had some other property.
This leads me to a revised condition on topics:
\ex. \textbf{Condition on Topics:} Topics must contain at least one unneccesary discourse move.

This condition must be further modified if it is to be a testable hypothesis. 
First, we must distinguish between necessary and unnecessary moves, and depending on how that contrast is resolved, we will likely have to further refine this condition.
As a preliminary attempt to define necessity with respect to discourse moves, I use the notion of \textit{informativity} as defined by \textcite{buring1999topic}.
\ex. Informativity: A move M ind discourse D is informative iff
\a. M is an assertion and the common ground prior to M does neither entail $\llbracket\text{M}\rrbracket^o$ nor it's complement, or
\b. M is a question and for at least some element p in $\llbracket\text{M}\rrbracket^o$, the common ground prior to M does neither entail p nor p's complement.
\z.

Necessary moves, then, are uninformative moves.
The QUD of figure \ref{fig:good-dtree} has the interpretation given below:
\ex.\label{qud-denote} $\llbracket\text{What is the history of GG?}\rrbracket^o =$\\
$\left\{ \left\{ x | x \text{ is a figure in the history of GG} \right\}, \left\{ y | y \text{ is an event in the history of GG} \right\},\dots \right\}$

The QUD, and therefore the common ground, then entails the question ``Who are the figures in the history of GG?'' which is therefore, uninformative.
\ex.\label{subq-denote} $\llbracket\text{Who are the figures in the history of GG?}\rrbracket^o =$\\
$\left\{ x | x \text{ is a figure in the history of GG} \right\} \in $\ref{qud-denote}

The unnecessary move, ``Who is an underrated figure in the history of GG?'', imposes a partition on the extension of \ref{subq-denote} which is not entailed by the common ground and is therefore informative.
\ex.\label{ct-denote} $\llbracket\text{Who is an underrated figure in the history of GG?}\rrbracket^o =$\\
$\left\{ x | x \text{ is a figure in the history of GG} \wedge x \text{ is underrated} \right\}$

\end{document}
