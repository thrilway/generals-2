%        File: ArgumentPoints.tex
%     Created: Fri Nov 20 01:00 PM 2015 E
% Last Change: Fri Nov 20 01:00 PM 2015 E
%
% arara: pdflatex: {options: "-draftmode"}
% arara: biber
% arara: pdflatex: {options: "-draftmode"}
% arara: pdflatex: {options: "-file-line-error-style"}
\documentclass[GPFinal]{subfiles}

\begin{document}
\begin{center}
  \textbf{Specifying why a doctor isn't Mary}\\
  Dan Milway\\
  \today
\end{center}
\ex. \textbf{The Data}
\a.
\a.* A doctor is Mary.
\b. A newly-MINTED doctor is Mary.
\z.
\b.
\a.* A linguist is Eric Lenneberg.
\b. An UNDERrated linguist is Eric Lenneberg.
\z.
\b.
\a.* A building is Robarts.
\b. A buiding NO-one likes is Robarts.
\z.

\section{Previous Accounts}
\begin{itemize}
  \item \textcite{moro1997raising}
    \begin{itemize}
      \item Specificational clauses (SCs) are inverted predication
      \item DP1 is the predicate, DP2 is the argument.
    \end{itemize}
  \item \textcite{heycockkroch1999pseudocleft}
    \begin{itemize}
      \item SCs are equational.
      \item SC subjects are type $e$.
      \item Indefinite DPs are predicative (type $\langle e,t,\rangle$).
      \item $\therefore$ *Indefinite SC subjects.
    \end{itemize}
  \item \textcite{mikkelsen2004specifying}
    \begin{itemize}
      \item Some indefinites \textit{can} be SC subjects (see \Last)
      \item SCs have a rigid information structure (DP2 = Focus)
    \end{itemize}
    \begin{itemize}
      \item SC subjects must be Topics (discourse old/given)
      \item Those indefinites that can be subjects consist of relatively old/given material
    \end{itemize}
\end{itemize}
\ex.\label{ex:MikkIS}\textbf{SCs have fixed information structure}
\a. Q: Who is the winner?\\
A1: The winner is JOHN.\hfill[Specificational]\\
A2: JOHN is the winner.\hfill[Predicational]
\b. Q: What is John?\\
A1: \#The WINNER is John.\hfill[Specificational]\\
A2: John is the WINNER.\hfill[Predicational]\\
\quelle{\parencite{mikkelsen2004specifying}}
\z.

\begin{itemize}
  \item \textcite{mikkelsen2004specifying} on simple indefinites:
    \begin{itemize}
      \item Even when simple indefinites are made sufficiently topical, they cannot be SC subjects.
    \end{itemize}
  \item \textcite{heycock2012specification}:
    \begin{itemize}
      \item The indefinite restriction is a restriction on weak DPs as SC subjects
	\begin{itemize}
	  \item weak DPs are those that are compatible with existential constructions. \textcite{milsark1974existential}
	  \item strong DPs are those that are incompatible with existential constructions.
	\end{itemize}
      \item Argument 1:
	\begin{itemize}
	  \item \Last shows focused DPs cannot be SC subjects.
	  \item In German, focused arguments cannot scramble. See \Next[b]. \parencite{lenerz1977zur}
	  \item In German, scrambled indefinite objects cannot be interpreted as weak indefinites. \parencite{dehoop1992case,diesing1992indefinites}
	\end{itemize}
      \item Argument 2:
	\begin{itemize}
	  \item Weak indefinites cannot be subjects of individual level predicates. \parencite{milsark1974existential}
	  \item A parallel pattern can be seen with respect to SCs
	\end{itemize}
      \item $\therefore$ \textit{weak} indefinites cannot be SC subjects.
    \end{itemize}
\end{itemize}
\ex. \textbf{In German, focus doesn't scramble}
  \ag. Wem hat Peter das Futter gegeben?\\
who.\textsc{dat} has Peter the.\textsc{acc} food given\\
``Who has Peter given the food?''
    \ag. Peter hat der Katze das Futter gegeben.\\
Peter has the.\textsc{dat} cat the.\textsc{acc} food given\\
``Peter has given the cat the food''\hfill[Default order]
    \bg. Peter hat das Futter der Katze gegeben.\\
Peter has the.\textsc{acc} food the.\textsc{dat} cat given\\
``Peter has given the food to the cat''\hfill[Scrambled order]
    \z.
  \bg. Was hat Peter der Katze gegeben?\\
what.\textsc{acc} has Peter the.\textsc{dat} cat given\\
``What has Peter given (to) the cat?''
    \ag. Peter hat der Katze das Futter gegeben.\\
Peter has the.\textsc{dat} cat the.\textsc{acc} food given\\
``Peter has given the cat the food''\hfill[Default order]
    \bg.\# Peter hat das Futter der Katze gegeben.\\
Peter has the.\textsc{acc} food the.\textsc{dat} cat given\\
``Peter has given the food to the cat''\hfill[Scrambled order]
    \z.
  \z.

\section{The indefinite restriction is not a restriction on \textit{weak} indefinites}
\begin{itemize}
	\item Weak indefinites can be SC subjects
\end{itemize}
\ex. 
\a. There is \textbf{a building no-one likes} on St George Street.
\b. \textbf{a building no-one likes} is Robarts.

\ex.
\a. There are \textbf{sm side-effects}.
\b. \textbf{Sm side-effects} are headaches and dizziness. 

\begin{itemize}
	\item DPs with strong determiners can't (all) be SC subjects 
\end{itemize}
\ex.
\a.? Each doctor is Mary, Bill, Sue, and John. (*Specificational)
\b.? Most early generative grammarians are Chomsky and Halle. (*Specificational)
\b.? SOME side-effects are drowsiness and blurred vision. (*Specificational)

\section{The restriction is a requirement that SC subjects contain CTs\ldots}
\begin{itemize}
  \item Observation: Licit indefinite SC subjects must have an accented part
    \begin{itemize}
      \item DP2 must be focus. See \ref{ex:MikkIS}.
      \item DP1's accent must be \posscite{jackendoff1972Ssemantics} B-Accent, a secondary conrastive accent.
      \item \textcite{buring2003d} identifies B-Accent with Contrastive Topic (CT).
      \item The accented part of an SC subject is CT-marked.
    \end{itemize}
  \item SC subjects are not givenness topics.
\end{itemize}
\ex. Many philosophers have written about the mind-body problem.
\a.\# A philosopher who has written about the mind-body problem is Chomsky.
\b. A modern philosopher who has written about the mind-body problem is Chomsky.

\begin{itemize}
  \item SC subjects are not aboutness topics.
\end{itemize}
\ex. \textbf{\posscite{reinhart1981pragmatics} test for aboutness}\\
If sentence S is about constituent X, then S is paraphrasable by the sentence \textit{They said about }X\textit{, that }S$^\prime$, where S$^\prime$ is derived by replacing X in S with a proform.

\ex. \textbf{Background:} John's favourite singer = David Bowie
\a. \textbf{Sentence:}  (Mary said that) John's favourite singer is Iggy Pop.
\a. Mary said of John's favourite singer that \{he/?it\} is Iggy Pop. (=Mary said David Bowie is Iggy Pop)
\a. Mary said of singers that John's favourite (one) is Iggy Pop. ($\neq$Mary said David Bowie is Iggy Pop)
\c. Mary said of John that his favourite singer is Iggy Pop. ($\neq$Mary said David Bowie is Iggy Pop)
\d. Mary said of people's favourite singers that John's is Iggy Pop. ($\neq$Mary said David Bowie is Iggy Pop)
\z.
\z.

\section{\ldots but not be CTs}
\begin{itemize}
  \item If the entire indefinite is novel, it cannot be an SC subject
\end{itemize}
\ex. (Tell me about campus.)\\
\#[A building on campus no-one likes]$_CT$ is Robarts.

\begin{itemize}
  \item Simple indefinites can only be CTs if they contain them 
    \begin{itemize}
      \item The indefinite article is radically underspecified (existential, predicational, generic)
      \item Cannot be given/presupposed
      \item \textit{a [doctor]}$_{CT}$ = \textit{[a doctor]}$_{CT}$
    \end{itemize}
  \item The bar on simple indefinite SCs follows from a requirement that SC subject contain but not be CTs
\end{itemize}
\section{The givenness requirement on CT-Foc structures}
\begin{itemize}
  \item In addition to new/contrastive material, as marked by focus and CT accents, CT-Foc structured utterences must contain some given material.
  \item Examples of CT-Foc structures all meet this requirement.
  \item SCs with simple indefinites do not meet this requirement.
\end{itemize}
\ex. FRED$_A$ ate the BEANS$_B$.
\a.[Focus: ] Fred
\b.[CT: ] the beans
\b.[Given: ] $x$ ate $y$

\ex. An UNDERrated linguist is Eric Lenneberg
\a.[Focus: ] Eric Lenneberg
\b.[CT: ] under
\b.[Given: ] $x$ is a linguist who is inaccurately rated in $y$ direction.

\ex. *A linguist is Eric Lenneberg.
\a.[Focus: ] Eric Lenneberg
\b.[CT: ] $x$ is a linguist
\b.[Given: ] $\emptyset$

\begin{itemize}
  \item This requirement is not phonological
    \begin{itemize}
      \item a requirement that something be deaccented is met by \Last (the copula is deaccented)
    \end{itemize}
  \item It is not formulable as a syntactic constraint
    \begin{itemize}
      \item There is no restriction on what category can be new/contrastive
      \item There is no restriction on the syntactic configuration of CT, Foc, and Given material.
    \end{itemize}
  \item The requirement cannot be solely semantic.
    \begin{itemize}
      \item There is some restriction on what can be contrastive, new, or given (indefinite articles and copulas cannot be given).
      \item \Last encodes an acceptable proposition.
    \end{itemize}
  \item The requirement must be pragmatic.
\end{itemize}
\printbibliography
\end{document}


