%        File: ArgumentPoints.tex
%     Created: Fri Nov 20 01:00 PM 2015 E
% Last Change: Fri Nov 20 01:00 PM 2015 E
%
% arara: pdflatex: {options: "-draftmode"}
% arara: biber
% arara: pdflatex: {options: "-draftmode"}
% arara: pdflatex: {options: "-file-line-error-style"}
\documentclass[GPFinal]{subfiles}

\begin{document}
\ex. \textbf{The Data}
\a.
\a.* A doctor is Mary.
\b. A newly-MINTED doctor is Mary.
\z.
\b.
\a.* A linguist is Eric Lenneberg.
\b. An UNDERrated linguist is Eric Lenneberg.
\z.
\b.
\a.* A building is Robarts.
\b. A buiding NO-one likes is Robarts.
\z.

\section{Previous Accounts}
\begin{itemize}
  \item \textcite{heycockkroch1999pseudocleft}
    \begin{itemize}
      \item SCs are equational.
      \item SC subjects are type $e$.
      \item Indefinite DPs are predicative (type $\langle e,t,\rangle$).
      \item $\therefore$ *Indefinite SC subjects.
    \end{itemize}
  \item \textcite{mikkelsen2004specifying}
    \begin{itemize}
      \item Some indefinites \textit{can} be SC subjects (see \Last)
      \item SCs have a rigid information structure (DP2 = Focus)
    \end{itemize}
    \begin{itemize}
      \item SC subjects must be Topics (discourse old/given)
      \item Those indefinites that can be subjects consist of relatively old/given material
    \end{itemize}
\end{itemize}
\ex.\textbf{SCs have fixed information structure}
\a. Q: Who is the winner?\\
A1: The winner is JOHN.\hfill[Specificational]\\
A2: JOHN is the winner.\hfill[Predicational]
\b. Q: What is John?\\
A1: \#The WINNER is John.\hfill[Specificational]\\
A2: John is the WINNER.\hfill[Predicational]\\
\quelle{\parencite{mikkelsen2004specifying}}
\z.
\begin{itemize}
  \item \textcite{mikkelsen2004specifying} on simple indefinites:
    \begin{itemize}
      \item Even when simple indefinites are made sufficiently topical, they cannot be SC subjects.
      \item Mikkelsen 
    \end{itemize}<++>
  \item \textcite{heycockkroch1999pseudocleft}:
    \begin{itemize}
      \item Parallel 1:
	\begin{itemize}
	  \item \Last shows focused DPs cannot be SC subjects.
	  \item In German, focused arguments cannot scramble. \parencite{lenerz1977zur}
	  \item In German, scrambled indefinite objects cannot be interpreted as weak indefinites. \parencite{dehoop1992case,diesing1992indefinites}
	\end{itemize}
      \item Parallel 2:
	\begin{itemize}
	  \item Weak indefinites cannot be subjects of individual level predicates. \parencite{milsark1974existential}
	  \item A parallel pattern can be seen with respect to SCs
	\end{itemize}
      \item $\therefore$ \textit{weak} indefinites cannot be SC subjects.
    \end{itemize}
\end{itemize}
\ex. \textbf{In German, focus doesn't scramble}
\ag.Wem hat Peter das Futter gegeben?\\
who.\textsc{dat} has Peter the.\textsc{acc} food given\\
``Who has Peter given the food?''
\ag. Peter hat der Katze das Futter gegeben.\\
Peter has the.\textsc{dat} cat the.\textsc{acc} food given\\
``Peter has given the cat the food''\hfill[Default order]
\bg. Peter hat das Futter der Katze gegeben.\\
Peter has the.\textsc{acc} food the.\textsc{dat} cat given\\
``Peter has given the food to the cat''\hfill[Scrambled order]
\z.
\bg. Was hat Peter der Katze gegeben?\\
what.\textsc{acc} has Peter the.\textsc{dat} cat given\\
``What has Peter given (to) the cat?''
\ag. Peter hat der Katze das Futter gegeben.\\
Peter has the.\textsc{dat} cat the.\textsc{acc} food given\\
``Peter has given the cat the food''\hfill[Default order]
\bg.\# Peter hat das Futter der Katze gegeben.\\
Peter has the.\textsc{acc} food the.\textsc{dat} cat given\\
``Peter has given the food to the cat''\hfill[Scrambled order]
\z.
\z.

\section{The givenness requirement on CT-Foc structures}
\begin{itemize}
  \item In addition to new/contrastive material, as marked by focus and CT accents, CT-Foc structured utterences must conatian some given material.
  \item Examples of CT-Foc structures all meet this requirement.
  \item SCs with simple indefinites do not meet this requirement.
\end{itemize}
\ex. FRED$_A$ ate the BEANS$_B$.
\a.[Focus: ] Fred
\b.[CT: ] the beans
\b.[Given: ] $x$ ate $y$

\ex. An UNDERrated linguist is Eric Lenneberg
\a.[Focus: ] Eric Lenneberg
\b.[CT: ] under
\b.[Given: ] $x$ is a linguist who is inaccurately rated in $y$ direction.

\ex. *A linguist is Eric Lenneberg.
\a.[Focus: ] Eric Lenneberg
\b.[CT: ] $x$ is a linguist
\b.[Given: ] $\emptyset$

\begin{itemize}
  \item This requirement is not phonological
    \begin{itemize}
      \item a requirement that something be deaccented is met by \Last (the copula is deaccented)
    \end{itemize}
  \item It is not formulable as a syntactic constraint
    \begin{itemize}
      \item There is no restriction on what category can be new/contrastive
      \item There is no restriction on the syntactic configuration of CT, Foc, and Given material.
    \end{itemize}
  \item The requirement cannot be solely semantic.
    \begin{itemize}
      \item There is some resrtiction on what can be contrastive, new, or given (indefinite articles and copulas cannot be given).
      \item \Last encodes an acceptable proposition.
    \end{itemize}
  \item The requirement must be pragmatic.
\end{itemize}
\end{document}


