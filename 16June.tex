%        File: 16June.tex
%     Created: Mon Jun 15 11:00 AM 2015 E
% Last Change: Mon Jun 15 11:00 AM 2015 E
%
% arara: pdflatex
% arara: bibtex
% arara: pdflatex
% arara: pdflatex
\documentclass[letterpaper]{article}

%\usepackage[utf8]{inputenc}

\usepackage[margin=1in]{geometry}
\usepackage[backend=bibtex,style=authoryear]{biblatex}

\usepackage{linguex}

\usepackage{stmaryrd}
\usepackage[]{amsmath}
\usepackage{forest}

\forestset{tree defaults/.style={for tree={parent anchor=south, child anchor=north},every tree node/.style={align=center,anchor=north},level/.style={sibling distance=50mm/#1},baseline}}

\forestset{en/.style={parent anchor=center, child anchor=center}}
\forestset{em/.style={parent anchor=north west, child anchor=north west}}

\usetikzlibrary{positioning}


\bibliography{GP2}

\begin{document}

\begin{itemize}
  \item \textcite{rooth1992theory} discusses four focus-related effects.
    \begin{itemize}
      \item Focusing Adverbs\\
	\textit{e.g.}, \textit{only}
      \item Contrast\\
	An AMERICAN farmer spoke to a CANADIAN farmer.
      \item Scalar Implicatures
      \item Questions and answers
    \end{itemize}
  \item Based in alternative semantics
    \begin{itemize}
      \item $\llbracket \text{S}\rrbracket^o = $ the truth conditions of S
      \item $\llbracket \text{S}\rrbracket^f = $ a set of propositions.
      \item $\llbracket \text{S}\rrbracket^o \in\llbracket \text{S}\rrbracket^f$
    \end{itemize}
  \item B\"uring's discourse trees are derived from alternative semantics so it should be able to handle Rooth's focus effects.
  \item Suppose S represents a single discourse move.
  \item $\llbracket \text{S}\rrbracket^o$ is a node $\alpha$.
  \item $\llbracket \text{S}\rrbracket^f$ is a node $\gamma$, that dominates $\alpha$
  \item there is a node $\beta$, immediately dominated by $\gamma$
  \item $\{\alpha\} \cap \beta = \emptyset$
  \item $\{\alpha\} \cup \beta \subseteq \gamma$ (maybe =)
\end{itemize}
\ex. 
\begin{forest}
  tree defaults
  [$\gamma$
    [$\alpha$]
    [$\beta$]
  ]
\end{forest}

\textbf{Adverbs}
\begin{itemize}
  \item According to Rooth, the adverbs quantify over a subset of the focus interpretation.
  \item \textcite{beaver2003always} argue that not all focus adverbs are alike.
    \begin{itemize}
      \item \textit{only} is a simple quantifier.
      \item \textit{always} introduces context dependent variables.
    \end{itemize}
  \item Focus adverbs modify/specify parts of the discourse tree 
  \item \textit{e.g.}, \textit{only} asserts that $\beta$ is empty.
\end{itemize}
\ex. Mary only \dots
\a.\label{bill-foc} introduced BILL to Sue.
\b.\label{sue-foc} introduced Bill to SUE.

\ex.
\a.
\begin{forest}
  tree defaults
  [{$\left\{ x \in D_e | \text{introduced}(\textbf{m},x,\textbf{s}) \right\}$}
    [\textbf{b}]
    [$\emptyset$]
  ]
\end{forest}
\b.
\begin{forest}
  tree defaults
  [{$\left\{ y \in D_e | \text{introduced}(\textbf{m},\textbf{b},y) \right\}$}
    [\textbf{s}]
    [$\emptyset$]
  ]
\end{forest}

\textbf{Contrasting phrases}
\begin{itemize}
  \item This phenomenon seems to be clause-bound
  \item D-trees do not seem applicable.〉
\end{itemize}
\textbf{Scales}
\ex. Context: Steve, Paul and I took a quiz.\\
A: How did it go?
\a. Well, I \textsc{passed}.
\b. Well, [\textsc{i}] passed.

\begin{itemize}
  \item Rooth: The set from which the scale is constructed, is a subset of the focus value.
\end{itemize}
\ex. $\llbracket [\text{passed}]\rrbracket^f = {\text{passed}, \text{aced}}$

\begin{itemize}
  \item $\beta$ or $\gamma$ are restricted to the scale set.
\end{itemize}
\textbf{Q-A}
\begin{itemize}
  \item Exactly what D-Trees are designed for.
\end{itemize}
\textbf{My proposal}
\begin{itemize}
  \item Focus effects are by reference to only $\alpha, \beta$ and $\gamma$ or the structural relation that holds between them.
  \item Crucially none of them seem to reference members or sub-parts of $\beta$ or $\gamma$.
\end{itemize}

\printbibliography
\end{document}


