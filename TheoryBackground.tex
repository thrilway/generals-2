%        File: TheoryBackground.tex
%     Created: Tue Sep 08 12:00 PM 2015 E
% Last Change: Tue Sep 08 12:00 PM 2015 E
%
% arara: pdflatex: {options: "-draftmode"}
% arara: biber
% arara: pdflatex: {options: "-draftmode"}
% arara: pdflatex: {options: "-file-line-error-style"}
\documentclass[GPFinal]{subfiles}

\begin{document}
The theory of Contrastive Topic (CT) that I will be assuming here is a slight modification of that of \textcite{buring2003d,buringforthcomingtopic} which itself builds off of \citeauthor{roberts2012information}' (\citeyear{roberts2012information}) formal pragmatics and \citeauthor{rooth1992theory}'s (\citeyear{rooth1992theory}) alternative semantics.

Alternative semantics, as developed by \textcite{rooth1992theory}, proposes that, in addition to ordinary interpretations ($\llbracket\cdot\rrbracket^\mathcal{O}$), sentences receive a focus interpretation ($\llbracket\cdot\rrbracket^f$) which is derived from the ordinary interpretation and the focused constituent.
Consider the following example.
\ex.\label{ex:BasicFocus} [Mary]$_F$ answered Sue.

The ordinary interpretation of this sentence is the proposition it expresses
\ex.\label{ex:OrdinaryInterpretation} $\llbracket\ref{ex:BasicFocus}\rrbracket^\mathcal{O} = [answered(\mathbf{m}, \mathbf{s})]$

The focus interpretation is the set of propositions generated by replacing the focused material with a variable.
\ex.\label{ex:FocusInterpretation} $\llbracket\ref{ex:BasicFocus}\rrbracket^f = \left\{ answered(x, \mathbf{s}) | x \in D_e \right\}$

Note that the focus semantics of \ref{ex:BasicFocus} is equivalent to the ordinary interpretation of the question \textit{Who answered Sue?} following \textcite{hamblin1973questions}. 
This relation between focus interpretation and question interpretation is key to the model of discourse I assume here.

\textcite{roberts2012information} models discourse as a cooperative game, following \textcite{lewis1979scorekeeping}, the goal of which is to answer the \textit{questions under discussion} (QUDs).
Utterences are represented as moves, with questions being setup moves and assertions being payoff moves.
At a given point in the discourse there is a QUD, and discourse proceeds either by answering that QUD or by asking a subquestion (\textit{i.e.} one whose answer is a partial answer to the QUD), which becomes the new QUD.
Roberts models the QUDs as a stack which new subquestions are pushed into and questions are popped off of upon being answered
So, a move is considered relevant based on the question at the top of the QUD stack.

\ex.[($\mathcal{D}_0$)] Who ate what?
	\a.[\texttt{a}.] What did Hilary eat?
		\a.[\texttt{i}.] Did Hilary eat bagels?\\
		Ans(\texttt{a}$_\texttt{i}$) = yes
		\b.[\texttt{ii}.]Did Hilary eat tofu?\\
		Ans(\texttt{a}$_\texttt{ii}$) = no
		\z.
	\b.[\texttt{b}.] What did Robin eat?
		\a.[\texttt{i}.]Did Robin eat bagels?\\
		Ans(\texttt{b}$_\texttt{i}$) = no
		\b.[\texttt{ii}.]Did Robin eat tofu?\\
		Ans(\texttt{b}$_\texttt{ii}$) = yes
		\z.
	\z.

Note, that this discourse goes beyond the explicitness we see in natural speech.
For example, when question (a) is asked, we don't require that (a$_\text{i}$) and (a$_\text{ii}$) are asked so that we may answer \textit{yes} or \textit{no}.
Instead we can answer with an assertion that includes a focused constituent that matches the wh-word of the QUD.
\ex. A: \texttt{a}/\#\texttt{b}\\
B: Hilary ate [bagels]$_F$.

To ensure that an assertion is used felicitously, Roberts exploits the fact that focus interpretations of assertions are of the same type as question interpretations.
An assertion, like that in \Last is felicitous if its focus interpretation is equal to the interpretation of the QUD.
\ex.
\a.
$\llbracket\text{Hilary ate [bagels]}_F.\rrbracket^f = 
\begin{Bmatrix}
  \text{Hilary ate bagels.}\\
  \text{Hilary ate tofu.}
\end{Bmatrix}
$
\b.
$\llbracket\text{What did Hilary eat?}\rrbracket^\mathcal{O} =
\begin{Bmatrix}
  \text{Hilary ate bagels.}\\
  \text{Hilary ate tofu.}
\end{Bmatrix}
$\hfill (=\Last[a])
\b.
$\llbracket\text{What did Robin eat?}\rrbracket^\mathcal{O} =
\begin{Bmatrix}
  \text{Robin ate bagels.}\\
  \text{Robin ate tofu.}
\end{Bmatrix}
$\hfill ($\neq$\Last[a])


\textcite{buring2003d} represents Roberts' structured discourses as \textit{d(iscourse)-trees}.
The discourse $\mathcal{D}_0$, then is represented by the tree below.
\ex.
\begin{forest}
  tree defaults
  [question\\Who ate what?,for tree={align=center}
    [subquestion\\\texttt{a}
      [subsubquestion\\\texttt{a}$_\texttt{i}$
	[Ans(\texttt{a}$_\texttt{i}$)]
      ]
      [subsubquestion\\\texttt{a}$_\texttt{ii}$
	[Ans(\texttt{a}$_\texttt{ii}$)]
      ]
    ]
    [subquestion\\\texttt{b}
      [subsubquestion\\\texttt{b}$_\texttt{i}$
	[Ans(\texttt{b}$_\texttt{i}$)]
      ]
      [subsubquestion\\\texttt{b}$_\texttt{ii}$
	[Ans(\texttt{b}$_\texttt{ii}$)]
      ]
    ]
  ]
\end{forest}

\end{document}


