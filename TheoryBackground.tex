%        File: TheoryBackground.tex
%     Created: Tue Sep 08 12:00 PM 2015 E
% Last Change: Tue Sep 08 12:00 PM 2015 E
%
% arara: pdflatex: {options: "-draftmode"}
% arara: biber
% arara: pdflatex: {options: "-draftmode"}
% arara: pdflatex: {options: "-file-line-error-style"}
\documentclass[GPFinal]{subfiles}

\begin{document}
The theory of Contrastive Topic (CT) that I will be assuming here is a slight modification of that of \textcite{buring2003d,buringforthcomingtopic} which itself builds off of \citeauthor{roberts2012information}' (\citeyear{roberts2012information}) formal pragmatics and \citeauthor{rooth1992theory}'s (\citeyear{rooth1992theory}) alternative semantics.

Alternative semantics, as developed by \textcite{rooth1992theory}, proposes that, in addition to ordinary interpretations ($\llbracket\cdot\rrbracket^\mathcal{O}$), sentences receive a focus interpretation ($\llbracket\cdot\rrbracket^f$) which is derived from the ordinary interpretation and the focused constituent.
Consider the following example.
\ex.\label{ex:BasicFocus} [Mary]$_F$ answered Sue.

The ordinary interpretation of this sentence is the proposition it expresses
\ex.\label{ex:OrdinaryInterpretation} $\llbracket\ref{ex:BasicFocus}\rrbracket^\mathcal{O} = [answered(\mathbf{m}, \mathbf{s})]$

The focus interpretation is the set of propositions generated by replacing the focused material with a variable.
\ex.\label{ex:FocusInterpretation} $\llbracket\ref{ex:BasicFocus}\rrbracket^f = \left\{ answered(x, \mathbf{s}) | x \in D_e \right\}$

Note that the focus semantics of \ref{ex:BasicFocus} is equivalent to the ordinary interpretation of the question \textit{Who answered Sue?} following \textcite{hamblin1973questions}. 
This relation between focus interpretation and question interpretation is key to the model of discourse I assume here.

\textcite{roberts2012information} models discourse as a cooperative game, following \textcite{lewis1979scorekeeping}, the goal of which is to answer the \textit{questions under discussion} (QUDs).
Utterences are represented as moves, with questions being setup moves and assertions being payoff moves.

\ex. Who ate what?
	\a. What did Hilary eat?
		\a.\label{ex:HilaryBagels} Did Hilary eat bagels\\
		Ans\ref{ex:HilaryBagels} = yes
		\b.\label{ex:HilaryTofu} Did Hilary eat tofu\\
		Ans\ref{ex:HilaryTofu} = yes
		\z.
	\b. What did Robin eat?
		\a.\label{ex:RobinBagels} Did Robin eat bagels\\
		Ans\ref{ex:RobinBagels} = yes
		\b.\label{ex:RobinTofu} Did Robin eat tofu\\
		Ans\ref{ex:RobinTofu} = yes
		\z.
	\z.

\end{document}


