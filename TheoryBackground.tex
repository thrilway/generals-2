%        File: TheoryBackground.tex
%     Created: Tue Sep 08 12:00 PM 2015 E
% Last Change: Tue Sep 08 12:00 PM 2015 E
%
% arara: pdflatex: {options: "-draftmode"}
% arara: biber
% arara: pdflatex: {options: "-draftmode"}
% arara: pdflatex: {options: "-file-line-error-style"}
\documentclass[GPFinal]{subfiles}

\begin{document}
The notion of Contrastive Topic (CT) as discussed by \textcite{buring2003d,buringforthcomingtopic} is central to the analysis I will propose for the indefinite restriction  
B\"uring's theory itself builds off of \citeauthor{roberts2012information}' (\citeyear{roberts2012information}) formal pragmatics and \citeauthor{rooth1992theory}'s (\citeyear{rooth1992theory}) alternative semantics account of focus.
In the two following subsections, I will introduce the latter two theories before introducing B\"uring's CT-theory in section \ref{sec:BuringCT}.

\subsection{Alternative Semantics \parencite{rooth1992theory}}
Alternative semantics, as developed by \textcite{rooth1992theory}, proposes that, in addition to ordinary interpretations ($\llbracket\cdot\rrbracket^\mathcal{O}$), sentences receive a focus interpretation ($\llbracket\cdot\rrbracket^f$) which is derived from the ordinary interpretation and the focused constituent.
Consider the following example.
\ex.\label{ex:BasicFocus} [Mary]$_F$ answered Sue.

The ordinary interpretation of this sentence is the proposition it expresses
\ex.\label{ex:OrdinaryInterpretation} $\llbracket\ref{ex:BasicFocus}\rrbracket^\mathcal{O} = [answered(\mathbf{m}, \mathbf{s})]$

The focus interpretation is the set of propositions generated by replacing the focused material with a variable.
\ex.\label{ex:FocusInterpretation} $\llbracket\ref{ex:BasicFocus}\rrbracket^f = \left\{ answered(x, \mathbf{s}) | x \in D_e \right\}$

Note that the focus semantics of \ref{ex:BasicFocus} is equivalent to the ordinary interpretation of the question \textit{Who answered Sue?} following \textcite{hamblin1973questions}. 
This relation between focus interpretation and question interpretation is key to the model of discourse I assume here.

\subsection{Discourse Pragmatics \parencite{roberts2012information}}
\textcite{roberts2012information} models discourse as a cooperative game, following \textcite{lewis1979scorekeeping}, the goal of which is to answer the \textit{questions under discussion} (QUDs).
Utterences are represented as moves, with questions being setup moves and assertions being payoff moves.
At a given point in the discourse there is a QUD, and discourse proceeds either by answering that QUD or by asking a subquestion (\textit{i.e.} one whose answer is a partial answer to the QUD), which becomes the new QUD.
Roberts models the QUDs as a stack which new subquestions are pushed into and questions are popped off of upon being answered
So, a move is considered (ir)relevant based on the question at the top of the QUD stack.

Roberts' model of a particular discourse is given below as a series of questions, subquestions, and answers. 
\ex.[($\mathcal{D}_0$)] Who ate what?
	\a.[\texttt{a}.] What did Hilary eat?
		\a.[\texttt{i}.] Did Hilary eat bagels?\\
		Ans(\texttt{a}$_\texttt{i}$) = yes
		\b.[\texttt{ii}.]Did Hilary eat tofu?\\
		Ans(\texttt{a}$_\texttt{ii}$) = no
		\z.
	\b.[\texttt{b}.] What did Robin eat?
		\a.[\texttt{i}.]Did Robin eat bagels?\\
		Ans(\texttt{b}$_\texttt{i}$) = no
		\b.[\texttt{ii}.]Did Robin eat tofu?\\
		Ans(\texttt{b}$_\texttt{ii}$) = yes
		\z.
	\z.

Note, that this discourse goes beyond the explicitness we see in natural speech.
For example, when question (a) is asked, we don't require that (a$_\text{i}$) and (a$_\text{ii}$) are asked so that we may answer \textit{yes} or \textit{no}.
Instead we can answer with an assertion that includes a focused constituent that matches the wh-word of the QUD.
\ex. A: \texttt{a}/\#\texttt{b}\\
B: Hilary ate [bagels]$_F$.

To ensure that an assertion is used felicitously, Roberts exploits the fact that focus interpretations of assertions are of the same type as question interpretations.
An assertion, like that in \Last is felicitous if its focus interpretation is equal to the interpretation of the QUD.
\ex.
\a.
$\llbracket\text{Hilary ate [bagels]}_F.\rrbracket^f = 
\begin{Bmatrix}
  \text{Hilary ate bagels.}\\
  \text{Hilary ate tofu.}
\end{Bmatrix}
$
\b.
$\llbracket\text{What did Hilary eat?}\rrbracket^\mathcal{O} =
\begin{Bmatrix}
  \text{Hilary ate bagels.}\\
  \text{Hilary ate tofu.}
\end{Bmatrix}
$\hfill (=\Last[a])
\b.
$\llbracket\text{What did Robin eat?}\rrbracket^\mathcal{O} =
\begin{Bmatrix}
  \text{Robin ate bagels.}\\
  \text{Robin ate tofu.}
\end{Bmatrix}
$\hfill ($\neq$\Last[a])

Roberts goes on to address contrastive topics, which she refers to as \textit{dependent focus}, in much the same way as she treats focus.
Structures with CT and focus are given focus interpretation, meaning they are interpreted as a set of alternatives under alternative semantics.
An example of a CT-F utterance and its focus interpretation is given below in \Next.
\ex.
\a. [Hilary]$_{CT}$ ate [bagels]$_F$.
\b. $\left\{ x\text{ ate }y | x,y \in D_e \right\}$

This suggests that \Last[a] presupposes the question in \Last[b] (\textit{Who ate what?}), a proposal that Roberts shows does not hold up to further scrutiny.
This hypothesis predicts that \Last[a] ought to have the same felicity conditions if its CT and F marking were reversed as in \Next below.
\ex. 
\a. [Hilary]$_F$ ate [bagels]$_{CT}$.
\b. $\left\{ x\text{ ate }y | x,y \in D_e \right\}$

Roberts suggests that, rather than only presupposing a QUD, CT-F structured utterances also presuppose ``a possibly complex strategy of questions.'' \parencite[][p.50]{roberts2012information}
As Roberts acknowledges, this is a very preliminary account of the pragmatics of CT which will require further empirical and theoretical investigation.

\subsection{Contrastive Topic \parencite{buring2003d,buringforthcomingtopic}}
\textcite{buring2003d} represents Roberts' structured discourses as \textit{d(iscourse)-trees}.
The discourse $\mathcal{D}_0$, then is represented by the tree below.
\ex.
\begin{forest}
  tree defaults
  [question\\Who ate what?,for tree={align=center}
    [subquestion\\\texttt{a}
      [subsubquestion\\\texttt{a}$_\texttt{i}$
	[Ans(\texttt{a}$_\texttt{i}$)]
      ]
      [subsubquestion\\\texttt{a}$_\texttt{ii}$
	[Ans(\texttt{a}$_\texttt{ii}$)]
      ]
    ]
    [subquestion\\\texttt{b}
      [subsubquestion\\\texttt{b}$_\texttt{i}$
	[Ans(\texttt{b}$_\texttt{i}$)]
      ]
      [subsubquestion\\\texttt{b}$_\texttt{ii}$
	[Ans(\texttt{b}$_\texttt{ii}$)]
      ]
    ]
  ]
\end{forest}

B\"uring also distinguishes between the focus value ($\llbracket\cdot\rrbracket^f$) and the CT value ($\llbracket\cdot\rrbracket^{ct}$) of an utterance and defines an algorithm for determining the CT value, given below in \Next.
\ex. CT-value formation:
\a.[step 1:] Replace the focus with a \textit{wh}-word and front the latter; if focus marks the finite verb or negation, front the finite verb instead.
\b.[step 2:] Form a set of questions from the result of step 1 by replacing the contrastive topic with some alternative to it.\hfill\parencite{buring2003d}
\z.

Note, as demonstrated below, this algorithm generates a set of questions, which is a set of sets of propositions.
This way, \textcite{buring2003d} is able to build into his representations the fact that a CT-F structure presupposes a QUD and a strategy for answering it.
\ex.
\a.\label{ex:HilBagCT-F} [Hilary]$_{CT}$ ate [bagels]$_F$.
	\b. CT-value formation:
		\a.[step 1:] What did Hilary eat?
		\b.[step 2:] What did Hilary eat?\\
		What did Robin eat?
		\z.
	\b. $\llbracket$[Hilary]$_{CT}$ ate [bagels]$_F$.$\rrbracket^{ct} = \left\{ \left\{ x\text{ ate }y | y \in D_e \right\} | x \in D_e \right\}$
	\z.

Under this analysis of CT-value, the CT-F structure of an utterance is represented by the value. 
So the CT-value of \Last[a] is distinct from that \Next[a], below, which inverts the CT-F structure.
\ex.
\a.\label{ex:HilBagF-CT} [Hilary]$_F$ ate [bagels]$_{CT}$.
\b. CT-value formation:
\a.[step 1:] Who ate bagels?
\b.[step 2:] Who ate bagels?\\
Who ate tofu?
\z.
\b. $\llbracket$[Hilary]$_F$ ate [bagels]$_{CT}$.$\rrbracket = \left\{ \left\{ x\text{ ate }y | x \in D_e \right\} y \in D_e \right\}$\hfill($\neq\llbracket\LLast[a]\rrbracket^{ct}$)
\z.

The nested nature of these CT-values, makes them directly translatable into d-trees which I provide below.
\ex.
\a. \begin{forest}
  tree defaults
  [{$\llbracket\ref{ex:HilBagCT-F}\rrbracket^{ct}$}
    [What did Robin eat?]
    [What did Hilary eat
      [Hilary ate bagels]
    ]
  ]
\end{forest}
\b.
\begin{forest}
  tree defaults
  [{$\llbracket\ref{ex:HilBagF-CT}\rrbracket^{ct}$}
    [Who ate tofu?]
    [Who ate bagels
      [Hilary ate bagels]
    ]
  ]
\end{forest}

D-trees provide a perspicuous way of representing varios aspects of discourse structure in a way that leverages a vocabulary already used by generative linguists.
They allow us to define pragmatic notions such as assertions, questions, alternatives, \textit{etc} in terms of nodes, sisterhood, dominance, \textit{etc.}
For instance, assertions and questions are distinguished by the fact that the former are terminal nodes while the latter are non-terminal.

It should be noted that CT-F structures are used in a variety of discourse contexts to acheive subtly different conversational goals.
Consider the following examples.
\ex.\label{ex:ChinaCTF}
\a.[A:] When are you going to China? \hfill \parencite{roberts2012information}
\b.[B:] I'm going to [China]$_{CT}$ in [April]$_F$.
\z.

\ex.\label{ex:CaftansCTF}
\a.[A:] What did the pop stars wear? \hfill \parencite{buring2003d}
\b.[B:] The [female]$_{CT}$ popstars wore [caftans]$_F$.
\z.

\ex.\label{ex:DoctorChiroCTF}
\a.[A:] Who's a good doctor to treat my back pain?
\b.[B:] [My sister Monica]$_{F}$ is a [chiropractor]$_{CT}$.
\z.

All of these instances of CT-F structures signal what B\"uring calls \textit{implicit moves}, each instance has a different sort of implicit move that can be easily represented by its d-tree.
In \ref{ex:ChinaCTF} the assertion directly answers the question, but implies the existence of a relevant superquestion (\textit{When are you going to which place?}).
The d-tree in \Next shows this by marking the explicit moves in bold.
\ex.
\begin{forest}
  tree defaults
  [When are you going which place?
    [When are you going to \ldots?]
    [\textbf{When are you going to China?}
      [\textbf{April}]
    ]
  ]
\end{forest}

The assertion in \ref{ex:CaftansCTF}, on the other hand, does not answer the explicit question, but instead answers an implied subquestion (\textit{What did the female pop-stars wear}).
Again this can be represented clearly in the d-tree in \Next.
\ex.
\begin{forest}
  tree defaults
  [\textbf{What did the pop stars wear?}
    [What did the male pop stars wear?]
    [What did the female pop stars wear?
      [\textbf{The female pop stars wore caftans.}]
    ]
  ]
\end{forest}

Finally, the assertion in \ref{ex:DoctorChiroCTF} answers neither the explicit question, nor an implied subquestion.
Instead, it answers an implicit subquestion of a superquestion of the explicit question, as we can see in its d-tree in \Next.
\ex.
\begin{forest}
  tree defaults
  [Who's a good $\langle profession\rangle$ to treat back pain?
    [\textbf{Who's a good doctor \ldots?}]
    [Who's a good \ldots?]
    [Who's a good chiropractor \ldots?
      [\textbf{Monica}]
    ]
  ]
\end{forest}

So, although a given CT-F structure can be mapped onto a single d-tree in a predictable way, context in which it is uttered determines its place in and effect on the discourse.
Implicit in \textcite{buring2003d} is an informal condition on CT felicity which I give in \Next.
\ex. M is a move that uses a CT-F structure.\\
Q is a question.\\
M is felicitous in the context of the QUD Q iff the M defines a d-tree DT such that Q is represented in DT.

Though informal, this condition can effectively rule out several examples of infelicitous CT-F structures.
\ex.\label{ex:HilBagelInfel} 
\a.
\a.[A:] Who ate bagels?
\b.[B:] \#[Hilary]$_{CT}$ ate [bagels]$_F$.
\z.
\b. $\llbracket$[Hilary]$_{CT}$ ate [bagels]$_F\rrbracket^{ct}$\\
\begin{forest}
  tree defaults
  [What did who eat?
    [What did Robin eat?]
    [What did Hilary eat?
      [\textbf{Hilary ate bagels}]
    ]
  ]
\end{forest}
\z.

\ex.
\a.
\a.[A:] Who's a good doctor to treat my back pain?
\b.[B:]\# [My sister Monica]$_{CT}$ is a [chiropractor]$_{F}$
\z.
\b.
\begin{forest}
  tree defaults
  [Who's a good $\langle profession\rangle$ to treat back pain?
    [Monica's a good $\langle profession\rangle$ to treat back pain?
      [\textbf{Monica's a good chiropractor}]
    ]
    [Joe's a good $\langle profession\rangle$ to treat back pain?
      [\ldots]
    ]
  ]
\end{forest}

<<<<<<< Updated upstream
\subsection{Implied questions are mention-all questions}
Consider the implied questions generated by the CT-Foc structure of \ref{ex:ChinaCTF} and \ref{ex:CaftansCTF}, reproduced below in \Next and \NNext.
\ex.
\a.[Utterance: ] I'm going to [China]$_{CT}$ in [April]$_F$.
\b.[Explicit Q: ] When are you going to China?
\b.[Implied Q: ] When are you going where? (superquestion)

\ex.
\a.[Utterance: ] The [female]$_{CT}$ popstars wore [caftans]$_F$
\b.[Explicit Q: ] What did the popstars wear?
\b.[Implied Q: ] What did the female popstars wear? (subquestion) 

In these and othe cases, the CT marking triggers an exhaustivity implicature.
Specifically, the utterance in \LLast implies that the speaker is going nowhere else in April and the utterance in \Last implies that the male popstars didn't also wear caftans.
=======
\subsection{CT-marking, implied questions, and exhaustivity}
CT-Marking also seems to trigger an exhaustivity implicature.
Consider the CT marked sentences in \ref{ex:ChinaCTF} and \ref{ex:CaftansCTF} and the sorts of discourse that can and cannot follow them.
\ex. I'm going to [China]$_{CT}$ in April.
\a.\# And I'm going to [Ireland]$_{CT}$ in April.
\b. In fact, I'm also going to Ireland in April.

\ex. The [female]$_{CT}$ popstars wore caftans.
\a.\# And the [male]$_{CT}$ popstars wore caftans.
\b. In fact, the male popstars also wore caftans.

It seems that \LLast triggers an implicature that the property of where I'm going in April only holds of China, just as in \Last, there is an implicature that the property of being a popstar and having worn caftans only holds of female individuals.

It is important to note that the same exhaustivity implicature is not triggered by focus, as the now classic example in \Next shows
\ex. (Where can I buy an Italian newspaper?)\\
You can buy an Italian newspaper [at the Indigo on University Ave]$_F$.\\
They also sell Italian newspapers at most of the news stands on College Street West.
>>>>>>> Stashed changes

\end{document}
