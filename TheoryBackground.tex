%        File: TheoryBackground.tex
%     Created: Tue Sep 08 12:00 PM 2015 E
% Last Change: Tue Sep 08 12:00 PM 2015 E
%
% arara: pdflatex: {options: "-draftmode"}
% arara: biber
% arara: pdflatex: {options: "-draftmode"}
% arara: pdflatex: {options: "-file-line-error-style"}
\documentclass[GPFinal]{subfiles}

\begin{document}
What I refer to as Contrastive Topic here is closely related to what \textcite{jackendoff1972Ssemantics} refers to as the B-accent in his now classic examples, reproduced here in \ref{ex:JDoffExes}.
\ex.\label{ex:JDoffExes}
\a. (What about FRED? What did HE eat?)\\
FRED$_{\text{B}}$ ate the BEANS$_{\text{A}}$.
\b. (What about the BEANS? Who ate them?)\\
FRED$_{\text{A}}$ ate the BEANS$_{\text{B}}$.\hfill\parencite[261]{jackendoff1972Ssemantics}

Jackendoff identifies the A and B pitch accents with a falling contour and a rise-fall-rise contour, respectively, and addresses their discourse pragmatics.
Since Jackendoff's work, there has been research on the pragmatics, semantics, syntax, and prosody of these phenomena, some of which I outline in this chapter.
In section \ref{sec:rooth}, I discuss the theory of alternative semantics, first developed to model the interpretation of focus.
In section \ref{sec:roberts}, I introduce Roberts' (\citeyear{roberts2012information}) \textit{question under discussion} model of discourse, which provides a preliminary analysis of the pragmatics of focus.
Roberts' model is further refined by \textcite{buring2003d}, whose d-tree formalism I discuss in section \ref{sec:BuringCT}.
Finally, \textcite{constant2014diss} revises B\"uring's account of CT and develops a syntactic account of it which I discuss in section \ref{sec:Constant}
\subsection{Alternative Semantics \parencite{rooth1992theory}}\label{sec:rooth}
Alternative semantics, as developed by \textcite{rooth1992theory}, proposes that, in addition to ordinary interpretations ($\llbracket\cdot\rrbracket^\mathcal{O}$), sentences receive a focus interpretation ($\llbracket\cdot\rrbracket^f$) which is derived from the ordinary interpretation and the focused constituent.
Consider the following example.
\ex.\label{ex:BasicFocus} [Mary]$_F$ answered Sue.

The ordinary interpretation of this sentence is the proposition it expresses
\ex.\label{ex:OrdinaryInterpretation} $\llbracket\ref{ex:BasicFocus}\rrbracket^\mathcal{O} = [answered(\mathbf{m}, \mathbf{s})]$

The focus interpretation is the set of propositions generated by replacing the focused material with a variable.
\ex.\label{ex:FocusInterpretation} $\llbracket\ref{ex:BasicFocus}\rrbracket^f = \left\{ answered(x, \mathbf{s}) | x \in D_e \right\}$

Note that the focus semantics of \ref{ex:BasicFocus} is equivalent to the ordinary interpretation of the question \textit{Who answered Sue?} following \textcite{hamblin1973questions}. 
This relation between focus interpretation and question interpretation is key to the model of discourse I assume here.

\subsection{Discourse Pragmatics \parencite{roberts2012information}}\label{sec:roberts}
\textcite{roberts2012information} models discourse as a cooperative game, following \textcite{lewis1979scorekeeping}, the goal of which is to answer the \textit{questions under discussion} (QUDs).
Utterences are represented as moves, with questions being setup moves and assertions being payoff moves.
At a given point in the discourse there is an immediate QUD, and discourse proceeds either by answering that question or by asking a subquestion (\textit{i.e.} one whose answer is a partial answer to the QUD), which becomes the new immediate QUD.
Roberts models the QUDs as a stack structure, so new subquestions are pushed into the stack when asked, and the immediate QUD is popped off of stack upon being answered.
A move is considered (ir)relevant based on the question at the top of the QUD stack.

Roberts' model of a particular discourse is given below as a series of questions, subquestions, and answers. 
\ex.[($\mathcal{D}_0$)] Who ate what?
	\a.[\texttt{a}.] What did Hilary eat?
		\a.[\texttt{i}.] Did Hilary eat bagels?\\
		Ans(\texttt{a}$_\texttt{i}$) = yes
		\b.[\texttt{ii}.]Did Hilary eat tofu?\\
		Ans(\texttt{a}$_\texttt{ii}$) = no
		\z.
	\b.[\texttt{b}.] What did Robin eat?
		\a.[\texttt{i}.]Did Robin eat bagels?\\
		Ans(\texttt{b}$_\texttt{i}$) = no
		\b.[\texttt{ii}.]Did Robin eat tofu?\\
		Ans(\texttt{b}$_\texttt{ii}$) = yes
		\z.
	\z.

Note, that this discourse goes beyond the explicitness we see in natural speech.
For example, when question (a) is asked, we don't require that (a$_\text{i}$) and (a$_\text{ii}$) are asked so that we may answer \textit{yes} or \textit{no}.
Instead we can answer with an assertion that includes a focused constituent that matches the wh-word of the QUD.
\ex. A: \texttt{a}/\#\texttt{b}\\
B: Hilary ate [bagels]$_F$.

To ensure that an assertion is used felicitously, Roberts exploits the fact that focus interpretations of assertions are of the same type as question interpretations.
An assertion, like that in \Last is felicitous if its focus interpretation is equal to the interpretation of the QUD.
\ex.
\a.
$\llbracket\text{Hilary ate [bagels]}_F.\rrbracket^f = 
\begin{Bmatrix}
  \text{Hilary ate bagels.}\\
  \text{Hilary ate tofu.}
\end{Bmatrix}
$
\b.
$\llbracket\text{What did Hilary eat?}\rrbracket^\mathcal{O} =
\begin{Bmatrix}
  \text{Hilary ate bagels.}\\
  \text{Hilary ate tofu.}
\end{Bmatrix}
$\hfill (=\Last[a])
\b.
$\llbracket\text{What did Robin eat?}\rrbracket^\mathcal{O} =
\begin{Bmatrix}
  \text{Robin ate bagels.}\\
  \text{Robin ate tofu.}
\end{Bmatrix}
$\hfill ($\neq$\Last[a])

Roberts goes on to address contrastive topics, which she refers to as \textit{dependent focus}, in much the same way as she treats focus.
Structures with CT and focus are given focus interpretation, that is, they are interpreted as a set of alternatives under alternative semantics.
An example of a CT-F utterance and its focus interpretation is given below in \Next.
\ex.
\a. [Hilary]$_{CT}$ ate [bagels]$_F$.
\b. $\left\{ x\text{ ate }y | x,y \in D_e \right\}$

This suggests that \Last[a] presupposes the question in \Last[b] (\textit{Who ate what?}), a proposal that Roberts shows does not hold up to further scrutiny.
This hypothesis predicts that \Last[a] ought to have the same felicity conditions if its CT and F marking were reversed as in \Next below.
\ex. 
\a. [Hilary]$_F$ ate [bagels]$_{CT}$.
\b. $\left\{ x\text{ ate }y | x,y \in D_e \right\}$

Roberts suggests that, rather than only presupposing a QUD, CT-F structured utterances also presuppose ``a possibly complex strategy of questions.'' \parencite[][p.50]{roberts2012information}
As Roberts acknowledges, this is a very preliminary account of the pragmatics of CT which will require further empirical and theoretical investigation.

\subsection{The discourse pragmatics of Contrastive Topics \parencite{buring2003d,buringforthcomingtopic}}\label{sec:BuringCT}
\textcite{buring2003d} represents Roberts' structured discourses as \textit{d(iscourse)-trees}.
The discourse $\mathcal{D}_0$, then is represented by the tree below.
\ex.
\begin{forest}
  tree defaults
  [question\\Who ate what?,for tree={align=center}
    [subquestion\\\texttt{a}
      [subsubquestion\\\texttt{a}$_\texttt{i}$
	[Ans(\texttt{a}$_\texttt{i}$)]
      ]
      [subsubquestion\\\texttt{a}$_\texttt{ii}$
	[Ans(\texttt{a}$_\texttt{ii}$)]
      ]
    ]
    [subquestion\\\texttt{b}
      [subsubquestion\\\texttt{b}$_\texttt{i}$
	[Ans(\texttt{b}$_\texttt{i}$)]
      ]
      [subsubquestion\\\texttt{b}$_\texttt{ii}$
	[Ans(\texttt{b}$_\texttt{ii}$)]
      ]
    ]
  ]
\end{forest}

B\"uring also distinguishes between the focus value ($\llbracket\cdot\rrbracket^f$) and the CT value ($\llbracket\cdot\rrbracket^{ct}$) of an utterance and defines an algorithm for determining the CT value, given below in \Next.
\ex. CT-value formation:
\a.[step 1:] Replace the focus with a \textit{wh}-word and front the latter; if focus marks the finite verb or negation, front the finite verb instead.
\b.[step 2:] Form a set of questions from the result of step 1 by replacing the contrastive topic with some alternative to it.\hfill\parencite{buring2003d}
\z.

Note, as demonstrated below, this algorithm generates a set of questions, which is a set of sets of propositions.
This way, \textcite{buring2003d} is able to build into his representations the fact that a CT-F structure presupposes a QUD and a strategy for answering it.
\ex.
\a.\label{ex:HilBagCT-F} [Hilary]$_{CT}$ ate [bagels]$_F$.
	\b. CT-value formation:
		\a.[step 1: ] What did Hilary eat?
		\b.[step 2: ] $
		\begin{Bmatrix}
		  \text{What did Hilary eat?}\\
		  \text{What did Robin eat?}
		\end{Bmatrix}$
		\z.
	\b. $\llbracket$[Hilary]$_{CT}$ ate [bagels]$_F$.$\rrbracket^{ct} = \left\{ \left\{ x\text{ ate }y | y \in D_e \right\} | x \in D_e \right\}$
	\z.

Under this analysis of CT-value, the CT-F structure of an utterance is represented by the value. 
So the CT-value of \Last[a] is distinct from that \Next[a], below, which inverts the CT-F structure.
\ex.
\a.\label{ex:HilBagF-CT} [Hilary]$_F$ ate [bagels]$_{CT}$.
\b. CT-value formation:
\a.[step 1:] Who ate bagels?
\b.[step 2:] $
\begin{Bmatrix}
  \text{Who ate bagels?}\\
  \text{Who ate tofu?}
\end{Bmatrix}$
\z.
\b. $\llbracket$[Hilary]$_F$ ate [bagels]$_{CT}$.$\rrbracket = \left\{ \left\{ x\text{ ate }y | x \in D_e \right\} y \in D_e \right\}$\hfill($\neq\llbracket\LLast[a]\rrbracket^{ct}$)
\z.

The nested nature of these CT-values, makes them directly translatable into d-trees which I provide below.
\ex.
\a. \begin{forest}
  tree defaults
  [{$\llbracket\ref{ex:HilBagCT-F}\rrbracket^{ct}$}
    [What did Robin eat?]
    [What did Hilary eat
      [Hilary ate bagels]
    ]
  ]
\end{forest}
\b.
\begin{forest}
  tree defaults
  [{$\llbracket\ref{ex:HilBagF-CT}\rrbracket^{ct}$}
    [Who ate tofu?]
    [Who ate bagels
      [Hilary ate bagels]
    ]
  ]
\end{forest}

D-trees provide a perspicuous way of representing various aspects of discourse structure in a way that leverages a vocabulary already used by generative linguists.
They allow us to define pragmatic notions such as assertions, questions, alternatives, \textit{etc} in terms of nodes, sisterhood, dominance, \textit{etc.}
For instance, assertions and questions are distinguished by the fact that the former are terminal nodes while the latter are non-terminal.

It should be noted that CT-F structures are used in a variety of discourse contexts to achieve subtly different conversational goals.
Consider the following examples.
\ex.\label{ex:ChinaCTF}
\a.[A:] When are you going to China? \hfill \parencite{roberts2012information}
\b.[B:] I'm going to [China]$_{CT}$ in [April]$_F$.
\z.

\ex.\label{ex:CaftansCTF}
\a.[A:] What did the pop stars wear? \hfill \parencite{buring2003d}
\b.[B:] The [female]$_{CT}$ popstars wore [caftans]$_F$.
\z.

\ex.\label{ex:DoctorChiroCTF}
\a.[A:] Who's a good psychiatrist?
\b.[B:] [My sister Monica]$_{F}$ is a [psychologist]$_{CT}$.
\z.

All of these instances of CT-F structures signal what B\"uring calls \textit{implicit moves}, each instance has a different sort of implicit move that can be easily represented by its d-tree.
In \ref{ex:ChinaCTF} the assertion directly answers the question, but implies the existence of a relevant superquestion (\textit{When are you going to which place?}).
The d-tree in \Next shows this by marking the explicit moves in bold.
\ex.
\begin{forest}
  tree defaults
  [When are you going which place?
    [When are you going to \ldots?]
    [\textbf{When are you going to China?}
      [\textbf{April}]
    ]
  ]
\end{forest}

The assertion in \ref{ex:CaftansCTF}, on the other hand, does not answer the explicit question, but instead answers an implied subquestion (\textit{What did the female pop-stars wear}).
Again this can be represented clearly in the d-tree in \Next.
\ex.
\begin{forest}
  tree defaults
  [\textbf{What did the pop stars wear?}
    [What did the male pop stars wear?]
    [What did the female pop stars wear?
      [\textbf{The female pop stars wore caftans.}]
    ]
  ]
\end{forest}

Finally, the assertion in \ref{ex:DoctorChiroCTF} answers neither the explicit question, nor an implied subquestion.
Instead, it answers an implicit subquestion of a superquestion of the explicit question, as we can see in its d-tree in \Next.
\ex.
\begin{forest}
  tree defaults
  [Who's a good mental health professional?
    [\textbf{Who's a good psychiatrist \ldots?}]
    [Who's a good \ldots?]
    [Who's a good psychologist \ldots?
      [\textbf{Monica}]
    ]
  ]
\end{forest}

So, although a given CT-F structure can be mapped onto a single d-tree in a predictable way, the context in which it is uttered determines its place in and effect on the discourse.
Implicit in \textcite{buring2003d} is an informal condition on CT felicity which I give in \Next.
\ex. M is a move that uses a CT-F structure.\\
Q is a question.\\
M is felicitous in the context of the QUD Q iff the M defines a d-tree DT such that Q is represented in DT.

Though informal, this condition can effectively rule out several examples of infelicitous CT-F structures.
The infelicity of the CT-Foc structures in \Next and \NNext is predicted by the fact that the explicit question that they answer is not found in the d-trees they project. 
\ex.\label{ex:HilBagelInfel} 
\a.
\a.[A:] Who ate bagels?
\b.[B:] \#[Hilary]$_{CT}$ ate [bagels]$_F$.
\z.
\b. $\llbracket$[Hilary]$_{CT}$ ate [bagels]$_F\rrbracket^{ct}$\\
\begin{forest}
  tree defaults
  [What did who eat?
    [What did Robin eat?]
    [What did Hilary eat?
      [\textbf{Hilary ate bagels}]
    ]
  ]
\end{forest}
\z.

\ex.\label{ex:MonChiroInfel}
\a.
\a.[A:] Who's a good psychiatrist?
\b.[B:]\# [My sister Monica]$_{CT}$ is a [psychologist]$_{F}$
\z.
\b.
\begin{forest}
  tree defaults
  [Who's a good mental health professional?
    [Monica's a good mental health professional?
      [\textbf{Monica's a good psychologist}]
    ]
    [Joe's a good mental health professional?
      [\ldots]
    ]
  ]
\end{forest}

\subsection{The Topic Abstraction analysis of CT \parencite{constant2014diss}}
In his thesis, \textcite{constant2014diss} proposes and argues for a comprehensive revision of B\"uring's theory of contrastive topic.
This revision includes a more nuanced analysis of the syntax of CTs and a more precise description of the prosody associated with CTs in English\footnote{
	The thesis also includes a proposed semantics CTs and an analysis of the Mandarin discourse particle \textit{-ne} as a CT marker.
	These topics, however, are beyond the scope of this paper, so I will not address them.
} Which I will outline in turn in this section.

Constant analyzes the pitch contour in terms of <+CITE+>'s <+CITEYEAR+> ToBI formalism\footnote{\textcite[14--16]{constant2014diss} provides a succinct description of the ToBI system, so rather than reproduce that description, I encourage interested readers to seek out this portion of the thesis and works cited therein.} and shows that the the characteristic rise-fall-rise contour of CTs is analyzable as a pitch accent (L+H*) followed by a low phrase tone (L-) and a high boundary tone (H\%), as shown in \ref{ex:FredToBI} and \ref{ex:BeansToBI}.
\ex.\label{ex:FredToBI}
\a.[A:] What about FRED? What did HE eat?
\bg.[B:] FRED {\ldots} {ate the beans.}\\
L+H* L-H\% {}\\

\ex.\label{ex:BeansToBI}
\a.[A:] What about the BEANS? Who ate THEM?
\bg.[B:] {Fred ate the} BEANS \ldots\\
{} L+H* L-H\%\\

Constant further notes that the pitch accent and boundary tones are associated to different things.
The pitch accent, he argues, is associated with an F-marked constituent, while the boundary tone is associated with the right edge of the phrase that contains the f-marked.
This can be seen in \ref{ex:FemaleToBI} and \ref{ex:SingersToBI}, where the placement of the L+H* accent depends on the discourse context, while the L-H\% boundary tone is associated with the edge of the DP.
\ex.\label{ex:FemaleToBI}
\a.[A:] What did the singers wear?
\bg.[B:] The {\hspace{1em}\textsc{female}} singers \ldots {wore caftans.}\\
{} L+H* {\hspace{2em}L-} H\% {}\\

\ex.\label{ex:SingersToBI}
\a.[A:] What did the female performers wear?
\bg.[B:] {The female} {\hspace{1em}\textsc{singers}} {\hspace{1em}\ldots} {wore caftans.}\\
{} L+H* L-H\% {}\\

Based on this pattern (and other reasons), Constant proposes that what B\"uring calls CT-marking, is identical to F-Marking, and the distinction between CT and exhaustive focus (hereafter Exh, following Constant) is due the structural configuration of those phrases that contain F-marked constituents.

In order to show capture the distinction between CT and Exh, Constant proposes an operator in the left periphery, CT-$\lambda$ whose specifier is interpreted as a CT. 
So, CT phrases are raised, sometimes covertly, to the left periphery (in the sense of \textcite{rizzi1997fine}) and the CT-$\lambda$ operator cliticizes to the intonational phrase, yeilding the L-H\% boundary tone.
The proposed LF structure of \ref{ex:BeansToBI}, then, is given in \ref{fig:BeansLF}, where a dashed arrow indicates covert movement.
\ex.\label{fig:BeansLF}
\begin{forest}
  tree defaults
  [
	  [DP$_{i}$[the beans$_F$, roof,name=ct]]
	  [
		  [CT-$\lambda$]
		  [
			  [DP[Fred$_F$,roof]]
			  [
				  [ate]
				  [$t_i$,name=obj]
			  ]
		  ]
	  ]
  ]
  \draw[->,dashed] (obj) to[out=south west, in=south] (ct);
\end{forest}

Topicalization, as in \ref{ex:Topicalize}, then occurs when CT Abstraction is overt, according to Constant.
\ex.\label{ex:Topicalize} 
\a.[A:] What about the BEANS? Who ate THEM?
\bg.[B:] The BEANS \ldots{Fred ate.} \\
{} L+H* L-H\% {}\\

\subsection{Summary}
In this section, I have outlined some basic properties of CTs which will be useful in the discussion of SC subjects.
In semantico-pragmatic terms, CTs are interpreted as a nested set of alternatives, which imply a complex discourse structure.
That is, if an utterance without a CT indicates a question-answer move in the discourse (either asking a question and expecting a complete answer, or giving a complete answer to a question under discussion), then an utterance \textit{with} a CT indicates a question-subquestion-answer strategy.
In syntactic terms, a CT is a phrase which (\textit{i}) has an F-marked constituent, and (\textit{ii}) is generated in, or moves, often covertly, to the specifier of a a phrase projected by a CT operator (CT-$\lambda$).
Prosodically, the F-marking in a CT is realized as a rising pitch accent (L+H*) and the CT-$\lambda$ operator, which cliticizes to the phrase in its specifier, is realized as a rising boundary tone (L-H\%). 
\end{document}
