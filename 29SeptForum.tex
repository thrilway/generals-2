%        File: 29SeptForum.tex
%     Created: Mon Sep 28 10:00 AM 2015 E
% Last Change: Mon Sep 28 10:00 AM 2015 E
%
% arara: pdflatex: {options: "-draftmode"}
% arara: biber
% arara: pdflatex: {options: "-draftmode"}
% arara: pdflatex: {options: "-file-line-error-style"}
\documentclass[letterpaper]{article}

\usepackage[margin=1in]{geometry}
\usepackage[backend=biber,style=authoryear-comp,useprefix=true]{biblatex}

\usepackage{linguex}

\usepackage[normalem]{ulem}

\usepackage{stmaryrd}
\usepackage[]{amsmath}
\usepackage{amsfonts}
\usepackage{forest}

\forestset{tree defaults/.style={for tree={parent anchor=south, child anchor=north},every tree node/.style={align=center,anchor=north},level/.style={sibling distance=50mm/#1},baseline}}

\forestset{en/.style={parent anchor=center, child anchor=center}}
\forestset{em/.style={parent anchor=north west, child anchor=north west}}

\usetikzlibrary{positioning}
\DeclareNameFormat{labelname:poss}{% Based on labelname from biblatex.def
  \ifcase\value{uniquename}%
    \usebibmacro{name:last}{#1}{#3}{#5}{#7}%
  \or
    \ifuseprefix
      {\usebibmacro{name:first-last}{#1}{#4}{#5}{#8}}
      {\usebibmacro{name:first-last}{#1}{#4}{#6}{#8}}%
  \or
    \usebibmacro{name:first-last}{#1}{#3}{#5}{#7}%
  \fi
  \usebibmacro{name:andothers}%
  \ifnumequal{\value{listcount}}{\value{liststop}}{'s}{}}

\DeclareFieldFormat{shorthand:poss}{%
  \ifnameundef{labelname}{#1's}{#1}}

\DeclareFieldFormat{citetitle:poss}{\mkbibemph{#1}'s}

\DeclareFieldFormat{label:poss}{#1's}

\newrobustcmd*{\posscitealias}{%
  \AtNextCite{%
    \DeclareNameAlias{labelname}{labelname:poss}%
    \DeclareFieldAlias{shorthand}{shorthand:poss}%
    \DeclareFieldAlias{citetitle}{citetitle:poss}%
    \DeclareFieldAlias{label}{label:poss}}}

\newrobustcmd*{\posscite}{%
  \posscitealias%
  \textcite}

\newrobustcmd*{\Posscite}{\bibsentence\posscite}

\newrobustcmd*{\posscites}{%
  \posscitealias%
  \textcites}

\bibliography{GP2}
\newcommand\quelle[1]{{%
  \unskip\nobreak\hfil\penalty50
  \hskip2em\hbox{}\nobreak\hfil#1%
  \parfillskip=0pt \finalhyphendemerits=0 \par}}
\begin{document}
\begin{center}
  {\Large On the Topic of Indefinite Specificational Subjects\\}
  {\large Dan Milway}\\
  Forum Presentation
\end{center}
\section{The indefinite restriction}
\begin{itemize}
  \item Specificational and predicational clauses appear to be the inverse of each other
\end{itemize}
\ex.
\a. Lydia is my favourite singer. (Predicational)
\b. My favourite singer is Lydia. (Specificational)
\z.

\begin{itemize}
  \item Simple indefinites cannot be subjects of specificational copular clauses (SCs)
\end{itemize}
\ex.\label{ex:BadSCs}
\a.* A doctor is Mike.
\b.* A building is Robarts.
\b.* A figure is Eric Lenneberg.
\z.

\begin{itemize}
  \item More complex indefinites can be SC subjects
\end{itemize}
\ex.\label{ex:GoodSCs}
\a. A newly-minted doctor is Mike.
\b. A building on campus no one likes is Robarts.
\b. An underrated figure in the history of generative grammar is Eric Lenneberg.
\z.

\begin{itemize}
  \item A proper analysis of the indefinite restriction will bring us closer to a proper analysis of SCs in general.
  \item If SCs are inverted predicational clauses \parencite{moro1997raising,mikkelsen2004specifying}, their subjects should be predicates (type $\langle e,t\rangle$)
    \begin{itemize}
      \item Indefinites are prototypical predicates and should be good SC subjects
      \item The indefinite restriction should be pragmatic.
    \end{itemize}
  \item If SCs are distinct from predicational clauses \parencite{heycockkroch1999pseudocleft,heycock2012specification}, their subjects are likely argumental (type $e$)
    \begin{itemize}
      \item The indefinite restriction should be semantic
    \end{itemize}
\end{itemize}
\textbf{My Claim:} SC subjects must contain \textit{contrastive topics} and the indefinite restriction can be derived from this requirement.
\section{Outline}
\begin{itemize}
  \item A brief theoretical background on contrastive topics (CTs).
  \item Predicting good indefinite subjects.
  \item Ruling out bad indefinite subjects.
\end{itemize}
\paragraph{A note on notation}
The acceptability judgement notations are not meant to presuppose any analysis.
Strings that are unacceptable regardless of context are marked *.
Those that are unacceptable in a given context are marked \#.
\section{Theoretical Background}
\begin{itemize}
  \item The notion of CT comes from \posscite{jackendoff1972Ssemantics} A- and B-Accents.
\end{itemize}
\ex.
\a.
\a.[A:] Well, what about FRED? What did HE eat?
\b.[B:] FRED$_B$ ate the BEANS$_A$.
\z.
\b.
\a.[A:] Well, what about the BEANS? Who ate THEM?
\b.[B:] FRED$_A$ ate the BEANS$_B$
\z.
\z.

\begin{itemize}
  \item The A-Accent, which corresponds to the wh-consituent of the question is the focus.
  \item Following \textcite{buring2003d}, I refer to B-Accent as the CT (\textcite{roberts2012information} calls it \textit{dependant focus})
  \item In addition to an ordinary interpretation ($\llbracket\cdot\rrbracket^\mathcal{O}$), utterances have focus interpretation ($\llbracket\cdot\rrbracket^f$)\parencite{rooth1992theory}, and a CT interpretation ($\llbracket\cdot\rrbracket^{ct}$)\parencite{buring2003d}
  \item Focus interpretations are sets of alternatives \parencite{rooth1992theory} equivalent to the interpretations of questions \parencite{groenendijkstokhof1996questions}
  \item CT interpretations are sets of sets of alternatives \parencite{buring2003d}.
\end{itemize}
\ex. 
\a. $\llbracket\text{FRED}_B\text{ ate the BEANS}_A\rrbracket^{f} = 
\begin{Bmatrix}
  \text{Fred ate the corn}\\
  \text{Fred ate the beans}\\
  \text{Fred ate the apples}\\
  \cdots
\end{Bmatrix}
= \llbracket\text{What did Fred eat?}\rrbracket
$
\b.$\llbracket\text{FRED}_B\text{ ate the BEANS}_A\rrbracket^{ct} = 
\begin{Bmatrix}
  \text{What did Fred eat?}\\
  \text{What did Mary eat?}\\
  \text{What did Robin eat?}\\
  \cdots
\end{Bmatrix}
= \llbracket\text{What did who eat?}\rrbracket
$

\begin{itemize}
  \item B\"uring represents these interpretations as d(iscourse)-trees
\end{itemize}
\ex. $\llbracket\text{FRED}_{CT}\text{ ate the BEANS}_{F}\rrbracket^{ct}$\\
\begin{forest}
  tree defaults
  [What did who eat?
    [What did Mary eat?]
    [What did Robin eat?]
    [What did Fred eat?
      [Fred ate the corn]
      [Fred ate the apples]
      [\textbf{Fred ate the beans}]
    ]
  ]
\end{forest}

\begin{itemize}
  \item Generally, CT-F structures identify a question and superquestion that they answer.
  \item CT-F structures have multiple discourse functions that the d-trees can represent.
\end{itemize}
\ex.\label{ex:ChinaCTF}
\a.
\a.[A:] When are you going to China? \hfill \parencite{roberts2012information}
\b.[B:] I'm going to [China]$_{CT}$ in [April]$_F$.
\z.
\b.
\begin{forest}
  tree defaults
  [When are you going which place?
    [When are you going to \ldots?]
    [\textbf{When are you going to China?}
      [\textbf{April}]
    ]
  ]
\end{forest}
\z.

\ex.\label{ex:CaftansCTF}
\a.
\a.[A:] What did the pop stars wear? \hfill \parencite{buring2003d}
\b.[B:] The [female]$_{CT}$ popstars wore [caftans]$_F$.
\z.
\b.
\begin{forest}
  tree defaults
  [\textbf{What did the pop stars wear?}
    [What did the male pop stars wear?]
    [What did the female pop stars wear?
      [\textbf{The female pop stars wore caftans.}]
    ]
  ]
\end{forest}
\z.

\ex.\label{ex:DoctorChiroCTF}
\a.
\a.[A:] Who's a good doctor to treat my back pain?
\b.[B:] [My sister Monica]$_{F}$ is a [chiropractor]$_{CT}$.
\z.
\b.
\begin{forest}
  tree defaults
  [Who's a good $\langle profession\rangle$ to treat back pain?
    [\textbf{Who's a good doctor \ldots?}]
    [Who's a good \ldots?]
    [Who's a good chiropractor \ldots?
      [\textbf{Monica}]
    ]
  ]
\end{forest}
\z.

\section{When indefinites can be SC subjects}
\ex.\label{ex:GoodSCsRepeat}
\a. A newly-minted doctor is Mike.
\b. A building on campus no one likes is Robarts.
\b. An underrated figure in the history of generative grammar is Eric Lenneberg.
\z.

\begin{itemize}
  \item All of the SCs in \Last are most naturally uttered with intonational stress on some part of the subject.
  \item The placement of stress marks the CT.
  \item DP2 is the focus.
\end{itemize}
\ex. $\llbracket\text{A building on campus NO ONE likes is Robarts}\rrbracket^{ct}$\\
\begin{forest}
  tree defaults
  [What is a BOC that who likes?
    [What is a BOC someone likes?]
    [What is a BOC no one likes?
      [A BOC no one likes is \ldots]
      [\textbf{A BOC no one likes is Robarts}]
    ]
  ]
\end{forest}

\begin{itemize}
  \item The SC in \Last is felicitous in discourse contexts in which who likes what building is under discussion
\end{itemize}
\ex. 
\a. Not many people like Sid Smith but,
\a. A building on campus NO ONE likes is Robarts.
\b.\# A BUILDING on campus no one likes is Robarts.
\z.
\b. No one likes Philosopher's Walk but,
\a.\# A building on campus NO ONE likes is Robarts.
\b. A BUILDING on campus no one likes is Robarts.
\z.

\textbf{Generalization:} Subjects of specificational clauses must contain both discourse-new and discourse-given material.

\section{When indefinites cannot be SC subjects: simple indefinites}
\ex.
\a.* A doctor is Mike.
\b.* A building is Robarts.
\b.* A figure is Eric Lenneberg.
\z.

\begin{itemize}
  \item The generalization above obviously rules these out.
    \begin{itemize}
      \item The substantive material in these subjects is the noun, which cannot be simultaneously new and given.
    \end{itemize}
  \item Simple indefinites allow us to probe why such a generalization should hold at all.
\end{itemize}
\ex. $\llbracket\text{A doctor}_{CT}\text{ is Mike}\rrbracket^{ct}$\\
\begin{forest}
  tree defaults
  [Who is what?
    [Who is a lawyer?]
    [Who is a socialist?
      [\ldots]
      [A socialist is Mike]
    ]
    [\ldots]
    [Who is brown-eyed?]
    [Who is a doctor?
      [\ldots]
      [A doctor is Mike]
    ]
  ]
\end{forest}

\begin{itemize}
  \item Impressionistically, \textit{Who is what?} is too vague to ever be asked. But why?
  \item \textcite[citing Hamblin][]{groenendijkstokhof1996questions} provide a possible explanation\\
    ``The possible answers to a question form an exhaustive set of mutually exclusive possibilities''
  \item One possible answer is a set of worlds which is mutually exclusive with every other possible answer.
  \item In \Last, the worlds in which Mike is a doctor would be exclude those worlds in which Mike has brown eyes and vice versa.
  \item The question \textit{Who is What?} is ill-formed, which rules out virtually every simple indefinite as SC subject.
\end{itemize}
\subsection{Another ruled-out SC}
\begin{itemize}
  \item Consider The Beatles.
  \item Each of the members had a primary instrument (John:Guitar, Paul:Bass, George:Guitar, Ringo:Drums), and all of them were also vocalists.
\end{itemize}
\ex. 
\a. Paul was the Bassist in The Beatles\\
A GUITARIST in the Beatles was John.
\b. Paul was a vocalist in the Beatles\\
\#A GUITARIST in the Beatles was John.

\begin{itemize}
  \item \textit{Guitarist} and \textit{bassist} are mutally exclusive in this context, so they can be contrasted
  \item \textit{Guitarist} and \textit{vocalist} are not mutually exclusive, so we cannot contrast them.
\end{itemize}

\section{Conclusion}
\begin{itemize}
  \item A requirement that SC subjects include CT-marked material plus well-formedness conditions on alternative sets/questions allows us to derive the indefinite restriction.
  \item The explanation is pragmatic in nature and consistent with \posscite{mikkelsen2004specifying} inversion analysis of SCs.
\end{itemize}
\printbibliography
\end{document}


