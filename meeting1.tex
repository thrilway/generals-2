\documentclass[letterpaper]{article}
\usepackage[margin=1in]{geometry}
\usepackage[]{amsmath}
\usepackage{stmaryrd}

\usepackage[]{graphicx}
\usepackage{linguex}

\usepackage{tikz}
\usepackage{forest}

\forestset{tree defaults/.style={for tree={parent anchor=south, child anchor=north},every tree node/.style={align=center,anchor=north},level/.style={sibling distance=50mm/#1},baseline}}
\forestset{en/.style={parent anchor=center, child anchor=center}}
\forestset{em/.style={parent anchor=north west, child anchor=north west}}

\usepackage[backend=bibtex,style=authoryear]{biblatex}

\bibliography{GP2}
\begin{document}
\section{Empirical Observations}
\begin{itemize}
	\item Adnominal contitionals under indefinite determiners seem to be dispreferred
\end{itemize}
\ex. 
\a. The consequences if we fail include imprisonment.
\b. One of the consequences if we fail is imprisonment.
\c.\# A consequence if we fail is imprisonment.

\begin{itemize}
	\item Indefinite ACs are okay when they can receive a \textit{law-like} interpretation.
\end{itemize}
\ex. \textbf{Scenario:} There is a nationwide celebration being planned. 
There will be events in towns and cities across the country.
Outdoor events are preferred but if it rains in a given town or city, there must be a nearby indoor facility available.
\a. A location if it rains cannot be more than 1 km from the primary location.

\nocite{*}
\printbibliography
\end{document}
