%        File: binary-branching-dtree.tex
%     Created: Thu Jun 04 10:00 AM 2015 E
% Last Change: Thu Jun 04 10:00 AM 2015 E
%
% arara: pdflatex
% arara: bibtex
% arara: pdflatex
% arara: pdflatex
\documentclass[letterpaper]{article}

\usepackage[margin=1in]{geometry}
\usepackage[backend=bibtex,style=authoryear]{biblatex}

\usepackage[]{amsmath}
\usepackage{stmaryrd}

\usepackage{linguex}

\bibliography{GP2}

\usepackage{forest}
\forestset{tree defaults/.style={for tree={parent anchor=south, child anchor=north},every tree node/.style={align=center,anchor=north},level/.style={sibling distance=50mm/#1},baseline}}

\begin{document}
\section{Restricted contrast}
The semantic content of licit specificational clauses can be divided into contrastive and presupposed (in the pragmatic sense) content.
Contrastive content can be further divided into Focus (F) and  Contrastive Topic (CT) each of which encodes information not previously part of the common ground.
The presupposed content is part of the common ground that defines the bounds of contrast encoded by F and CT.
Consider the specificational clauses below,

\section{Binary Branching D-Trees}
A binary branching discourse tree would work as follows.
Each question represents a strict partition of it's domain.
For instance the question \textit{What did the pop-stars wear?} partitions the domain of wearable things into \textit{things the pop-stars wore} and its complement \textit{things the pop-stars didn't wear}.
An answer of \textit{sunglasses} would partition the former subdomain into the singleton set containing \textit{sunglasses} and its complement \textit{other things the pop-stars wore}.

To see what this system rules out, consider the following scenario:
\ex. $
\begin{array}[t]{lll}
  D & = & \left\{ \text{Emily, Arsalan, Paula} \right\}\\
  Lawyers & = &\left\{ \text{Emily, Arsalan} \right\}\\
  Doctors & = &\left\{ \text{Arsalan, Paula} \right\}
\end{array}
$

In a dicourse about this scenario we could ask the question \textit{Who is a doctor?}, which could, intuitively, denote two possible partitions.
One that partitions the domain  into sets of individuals who are doctors and those who are not, and another that partitions the domain into sets of individuals who are doctors and those who have a non-doctor profession (\textit{i.e.}, lawyers).
\ex. $\left\{x \in D | x \text{ is a doctor} \right\}^\prime =$
\a.\label{good-partition} $\left\{ x \in D | x \text{ isn't a doctor} \right\} =$\\
$\left\{ \text{Emily} \right\}$\\
\textsc{or}
\b.\label{bad-partition} $\left\{ x \in D | x \text{ is a non-doctor} \right\} =$\\
$\left\{ \text{Emily, Arsalan} \right\}$

\ex.
\a. 
\begin{forest}
  tree defaults
  [Who is a doctor?
    [$\left\{ \text{Arsalan, Paula} \right\}$]
    [Who isn't a doctor?]
  ]
\end{forest}
\b.
\begin{forest}
  tree defaults
  [Who is a doctor?
    [$\left\{ \text{Arsalan, Paula} \right\}$]
    [Who is a non-doctor?]
  ]
\end{forest}

Under a binary branching model of discourse, \ref{good-partition} is a possible partition ($X \cap X^\prime = \emptyset$) , while \ref{bad-partition} is not ($X \cap X^\prime \neq \emptyset$)

Given the CT interpretation algorithm, the binary branching requirement on D-Trees, and the proposal that SSs are CTs, we can explain the deviance of the utterance \textit{a doctor is John}.
First, the CT interpretation of the offending clause is given below.
\ex. $\llbracket \text{a doctor is John}\rrbracket^{CT} =$\\
$\left\{ \left\{ P(x) | x \in D_e \right\} P \in D_{et} \right\}$\\
$\approx$ Who is what?

Converting this CT interpretation into binary branching structure, yields a branching node corresponding to the focus interpretation (\textit{Who is a doctor}), and a higher node dividing the the domain of individuals into doctors and non-doctors.
\ex.
\begin{forest}
  tree defaults
  [Who is what?
    [Who is a doctor?
      [John]
      [Who else?]
    ]
    [Who is a non-doctor?
    ]
  ]
\end{forest}

As we saw above in \ref{bad-partition}, the top node in this tree does not represent a proper partition, and as such is ill-formed.


\end{document}


