%        File: 17Mar-Forum.tex
%     Created: Sun Mar 15 09:00 PM 2015 E
% Last Change: Sun Mar 15 09:00 PM 2015 E
%

\documentclass[letterpaper,12pt]{article}
\usepackage[margin=1in]{geometry}
%\usepackage{natbib}
\usepackage{linguex}

\usepackage[]{amsmath}
\usepackage{stmaryrd}

\usepackage{forest}

\forestset{tree defaults/.style={for tree={parent anchor=south, child anchor=north},every tree node/.style={align=center,anchor=north},level/.style={sibling distance=50mm/#1},baseline}}

\usepackage[backend=bibtex,style=authoryear]{biblatex}

\bibliography{GP2}
\begin{document}
\begin{center}
	\textbf{Adnominal conditionals}\\
	Forum Presentation\\
	Dan Milway \\
	\today\\
\end{center}

\begin{itemize}
  \item Adnominal conditionals (ACs) are conditional clauses that seem to modify nouns.
\end{itemize}
\ex. OK, let's begin.
We have a lot riding on this job, so please be very careful and make no mistakes.
If we succeed at what we are trying to do, the benefits will be enormous.
\textbf{But we all know the consequences if we fail.} \hfill (\cite{lasersohn1996adnominal})
\z.

\ex.
\a. The fine if you park in a handicapped spot is higher than the fine if your
meter expires.
\b. The price if you pay now is predictable; the price if you wait a year is not.
\c. The outcome if John gets his way is sure to be unpleasant for the rest of
us. \hfill (\cite{lasersohn1996adnominal})

\begin{itemize}
	\item I will be focusing on ACs in indefinite subjects of specificational copular clauses (SCCs).
	\item While indefinite DPs are the correct type ($\langle e,t\rangle$) to subjects of SCCs \parencite{mikkelsen2004specifying}, they seem to be disallowed in that position.
\end{itemize}
\ex.\# A doctor is Rose.

\begin{itemize}
	\item Plain indefinite DPs with ACs are also unacceptable.
\end{itemize}
\ex.\label{ac-unacceptable} \# A doctor if you have high blood pressure is Rose.

\begin{itemize}
	\item Indefinite DPs with ACs and (certain?) adjectives, however, are acceptable.
\end{itemize}
\ex.\label{acceptable} A good doctor if you have high blood pressure is Rose.

\begin{description}
	\item[\textbf{Question:}] Why is this so?
		\begin{description}
			\item[\textbf{Sub-question 1:}] What requirements are placed on indefinite subjects of SCCs.
			\item[\textbf{Sub-question 2:}] What are the properties of ACs that allow \ref{acceptable} to be accepted.
		\end{description}
\end{description}

\begin{itemize}
	\item To SubQ1: \textcite{mikkelsen2004specifying} argues that subjects of SCCs must be Topics.
		\begin{itemize}
			\item A clause's  \textit{Topic} is usually thought to be discourse-old, and be what the discourse is about.
			\item \textcite{roberts2012information}, and \textcite{buring1999topic} provide a more rigourous discussion of topic-hood
		\end{itemize}
	\item To SubQ2: Given the most recent analysis of ACs \parencite{franaappearmodality} there is no obvious reason that \ref{acceptable} is acceptable while \ref{ac-unacceptable} is not.
	\item According to \textcite{franaappearmodality}, ACs are devices for restricting the bases of modal adjectives
\end{itemize}
\ex.
\begin{forest}
	tree defaults
	[{NP:$\langle e,t\rangle$} 
		[{$\langle\langle s, et\rangle, et\rangle$} 
			[{Mod-Adj: $\langle st\langle\langle s, et\rangle, et\rangle\rangle$}]
			[{$\langle s,t\rangle$} 
				[{Base: $\langle s,t\rangle$}] 
				[{if-clause: $\langle s,t\rangle$}]
			]
		]
		[{N:$\langle s, et\rangle$}]
	]
\end{forest}

\begin{itemize}
	\item The modal adjective can be overt or covert, but only overt adjectives render an AC-modified DP acceptable as the subject of an SCC.
\end{itemize}
\ex. 
\a. a consequence if we fail = a \textsc{necessary} consequence if we fail.
\b.\# a consequence if we fail is imprisonment.
\z.

\ex.
\a. a possible consequence if we fail is imprisonment.

\begin{itemize}
	\item Furthermore, the meaning predicted by \textcite{franaappearmodality} for the AC in \ref{acceptable} is odd.
\end{itemize}
\ex. $\llbracket$ a good doctor if you have high blood pressure $\rrbracket^{w,c,f}$\\
= $\exists x \forall w'. Acc(w')(w) \wedge HBP(you)(w') [x \text{ is good in } w'\,and\,x \text{ is a doctor in } w']$

\begin{itemize}
	\item x being a good doctor would be contingent on \textit{you} having high blood pressure.
	\item \textit{you} is interpreted impersonally, so it really means if \textit{someone} has high blood pressure, x is a good doctor.
	\item x does not even have to be particularly good at treating high blood pressure.
	\item \citeauthor{franaappearmodality}'s (\citeyear{franaappearmodality}) analysis needs reworking.
	\item Adjectives like \textit{good} are evaluated relative to a comparison class. \parencite{klein1980semantics}
	\item the AC in \ref{acceptable} seems to be restricting the comparison class of \textit{good}.
\end{itemize}
\printbibliography
\end{document}

