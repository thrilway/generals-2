%        File: 17Mar-Forum.tex
%     Created: Sun Mar 15 09:00 PM 2015 E
% Last Change: Sun Mar 15 09:00 PM 2015 E
%

\documentclass[letterpaper,12pt]{article}

\usepackage[margin=1in]{geometry}
\usepackage{natbib}
\usepackage{linguex}

\usepackage[]{amsmath}
\usepackage{stmaryrd}

\usepackage{forest}

\forestset{tree defaults/.style={for tree={parent anchor=south, child anchor=north},every tree node/.style={align=center,anchor=north},level/.style={sibling distance=50mm/#1},baseline}}

\usepackage[backend=bibtex,style=authoryear]{biblatex}

\bibliography{GP2}
\begin{document}
\begin{itemize}
  \item Adnominal conditionals (ACs) are conditional clauses that seem to modify nouns.
\end{itemize}
\ex. OK, let's begin.
We have a lot riding on this job, so please be very careful and make no mistakes.
If we succeed at what we are trying to do, the benefits will be enormous.
\textbf{But we all know the consequences if we fail.} \hfill (\cite{lasersohn1996adnominal})
\z.

\ex.
\a. The fine if you park in a handicapped spot is higher than the fine if your
meter expires.
\b. The price if you pay now is predictable; the price if you wait a year is not.
\c. The outcome if John gets his way is sure to be unpleasant for the rest of
us. \hfill (\cite{lasersohn1996adnominal})
\begin{itemize}
  \item 
\end{itemize}<++>
\end{document}

