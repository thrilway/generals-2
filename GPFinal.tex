%        File: GPFinal.tex
%     Created: Wed Jun 17 01:00 PM 2015 E
% Last Change: Wed Jun 17 01:00 PM 2015 E
%
% arara: pdflatex: {options: "-draftmode"}
% arara: bibtex
% arara: pdflatex: {options: "-draftmode"}
% arara: pdflatex: {options: "-file-line-error-style"}
\documentclass[letterpaper]{article}

\usepackage[margin=1in]{geometry}
\usepackage[backend=bibtex,style=authoryear]{biblatex}

\usepackage{linguex}

\usepackage{stmaryrd}
\usepackage[]{amsmath}
\usepackage{amsfonts}
\usepackage{forest}

\forestset{tree defaults/.style={for tree={parent anchor=south, child anchor=north},every tree node/.style={align=center,anchor=north},level/.style={sibling distance=50mm/#1},baseline}}

\forestset{en/.style={parent anchor=center, child anchor=center}}
\forestset{em/.style={parent anchor=north west, child anchor=north west}}

\usetikzlibrary{positioning}
\DeclareNameFormat{labelname:poss}{% Based on labelname from biblatex.def
  \ifcase\value{uniquename}%
    \usebibmacro{name:last}{#1}{#3}{#5}{#7}%
  \or
    \ifuseprefix
      {\usebibmacro{name:first-last}{#1}{#4}{#5}{#8}}
      {\usebibmacro{name:first-last}{#1}{#4}{#6}{#8}}%
  \or
    \usebibmacro{name:first-last}{#1}{#3}{#5}{#7}%
  \fi
  \usebibmacro{name:andothers}%
  \ifnumequal{\value{listcount}}{\value{liststop}}{'s}{}}

\DeclareFieldFormat{shorthand:poss}{%
  \ifnameundef{labelname}{#1's}{#1}}

\DeclareFieldFormat{citetitle:poss}{\mkbibemph{#1}'s}

\DeclareFieldFormat{label:poss}{#1's}

\newrobustcmd*{\posscitealias}{%
  \AtNextCite{%
    \DeclareNameAlias{labelname}{labelname:poss}%
    \DeclareFieldAlias{shorthand}{shorthand:poss}%
    \DeclareFieldAlias{citetitle}{citetitle:poss}%
    \DeclareFieldAlias{label}{label:poss}}}

\newrobustcmd*{\posscite}{%
  \posscitealias%
  \textcite}

\newrobustcmd*{\Posscite}{\bibsentence\posscite}

\newrobustcmd*{\posscites}{%
  \posscitealias%
  \textcites}

\bibliography{GP2}


\title{On the Topic of Indefinite Specificational Subjects\\\textit{Draft}}
\author{Daniel Milway}

\begin{document}
\maketitle
\section{Introduction}
This paper is concerned with specificational copular clauses (SCs), and specifically those SCs with indefinite subjects.
For the purposes of this paper, I will adopt \posscite{mikkelsen2004specifying} characterization, according to which SCs are those copular clauses with a predicational (type $\langle e,t\rangle$) precopular DP and a referential (type $e$) postcopular DP.
As was observed by \textcite{higgins1973pseudo}, there  seems to be a constraint on indefinites as subjects of SCs.
Consider the following\footnote{
	A note on natural language examples:
	Reported judgements are not intended to imply any sort of analysis.
	Strings marked with * are judged unacceptable (not necessarily ungrammatical).
	Those marked with \# are judged unacceptable/unnatural in a given context.
}.
\ex.\label{ex:badSCs}
\a.* A doctor is Derek.
\b.* A building is Robart's Library.
\c.* A figure is Eric Lenneberg. 
\z.

\textcite{mikkelsen2004specifying} argues that the constraint in question cannot be narrowly semantic or syntactic in nature.
Since indefinites are usually predicational, they should have the proper type to be SC subjects, and in fact certain indefinites are able to be SC subjects.
\ex.\label{ex:good-scs}
\a. A newly minted doctor is Derek.
\b. A building on campus no one likes is Robart's Library.
\c. An underrated figure in the history of Generative Grammar is Eric Lenneberg.
\z.

Mikkelsen suggests a pragmatic account of the constraint.
She notes that, in general, SC subjects are Topics in the Discourse-old sense, and argues that indefinites are by their very nature Discourse-new.
This approach faces two problems, one empirical and one conceptual.
The empirical problem, which Mikkelsen herself acknowledges, is that no amount of discourse can render the strings in \ref{ex:badSCs} acceptable.
\ex. Bill is a doctor. *A doctor is John (too). \hfill \parencite[Adapted from][p. 236]{mikkelsen2004specifying}

The conceptual issue has to do with defining topic in terms of discourse age, as the terms discourse-old and -new are far too vague to provide the basis of an analysis.

\section{Towards an account}
In large part, the account I will give is a refinement of Mikkelsen's.
According to the account I will present below, subjects of SCs are Contrastive Topics \parencite[in the sense of][]{buring2003d} and minimal indefinite DPs like those in \ref{ex:badSCs} cannot function as contrastive topics.

The paper is structured as follows.
First, I will present the formal analysis of contrastive topic (CT) which fill form the basis of my account.
Next, I will revise the analysis of CT.
I will then go on to show how treating specificational subjects as CTs allows us to predict whether a DP will be a licit subject.
After this I will address an apparent class of counterexamples.

\section{Contrastive Topics}
\textcite{buring2003d} builds a theory of contrastive topics based on \posscite{roberts2012information} formalization of information structure, and the alternative semantics analysis of focus \parencite{rooth1992theory}.
Since these theories are crucial to the understanding of B\"uring's theory of CTs, I will outline them before outlining and modifying B\"uring's theory.

\subsection{Alternative Semantics}
The main claims of an alternative semantics analysis of focus are that (a) in addition to the literal interpretation of a given utterance ($\llbracket \cdot \rrbracket^\mathcal{O}$), there is also a \textit{focus} interpretation ($\llbracket \cdot \rrbracket^f$), (b) the focus interpretation of an utterance is a set of alternative literal interpretations, and (c) those alternative interpretations are derived by replacing the value of the utterance's focused element with a variable.

Consider the alternative semantics analysis of the sentence below.
\ex.\label{ex:basicfocus} [Mary]$_F$ answered Sue

To derive the focus interpretation of this sentence, we compute the literal interpretation, and replace the portion of that interpretation that corresponds to the focused element with a variable.
\ex.
\a. $\llbracket \text{\ref{ex:basicfocus}}\rrbracket^\mathcal{O} = [answered(\textbf{m,s})]$
\b. $\llbracket \text{\ref{ex:basicfocus}}\rrbracket^f = \left\{ x | answered(x,\textbf{s}) \right\}$

\subsection{Information Structure}

\subsection{Discourse Trees and Contrastive Topics}
\textcite{buring2003d} presents \posscite{roberts2012information} formalization of discourse as a tree structure with non-terminal nodes representing questions and terminal nodes representing (partial) answers.

\section{A CT analysis of SC subjects}
I propose that SC necessarily have a CT+F structure, with the postcopular DP as focus and some element of the subject as CT.
That the postcopular DP is focused can be seen in the infelicity of an SC in in response to a question with about it.
\ex.\label{ex:BadFocus} Who is Ziggy Stardust?\\
\#The lead guitarist of The Spiders from Mars is Ziggy Stardust.

To see that SC subjects must include a \textit{contrastive} topic rather than a discourse-old topic, compare the example question-answer pairs below.
\ex.
\a.\label{ex:BadCT} Who is the lead guitarist of The Spiders from Mars?\\
\#The lead guitarist of The Spiders from Mars is Ziggy Stardust.
\b.\label{ex:GoodCT} Who is the guitarist in The Spiders from Mars?\\
The lead guitarist in The Spiders from Mars is Ziggy Stardust.

When an SC answers a question that contains its entire subject, as in \ref{ex:BadCT}, it is unnatural, but if part of the SC subject is absent from the question, as in \ref{ex:GoodCT}, the SC is rendered felicitous.
Futhermore, the SC in \ref{ex:GoodCT} must be uttered with intonational stress on \textit{lead} to be felicitous, and carries the implication that The Spiders From Mars has more than one guitarist.
The adjective \textit{lead}, then, represents the CT of \ref{ex:GoodCT}.

The CT interpretation of \ref{ex:GoodCT} would be the question: \textit{Who is which guitarist in the The Spiders From Mars?}, which is represented below.
\ex.
\begin{forest}
  tree defaults
  [Who is which guitarist in the The Spiders From Mars? 
    [Who is the lead guitarist\dots?
      [Ziggy Stardust]
      [\dots]
    ]
    [\dots]
  ]
\end{forest}


\printbibliography
\end{document}


