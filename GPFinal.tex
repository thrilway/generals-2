%        File: GPFinal.tex
%     Created: Wed Jun 17 01:00 PM 2015 E
% Last Change: Wed Jun 17 01:00 PM 2015 E
%
% arara: pdflatex: {options: "-draftmode"}
% arara: biber
% arara: pdflatex: {options: "-draftmode"}
% arara: pdflatex: {options: "-file-line-error-style"}
\documentclass[letterpaper]{article}
\usepackage{subfiles}
\usepackage[margin=1in]{geometry}
\usepackage[backend=biber,style=authoryear-comp,useprefix=true]{biblatex}

\usepackage{setspace}

\usepackage{linguex}

\usepackage[normalem]{ulem}

\usepackage{stmaryrd}
\usepackage[]{amsmath}
\usepackage{amsfonts}
\usepackage{forest}

\forestset{tree defaults/.style={for tree={parent anchor=south, child anchor=north},every tree node/.style={align=center,anchor=north},level/.style={sibling distance=50mm/#1},baseline}}

\forestset{en/.style={parent anchor=center, child anchor=center}}
\forestset{em/.style={parent anchor=north west, child anchor=north west}}

\usetikzlibrary{positioning}
\DeclareNameFormat{labelname:poss}{% Based on labelname from biblatex.def
  \ifcase\value{uniquename}%
    \usebibmacro{name:last}{#1}{#3}{#5}{#7}%
  \or
    \ifuseprefix
      {\usebibmacro{name:first-last}{#1}{#4}{#5}{#8}}
      {\usebibmacro{name:first-last}{#1}{#4}{#6}{#8}}%
  \or
    \usebibmacro{name:first-last}{#1}{#3}{#5}{#7}%
  \fi
  \usebibmacro{name:andothers}%
  \ifnumequal{\value{listcount}}{\value{liststop}}{'s}{}}

\DeclareFieldFormat{shorthand:poss}{%
  \ifnameundef{labelname}{#1's}{#1}}

\DeclareFieldFormat{citetitle:poss}{\mkbibemph{#1}'s}

\DeclareFieldFormat{label:poss}{#1's}

\newrobustcmd*{\posscitealias}{%
  \AtNextCite{%
    \DeclareNameAlias{labelname}{labelname:poss}%
    \DeclareFieldAlias{shorthand}{shorthand:poss}%
    \DeclareFieldAlias{citetitle}{citetitle:poss}%
    \DeclareFieldAlias{label}{label:poss}}}

\newrobustcmd*{\posscite}{%
  \posscitealias%
  \textcite}

\newrobustcmd*{\Posscite}{\bibsentence\posscite}

\newrobustcmd*{\posscites}{%
  \posscitealias%
  \textcites}

\bibliography{GP2}
\newcommand\quelle[1]{{%
  \unskip\nobreak\hfil\penalty50
  \hskip2em\hbox{}\nobreak\hfil#1%
  \parfillskip=0pt \finalhyphendemerits=0 \par}}

\title{On the Topic of Indefinite Specificational Subjects\\\textit{Draft}}
\author{Daniel Milway}

\begin{document}
\maketitle
\doublespacing
\textit{\textbf{Note:} Throughout this draft you will see <++> or <+Some Words+> these are artifacts of my latex editor.
They are notes to myself to remind me of parts that I need to complete.
}
\section{Introduction}
The specificational clause is one of the varieties of copular clauses identified by \textcite{higgins1973pseudo}, characterized by an apparently predicative DP in subject position (DP1) and an argumental DP in post-copular position (DP2).
They contrast with predicational copular clauses in which DP1 is argumental and DP2 is predicational.
\ex.
\a.\textbf{Specificational}\\
My favourite book is \textit{War \& Peace}.
\b. \textbf{Predicational}\\
\textit{War \& Peace} is my favourite book.

In predicational clauses, DP1 can be any argumental DP and DP2 can be any DP predicate.
In specificational clauses (SCs), there is a restriction on indefinite subjects.
\ex.
\a.\textbf{Specificational}\\
*A book is \textit{War \& Peace}.
\b. \textbf{Predicational}\\
\textit{War \& Peace} is a book.


\section{Previous work on SC subjects}
\subfile{LitReview}
\section{Theoretical Background}
\subfile{TheoryBackground}
\section{The Contrastive Topic requirement on SC subjects}
\subfile{MainArgument}
\section{Residual Data}
\subfile{Leftovers}
\section{Conclusions}
\printbibliography
\end{document}


