%        File: GPFinal.tex
%     Created: Wed Jun 17 01:00 PM 2015 E
% Last Change: Wed Jun 17 01:00 PM 2015 E
%
% arara: pdflatex: {options: "-draftmode"}
% arara: biber
% arara: pdflatex: {options: "-draftmode"}
% arara: pdflatex: {options: "-file-line-error-style"}
\documentclass[letterpaper]{article}
\usepackage{subfiles}
\usepackage[margin=1in]{geometry}
\usepackage[backend=biber,style=authoryear-comp,useprefix=false]{biblatex}

\usepackage{setspace}

\usepackage{linguex}

\usepackage[normalem]{ulem}

\usepackage{stmaryrd}
\usepackage[]{amsmath}
\usepackage{amsfonts}
\usepackage{amssymb}
\usepackage{forest}

\forestset{tree defaults/.style={for tree={parent anchor=south, child anchor=north},every tree node/.style={align=center,anchor=north},level/.style={sibling distance=50mm/#1},baseline}}

\forestset{en/.style={parent anchor=center, child anchor=center}}
\forestset{em/.style={parent anchor=north west, child anchor=north west}}
\forestset{el/.style={parent anchor=north, child anchor=north}}

\usetikzlibrary{positioning}
\DeclareNameFormat{labelname:poss}{% Based on labelname from biblatex.def
  \ifcase\value{uniquename}%
    \usebibmacro{name:last}{#1}{#3}{#5}{#7}%
  \or
    \ifuseprefix
      {\usebibmacro{name:first-last}{#1}{#4}{#5}{#8}}
      {\usebibmacro{name:first-last}{#1}{#4}{#6}{#8}}%
  \or
    \usebibmacro{name:first-last}{#1}{#3}{#5}{#7}%
  \fi
  \usebibmacro{name:andothers}%
  \ifnumequal{\value{listcount}}{\value{liststop}}{'s}{}}

\DeclareFieldFormat{shorthand:poss}{%
  \ifnameundef{labelname}{#1's}{#1}}

\DeclareFieldFormat{citetitle:poss}{\mkbibemph{#1}'s}

\DeclareFieldFormat{label:poss}{#1's}

\newrobustcmd*{\posscitealias}{%
  \AtNextCite{%
    \DeclareNameAlias{labelname}{labelname:poss}%
    \DeclareFieldAlias{shorthand}{shorthand:poss}%
    \DeclareFieldAlias{citetitle}{citetitle:poss}%
    \DeclareFieldAlias{label}{label:poss}}}

\newrobustcmd*{\posscite}{%
  \posscitealias%
  \textcite}

\newrobustcmd*{\Posscite}{\bibsentence\posscite}

\newrobustcmd*{\posscites}{%
  \posscitealias%
  \textcites}

\bibliography{GP2}
\newcommand\quelle[1]{{%
  \unskip\nobreak\hfil\penalty50
  \hskip2em\hbox{}\nobreak\hfil#1%
  \parfillskip=0pt \finalhyphendemerits=0 \par}}

\title{Specifying why a doctor isn't Mary\\\textit{Draft}}
\author{Daniel Milway}

\begin{document}
\maketitle
\doublespacing
\textit{\textbf{Note:} Throughout this draft you will see <++> or <+Some Words+> these are artifacts of my \LaTeX editor.
They are notes to myself to remind me of parts that I need to complete.
}
\section{Introduction}
The specificational clause is one of the varieties of copular clauses identified by \textcite{higgins1973pseudo}, characterized by an apparently predicative DP in subject position (DP1) and an argumental DP in post-copular position (DP2).
They contrast with predicational copular clauses in which DP1 is argumental and DP2 is predicational.
\ex.
\a.\textbf{Specificational}\\
My favourite book is \textit{War \& Peace}.
\b. \textbf{Predicational}\\
\textit{War \& Peace} is my favourite book.

In predicational clauses, DP1 can be any argumental DP and DP2 can be any DP predicate.
In specificational clauses (SCs), there is a restriction on indefinite subjects.
\ex.\label{ex:TheData}
\a.\textbf{Specificational}\\
*A book is \textit{War \& Peace}.
\b. \textbf{Predicational}\\
\textit{War \& Peace} is a book.

The restriction on indefinite SC subjects, which this paper addresses, presents a puzzle for any syntactic or semantic analysis of SCs because it is not an absolute ban on indefinite DPs in subject position.
Rather, as I will discuss in section \ref{sec:LitReview}, the fact that some indefinite DPs are able to act as SC subjects, as demonstrated below in \Next, means that, before we can adduce the indefinite restriction as evidence for or against a particular analysis of SCs, we must first understand its provenance.
\ex.\ref{ex:TheData} 
\a.
\a.* A doctor is Mary.
\b. A newly-minted doctor is Mary.
\z.
\b.
\a.* A linguist is Eric Lenneberg.
\b. An underrated linguist is Eric Lenneberg.
\z.
\b.
\a.* A building is Robarts.
\b. A buiding no-one likes is Robarts.
\z.

In the remainder of this paper I argue that the indefinite restriction is pragmatic in nature.
Specifically, I claim that there is a requirement that SC subjects contain both ``new'' and ``old'' information.
It is important to note that this claim is not only about \textit{indefinite} SC subjects, but SC subjects in general.
As such, I will demonstrate that definite DPs also meet this requirement.
In section \ref{sec:TheoryBackground}, I introduce some of the theoretical machinery required for my analysis and in section \ref{sec:MainArgument} I present my main claim and the arguments in its favour.
Section \ref{sec:Leftovers} addresses some residual issues of my analysis and section \ref{sec:Conclusion} concludes.
\section{The place of the indefinite restriction in linguistic theory}\label{sec:LitReview}
\subfile{LitReview}
\section{Theoretical Background}\label{sec:TheoryBackground}
\subfile{TheoryBackground}
\section{The Contrastive Topic requirement on SC subjects}\label{sec:MainArgument}
\subfile{MainArgument}
\section{A General Constraint on CT-Foc structures}
\subfile{GeneralConstraint}
\section{Residual Data}\label{sec:Leftovers}
\subfile{Leftovers}
\section{Conclusions}\label{sec:Conclusion}
In this paper I have presented a pragmatic account of the restriction on indefinite SC subjects.
According to this account, SC subjects must contain but not be a CT-marked constituent.
I have shown how this captures the fact that simple indefinites cannot be SC subjects.
Furthermore, I argued that the indefinite restriction can be derived from a more general constraint on CT-Foc structured sentences that requires them to include given/presupposed material.

\printbibliography
\end{document}


