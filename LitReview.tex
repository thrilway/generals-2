%        File: LitReview.tex
%     Created: Wed Aug 26 12:00 PM 2015 E
% Last Change: Wed Aug 26 12:00 PM 2015 E
%
% arara: pdflatex: {options: "-draftmode"}
% arara: biber
% arara: pdflatex: {options: "-draftmode"}
% arara: pdflatex: {options: "-file-line-error-style"}
\documentclass[GPFinal]{subfiles}

\begin{document}
Though the puzzle of indefinite SC subjects is well-known, it is not often discussed in any depth.
It is most commonly used as evidence against the inversion analysis of SCs.
Although the proper syntactic/semantic analysis of SCs is beyond the scope of this paper, the restriction on indefinite SC subjects is often used to argue for one of the two basic analyses, which I will call the \textit{inversion} and \textit{equative} analyses, respectively.
As such, I will briefly outline the analyses and how indefinite subjects fit into them.

The inversion analysis is argued for explicitly by \textcite{mikkelsen2004specifying} and \textcite{moro1997raising}, and states that that predicative copular clauses and SCs have identical underlying structures.
According to this analysis, the two sentences in \ref{ex:SCPCPair} are each derived from the same small clause structure, given in \ref{ex:CopUnderlying}, and differ in which constituent of the the small clause is raised.
Predicative clauses surface when the argument raises, and SCs surface when the predicate raises.
\ex.\label{ex:SCPCPair}
\a.\label{ex:SCPCPairPC} Ian is my favourite singer.
\b.\label{ex:SCPCPairSC} My favourite singer is Ian.

\ex.\label{ex:CopUnderlying} $[_{VP} \text{be} [_{SC} [_{Arg} \text{Ian}] [_{Pred} \text{my favourite singer}]]]$

Semantically, this analysis requires that SC subjects be construed as predicates (type $\langle e,t\rangle$ or higher) rather than arguments (type \textit{e}).

The equative analysis, as presented by \textcite{heycockkroch1999pseudocleft}, says that both DPs in SCs are type $e$ and the copula serves to equate them.
In \ref{ex:SCPCPairSC}, then, \textit{my favourite singer} and \textit{Ian} each refers to an individual, and the copula says that they refer to the same individual.
\textcite{heycockkroch1999pseudocleft} use the restriction on indefinite subjects to argue that SC subjects cannot be construed as predicates.
If SC subjects were inverted predicates, the argument goes, we would expect all predicative phrases, including indefinite descriptions, to be acceptable.
What \textcite{heycockkroch1999pseudocleft} fail to consider, however, is the fact that indefinites can sometimes be SC subjects as shown above in \ref{ex:GoodSCs}.

The restriction on indefinites, then is a fact that must be explained or allowed for in any syntactic/semantic analysis of SCs.

\subsection{\textcite{mikkelsen2004specifying}}
As far as I can tell, Line Mikkelsen's dissertation contains the only attempt to define the restriction on indefinite SC subjects.
Though she admits that her attempt falls short of a proper explication of the restriction, the attempt itself provides an excellent starting point for my attempt.

After arguing in favour of a predicate inversion analysis of SC, Mikkelsen considers the restriction on indefinites and concedes that, as \textcite{heycockkroch1999pseudocleft} argue, it is not predicted by the inversion analysis.
She does not concede, however, that it represents a strong argument against the inversion analysis, because the restriction on indefinite subjects is pragmatic rather than syntactic/semantic in nature.
\end{document}


