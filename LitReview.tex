%        File: LitReview.tex
%     Created: Wed Aug 26 12:00 PM 2015 E
% Last Change: Wed Aug 26 12:00 PM 2015 E
%
% arara: pdflatex: {options: "-draftmode"}
% arara: biber
% arara: pdflatex: {options: "-draftmode"}
% arara: pdflatex: {options: "-file-line-error-style"}
\documentclass[GPFinal]{subfiles}

\begin{document}
Though the puzzle of indefinite SC subjects is well-known, it is not often discussed in any depth.
It is most commonly used as evidence against the inversion analysis of SCs.
\subsection{The place of the indefinite restriction in linguistic theory}
Though rarely discussed in much depth, the restriction on indefinite SC subjects is often exploited for evidence in the debate over the proper syntactic/semantic analysis of SCs.

As such, I will briefly outline the analyses and how indefinite subjects fit into them.

The inversion analysis is argued for explicitly by \textcite{mikkelsen2004specifying} and \textcite{moro1997raising}, and states that that predicative copular clauses and SCs have identical underlying structures.
According to this analysis, the two sentences in \ref{ex:SCPCPair} are each derived from the same small clause structure, given in \ref{ex:CopUnderlying}, and differ in which constituent of the the small clause is raised.
Predicative clauses surface when the argument raises, and SCs surface when the predicate raises.
\ex.\label{ex:SCPCPair}
\a.\label{ex:SCPCPairPC} Ian is my favourite singer.
\b.\label{ex:SCPCPairSC} My favourite singer is Ian.

\ex.\label{ex:CopUnderlying} $[_{VP} \text{be} [_{SC} [_{Arg} \text{Ian}] [_{Pred} \text{my favourite singer}]]]$

Semantically, this analysis requires that SC subjects be construed as predicates (type $\langle e,t\rangle$ or higher) rather than arguments (type \textit{e}).

The equative analysis, as presented by \textcite{heycockkroch1999pseudocleft}, says that both DPs in SCs are type $e$ and the copula serves to equate them.
In \ref{ex:SCPCPairSC}, then, \textit{my favourite singer} and \textit{Ian} each refers to an individual, and the copula says that they refer to the same individual.
\textcite{heycockkroch1999pseudocleft} use the restriction on indefinite subjects to argue that SC subjects cannot be construed as predicates.
If SC subjects were inverted predicates, the argument goes, we would expect all predicative phrases, including indefinite descriptions, to be acceptable.

As I will describe in more detail in section \ref{sec:Mikkelsen}, \textcite{mikkelsen2004specifying} proposes that pragmatic factors are responsible for the indefinite restriction.
Specifically, SC's have a fixed information structure, requiring their subjects to be topics, a role which indefinites are not well-suited for.
\textcite{heycock2012specification}, responds to Mikkelsen's analysis and data  by arguing that, rather than a requirement that SC subjects be topics, the indefinite restriction is actually a restriction on \textit{weak} indefinites (\textit{i.e.}, DPs headed by weak determiners) as SC subjects.
This proposal may, in fact, be correct, although it does not seem explanatory for two reasons.
First, the exact nature of the contrast between weak and strong determiners is not well understood.
Second, the terms \textit{weak} and \textit{strong} in this context, properly refer to interpretations rather than lexical items.
A determiner is called \textit{strong} if it is always interpreted as strong, while \textit{weak determiners} can be interpreted as either weak or strong depending on the context.

Supposing we take \posscite{heycock2012specification} analysis to be true, the question changes from ``Why is the indefinite X a licit SC subject, while Y is illicit?'' to ``Why can X receive a strong interpretation, while Y cannot?''.

The restriction on indefinites, then is a fact that must be explained or allowed for in any syntactic/semantic analysis of SCs.

\subsection{\textcite{mikkelsen2004specifying}}\label{sec:Mikkelsen}
As far as I can tell, Line Mikkelsen's dissertation contains the only attempt to define the restriction on indefinite SC subjects.
Though she admits that her attempt falls short of a proper explication of the restriction, the attempt itself provides an excellent starting point for my attempt.

After arguing in favour of a predicate inversion analysis of SC, Mikkelsen considers the restriction on indefinites and concedes that, as \textcite{heycockkroch1999pseudocleft} argue, it is not predicted by the inversion analysis.
She does not concede, however, that it represents a strong argument against the inversion analysis.
The restriction on indefinites would only be strong evidence against an inversion analysis if it were a categorical restriction, which it is not.

Mikkelsen demonstrates the non-categorical nature of the restriction with the following examples
\ex.\label{ex:MikkPhilosopher} \textbf{A philosopher who seems to share the Kiparskys' intuition on some factive predicates} is Unger (1972) who argues that \dots\footnote{\textcite[][p. 195 fn8]{delacruz1976factives} cited by \textcite{mikkelsen2004specifying}}

\ex.\label{ex:MikkSpeaker} \textbf{Another speaker at the conference} was the \textit{Times} columnist Nicholas Kristof, who got Wilson's permission to mention the Niger trip in a column.\footnote{Seymore M. Hersh ``The Stovepipe'', The New Yorker, Oct 27, 2003, p. 86 cited by \textcite{mikkelsen2004specifying}}

\ex.\label{ex:MikkEmigre} \textbf{One Iraqi \'emigr\'e who has heard from the scientists' families} is Shakir al Kha Fagi, who left Iraq as a young man and runs a successful business in the Detroit area.\footnote{Seymore M. Hersh ``The Stovepipe'', The New Yorker, Oct 27, 2003, p. 86 cited by \textcite{mikkelsen2004specifying}}

\ex.\label{ex:MikkBarcan} \textbf{A doctor who might be able to help you} is Harry Barcan.\footcite{mikkelsen2004specifying}

Since the restriction is not categorical, she argues, it is not due to a semantic type mismatch, rather it must be pragmatic in nature.

Mikkelsen points out that, unlike predicational clauses, SCs have a fixed information structure.
As demonstrated in \ref{ex:MikkQandA}, SCs are infelicitous in contexts that focus the initial DP, while predicational clauses are more flexible.
\ex.\label{ex:MikkQandA}
\a. Q: Who is the winner?\\
A1: The winner is JOHN.\hfill[Specificational]\\
A2: JOHN is the winner.\hfill[Predicational]
\b. Q: What is John?\\
A1: \#The WINNER is John.\hfill[Specificational]\\
A2: John is the WINNER.\hfill[Predicational]\\
\quelle{\parencite{mikkelsen2004specifying}}
\z.

Mikkelsen argues that this fixed information structure of SCs follows from SCs being inversion structures.
Following \textcite{birner1994information,birner1996discourse}, She assumes that the discourse function of inversion is to mark the inverted material as linking a clause to previous discourse.
The inverted material, then, must be more discourse-familiar than the post-verbal logical subject.
Mikkelsen then shows that these discourse familiarity cosiderations can explain the acceptability of \ref{ex:MikkPhilosopher}-\ref{ex:MikkBarcan}.

This pragmatic account, while sufficient to explain the acceptability of \ref{ex:MikkPhilosopher}-\ref{ex:MikkBarcan}, does not explain why the restriction on simple indefinites as SC subjects, as shown in \ref{ex:BadSCs}, seems to be categorical.
That is, even if the material in a simple indefinite is familiar, the indefinite cannot be the subject of an SC.
\ex.\label{ex:MikkFamSC} Bill is a doctor. \#A doctor is John (too).

Mikkelsen suggests that the discourse familiarity requirement of inverted material clashes with the Novelty Condition on indefinites \parencite{heim1982semantics}.
She points out, however, that this cannot be the entire story, since the Novelty Condition only requires that indefinites introduce new discourse referents.
This means that, since the two instances of \textit{a doctor} in \ref{ex:MikkFamSC} do not share a discourse referent, the Novelty Condition does not rule out the indefinite subject.

Mikkelsen also suggests that those instances of familiar yet unacceptable simple indefinite SC subjects might be infelicitous because there is a general ban on repeating indefinites, as in the example below.
\ex.\label{ex:MikkRepeat} Sally is a doctor. \#A doctor came to dinner last night.

This, however, does not seem to hold.
Utterences, such as \ref{ex:MikkRepeat}, that are barred because of repeated indefinites are made better if the first occurrence of the indefinite is modified.
If the barred utterance has an SC with an indefinite subject, as in \ref{ex:MikkFamSC}, then only changing the SC will improve it.
\ex. I know many doctors.
  \a. \#A doctor is Patrick.
  \b. A doctor came to dinner last night.
  \z.


\end{document}

