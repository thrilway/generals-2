%        File: defence.tex
%     Created: Mon Feb 22 03:00 PM 2016 E
% Last Change: Mon Feb 22 03:00 PM 2016 E
%
% arara: pdflatex: {options: "-draftmode"}
% arara: biber
% arara: pdflatex: {options: "-draftmode"}
% arara: pdflatex: {options: "-file-line-error-style"}
\documentclass[letterpaper]{article}

\usepackage[margin=1in]{geometry}
\usepackage[backend=biber,style=authoryear-comp,useprefix=false]{biblatex}

\usepackage{linguex}

\usepackage[normalem]{ulem}

\usepackage{stmaryrd}
\usepackage[]{amsmath}
\usepackage{amsfonts}
\usepackage{amssymb}
\usepackage{forest}

\forestset{tree defaults/.style={for tree={parent anchor=south, child anchor=north},every tree node/.style={align=center,anchor=north},level/.style={sibling distance=50mm/#1},baseline}}

\forestset{en/.style={parent anchor=center, child anchor=center}}
\forestset{em/.style={parent anchor=north west, child anchor=north west}}
\forestset{el/.style={parent anchor=north, child anchor=north}}

\usetikzlibrary{positioning}
\DeclareNameFormat{labelname:poss}{% Based on labelname from biblatex.def
  \ifcase\value{uniquename}%
    \usebibmacro{name:last}{#1}{#3}{#5}{#7}%
  \or
    \ifuseprefix
      {\usebibmacro{name:first-last}{#1}{#4}{#5}{#8}}
      {\usebibmacro{name:first-last}{#1}{#4}{#6}{#8}}%
  \or
    \usebibmacro{name:first-last}{#1}{#3}{#5}{#7}%
  \fi
  \usebibmacro{name:andothers}%
  \ifnumequal{\value{listcount}}{\value{liststop}}{'s}{}}

\DeclareFieldFormat{shorthand:poss}{%
  \ifnameundef{labelname}{#1's}{#1}}

\DeclareFieldFormat{citetitle:poss}{\mkbibemph{#1}'s}

\DeclareFieldFormat{label:poss}{#1's}

\newrobustcmd*{\posscitealias}{%
  \AtNextCite{%
    \DeclareNameAlias{labelname}{labelname:poss}%
    \DeclareFieldAlias{shorthand}{shorthand:poss}%
    \DeclareFieldAlias{citetitle}{citetitle:poss}%
    \DeclareFieldAlias{label}{label:poss}}}

\newrobustcmd*{\posscite}{%
  \posscitealias%
  \textcite}

\newrobustcmd*{\Posscite}{\bibsentence\posscite}

\newrobustcmd*{\posscites}{%
  \posscitealias%
  \textcites}

\bibliography{GP2}
\newcommand\quelle[1]{{%
  \unskip\nobreak\hfil\penalty50
  \hskip2em\hbox{}\nobreak\hfil#1%
  \parfillskip=0pt \finalhyphendemerits=0 \par}}

\title{Specifying why a doctor isn't Mary\\
  Generals Paper Defence
}
\author{Daniel Milway}

\begin{document}
\maketitle
\begin{itemize}
  \item Simple indefinite DPs tend to make poor subjects of specificational copular clauses (SCCs).
\end{itemize}
\ex.\label{ex:BadSCs} 
\a.* A doctor is Mary.
\b.* A building is Robarts Library.
\b.* A linguist is Eric Lenneberg
\z.

\begin{itemize}
  \item More complex indefinites are better SCC subjects.
\end{itemize}
\ex.\label{ex:GoodSCCs}
\a. A newly-minted doctor is Mary.
\b. A building on campus no-one likes is Robarts Library.
\b. An underrated figure in the history of generative grammar is Eric Lenneberg.
\z.

\begin{itemize}
  \item My solution:
\end{itemize}
\ex.\label{def:CTCondition} \textbf{The Contrastive Topic requirement on Specificational Clauses}\\
A clause of the form $X$ \textsc{be} $Y$ is a licit specificational clause iff
\a. $\llbracket X\rrbracket(\llbracket Y\rrbracket)$ is defined,
\b. $Y$ is F-marked, \parencite{mikkelsen2004specifying}
\b.\label{def:Contain} Some constituent of $X$ is CT-marked, and
\b.\label{def:NotBe} $X$ is not CT-Marked.

\begin{itemize}
  \item \Last[a] and \Last[b] are assumed.
  \item \Last[c] rules in the SCCs in \ref{ex:GoodSCCs}, provided they are pronounced with appropriate intonation
\end{itemize}
\ex. Who is a veteran doctor?
\a. A NEWLY$_{CT}$-minted doctor is Mary.
\b.\# A newly-minted DOCTOR$_{CT}$ is Mary.
\z.

\begin{itemize}
  \item \LLast[d] rules out the strings in \ref{ex:BadSCs}
\end{itemize}
\ex. *A doctor is Mary

\begin{itemize}
  \item By \ref{def:Contain}, part of \textit{a doctor} must be CT-Marked.
  \item The indefinite article \textit{a} is not CT-Marked, so \textit{doctor} must be.
  \item \textit{a [doctor]}$_{CT} \approx$ \textit{[a doctor]}$_{CT}$
\end{itemize}
\section{Syntactic Instantiation (How?)}
\begin{itemize}
  \item Consider a particularly difficult SCC in \Next.
\end{itemize}
\ex. $[_{DP}$A figure $[_{PP}$in the history $[_{PP}$of generative$_{CT}$ grammar $]]]$ is Eric Lenneberg.

\subsection{CT Projection?}
\begin{itemize}
  \item A CT diacritic on an adjective embedded in an adjunct licenses the the indefinite subject.
  \item Assuming CT projects like focus, we don't expect it to project sufficiently in \Last.
\end{itemize}
\ex. \textbf{Focus Projection} \parencite{selkirk1996sentence}\\
\a. F-marking of the \textit{head} of a phrase licenses the F-marking of the phrase.
\b. F-marking of an \textit{internal argument} of a head licenses the F-marking of the head.

\begin{itemize}
  \item Selkirk explicitly argues that F-marking on adjuncts does not project.
  \item We cannot rely on CT projection to license indefinite SCC subjects.
\end{itemize}
\subsection{Agree + Pied-piping?}
\begin{itemize}
  \item Using Agree to license indefinite SCC subjects such as \LLast is also problematic.
  \item Assuming the Agree probe is seeking a CT diacritic/feature, it shouldn't find it in \LLast.
  \item The CT-Marked constituent is in at least one island.
  \item Agree cannot reliably license indefinite SCC subjects.
\end{itemize}
\subsection{Free Merge + Interface constraints}
\begin{itemize}
  \item 
\end{itemize}<++>
\section{Pragmatic Grounding (Why?)}

\end{document}


