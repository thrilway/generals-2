%        File: defence.tex
%     Created: Mon Feb 22 03:00 PM 2016 E
% Last Change: Mon Feb 22 03:00 PM 2016 E
%
% arara: pdflatex: {options: "-draftmode"}
% arara: biber
% arara: pdflatex: {options: "-draftmode"}
% arara: pdflatex: {options: "-file-line-error-style"}
\documentclass[letterpaper]{article}

\usepackage[margin=1in]{geometry}
\usepackage[backend=biber,style=authoryear-comp,useprefix=false]{biblatex}

\usepackage{linguex}

\usepackage[normalem]{ulem}

\usepackage{stmaryrd}
\usepackage[]{amsmath}
\usepackage{amsfonts}
\usepackage{amssymb}
\usepackage{forest}

\forestset{tree defaults/.style={for tree={parent anchor=south, child anchor=north},every tree node/.style={align=center,anchor=north},level/.style={sibling distance=50mm/#1},baseline}}

\forestset{en/.style={parent anchor=center, child anchor=center}}
\forestset{em/.style={parent anchor=north west, child anchor=north west}}
\forestset{el/.style={parent anchor=north, child anchor=north}}

\usetikzlibrary{positioning}
\DeclareNameFormat{labelname:poss}{% Based on labelname from biblatex.def
  \ifcase\value{uniquename}%
    \usebibmacro{name:last}{#1}{#3}{#5}{#7}%
  \or
    \ifuseprefix
      {\usebibmacro{name:first-last}{#1}{#4}{#5}{#8}}
      {\usebibmacro{name:first-last}{#1}{#4}{#6}{#8}}%
  \or
    \usebibmacro{name:first-last}{#1}{#3}{#5}{#7}%
  \fi
  \usebibmacro{name:andothers}%
  \ifnumequal{\value{listcount}}{\value{liststop}}{'s}{}}

\DeclareFieldFormat{shorthand:poss}{%
  \ifnameundef{labelname}{#1's}{#1}}

\DeclareFieldFormat{citetitle:poss}{\mkbibemph{#1}'s}

\DeclareFieldFormat{label:poss}{#1's}

\newrobustcmd*{\posscitealias}{%
  \AtNextCite{%
    \DeclareNameAlias{labelname}{labelname:poss}%
    \DeclareFieldAlias{shorthand}{shorthand:poss}%
    \DeclareFieldAlias{citetitle}{citetitle:poss}%
    \DeclareFieldAlias{label}{label:poss}}}

\newrobustcmd*{\posscite}{%
  \posscitealias%
  \textcite}

\newrobustcmd*{\Posscite}{\bibsentence\posscite}

\newrobustcmd*{\posscites}{%
  \posscitealias%
  \textcites}

\bibliography{GP2}
\newcommand\quelle[1]{{%
  \unskip\nobreak\hfil\penalty50
  \hskip2em\hbox{}\nobreak\hfil#1%
  \parfillskip=0pt \finalhyphendemerits=0 \par}}

\title{Specifying why a doctor isn't Mary\\
  Generals Paper Defence
}
\author{Daniel Milway}

\begin{document}
\maketitle
\maketitle
\section{The puzzle: the indefinite restriction}
\begin{itemize}
  \item Simple indefinite DPs tend to make poor subjects of specificational copular clauses (SCCs).
\end{itemize}
\ex.\label{ex:BadSCs} 
\a.* A doctor is Mary.
\b.* A building is Robarts Library.
\b.* A linguist is Eric Lenneberg
\z.

\begin{itemize}
  \item More complex indefinites are better SCC subjects.
\end{itemize}
\ex.\label{ex:GoodSCCs}
\a. A newly-minted doctor is Mary.
\b. A building on campus no-one likes is Robarts Library.
\b. An underrated figure in the history of generative grammar is Eric Lenneberg.
\z.

\ex.[\textbf{Question:}] What is the nature of the indefinite restriction? How is it best characterized?

\ex.[\textbf{My Answer:}] An SCC subject must contain, but not be, a contrastive topic marked constituent.

\subsection{Outline}
\begin{itemize}
  \item \posscite{mikkelsen2004specifying} pragmatic analysis of SCC subjects. 
  \item Theory background \parencite{buring2003d}
  \item Arguments for my solution
  \item Ruling in \ref{ex:GoodSCCs} and ruling out \ref{ex:BadSCs}
  \item Describing one reason the solution is a bit odd
  \item Conclusion
\end{itemize}

\section{\posscite{mikkelsen2004specifying} Analysis}
\begin{itemize}
  \item Mikkelsen argues for a discourse-pragmatic analysis of the indefinite restriction.
    \begin{itemize}
      \item This is in contrast to a semantic analysis argued for \textit{e.g.} by \textcite{heycock2012specification}. \ref{sec:Heycock}) 
    \end{itemize}
  \item She observes that SCCs have a rigid information structure: DP2 must be focus/DP1 cannot be focus.
\end{itemize}
\ex.\label{ex:SCCIS} 
\a. Who is the winner?
\a. The winner is John. \hfill (Specificational)
\b. John is the winner. \hfill (Predicational)
\z.
\b. What is John?
\a.\# The winner is John. \hfill (Specificational)
\b. John is the winner. \hfill (Predicational)

\begin{itemize}
  \item It follows that DP1 must be topical. (\textit{i.e.}, discourse given)
  \item Indefinites introduce new discourse referents. (\posscite{heim1982semantics} Novelty Condition)
  \item The indefinite restriction, then, is due to a clash between the topichood requirement on SCC subjects and the Novelty Condition on indefinites.
  \item Those indefinites that are licit SCC subjects are relatively given compared to DP2.
\end{itemize}
\textbf{However:}
\begin{itemize}
  \item Simple indefinites (\textit{a(n)} N) cannot be SCC subjects regardless of how given/topical they are.
\end{itemize}
\ex. Bill is a doctor. \#A doctor is John (too) \parencite[][p236]{mikkelsen2004specifying}

\begin{itemize}
  \item Mikkelsen leaves this puzzle open, suggesting the unacceptibility of \Last may be related to the general infelicity of two adjecent instances of a DP as in \Next.
\end{itemize}
\ex. Sally is a doctor. \#A doctor came to dinner last night. \parencite[][p236]{mikkelsen2004specifying}

\begin{itemize}
  \item The second sentence in \Last is acceptable if \textit{a doctor} is not uttered in the previous sentence.
  \item The SCC in \LLast cannot be made acceptable.
\end{itemize}
\ex. I know some doctors.
\a. A doctor came to dinner last night.
\b.\# A doctor is John.
\z.

\begin{itemize}
  \item If we require that \textit{part} of the SCC subject be marked as \textit{contrastive} topic, we can capture the data.
\end{itemize}
\section{Theoretical Background: Contrastive Topics}
\begin{itemize}
  \item The notion of Contrastive Topic (CT) comes from \posscite{buring2003d} analysis of \posscite{jackendoff1972Ssemantics} A- and B-Accents
\end{itemize}
\ex. (Well, what about FRED? What did HE eat?)\\
FRED$_B$ ate the BEANS$_A$.

\ex. (Well, what about the BEANS? Who ate THEM?)\\
FRED$_A$ ate the BEANS$_B$.

\begin{itemize}
  \item The A-Accent, which corresponds to the Wh-element in each example marks Focus.
  \item \textcite{buring2003d} refers to the B-accented element as the CT (cf. \posscite{roberts2012information} \textit{dependant focus})
  \item The function of CT, like that of Focus, is to generate alternatives.
  \item Focus alternatives are representable as sets of propositions \parencite{rooth1992theory} equivalent to the interpretations of questions \parencite{groenendijkstokhof1996questions}
  \item CT alternatives are representable as sets of sets of propositions.
\end{itemize}
\ex. 
\a. $\llbracket\text{FRED}_B\text{ ate the BEANS}_A\rrbracket^{f} = 
\begin{Bmatrix}
  \text{Fred ate the corn}\\
  \text{Fred ate the beans}\\
  \text{Fred ate the apples}\\
  \cdots
\end{Bmatrix}
= \llbracket\text{What did Fred eat?}\rrbracket
$
\b.$\llbracket\text{FRED}_B\text{ ate the BEANS}_A\rrbracket^{ct} = 
\begin{Bmatrix}
  \text{What did Fred eat?}\\
  \text{What did Mary eat?}\\
  \text{What did Robin eat?}\\
  \cdots
\end{Bmatrix}
= \llbracket\text{What did who eat?}\rrbracket
$

\begin{itemize}
  \item B\"uring represents these interpretations as d(iscourse)-trees
\end{itemize}
\ex. $\llbracket\text{FRED}_{CT}\text{ ate the BEANS}_{F}\rrbracket^{ct}$\\
\begin{forest}
  tree defaults
  [What did who eat?
    [What did Mary eat?]
    [What did Robin eat?]
    [\textbf{What did Fred eat?}
      [Fred ate the corn]
      [Fred ate the apples]
      [\textbf{Fred ate the beans}]
    ]
  ]
\end{forest}

\begin{itemize}
  \item Generally, CT-Foc structures identify a question and superquestion that they answer.
  \item CT-Foc structures have multiple discourse functions that the d-trees can represent.
\end{itemize}
\ex.\label{ex:ChinaCTF}
\a.
\a.[A:] When are you going to China? \hfill \parencite{roberts2012information}
\b.[B:] I'm going to [China]$_{CT}$ in [April]$_F$.
\z.
\b.
\begin{forest}
  tree defaults
  [When are you going to which place?
    [When are you going to \ldots?]
    [\textbf{When are you going to China?}
      [\textbf{April}]
    ]
  ]
\end{forest}
\z.

\ex.\label{ex:CaftansCTF}
\a.
\a.[A:] What did the pop stars wear? \hfill \parencite{buring2003d}
\b.[B:] The [female]$_{CT}$ pop stars wore [caftans]$_F$.
\z.
\b.
\begin{forest}
  tree defaults
  [\textbf{What did the pop stars wear?}
    [What did the male pop stars wear?]
    [What did the female pop stars wear?
      [\textbf{The female pop stars wore caftans.}]
    ]
  ]
\end{forest}
\z.

\section{Contrastive is the right kind of topic for SCC subjects}
\begin{itemize}
  \item SCC subjects are topics, but neither givenness nor aboutness are sufficient
\end{itemize}
\ex.[\textbf{Hypothesis 1: }] SCC subjects must be givenness topics

\begin{itemize}
  \item If SCC subjects must be relatively given, then fully given subjects ought to be preferred.
  \item Fully given subjects are infelicitous, partially given subjects are acceptable
\end{itemize}
\ex. Many philosophers have written about the mind-body problem.
\a.\# A philosopher who has written about the mind-body problem is Chomsky.
\b. A modern philosopher who has written about the mind-body problem is Chomsky.

\begin{itemize}
  \item (Relative) givenness isn't the factor that allows a DP to be an SCC subject.
\end{itemize}
\ex.[\textbf{Hypothesis 2: }] SCCs must be about their subjects.

\begin{itemize}
  \item A sentence's topic, is what the sentence is about.
\end{itemize}
\ex.\textbf{Test for aboutness \parencite{reinhart1981pragmatics}: }\\
If \textit{John saw Bill} is about \textbf{John}, then\\
\textit{Mary said John saw Bill} = \textit{Mary said about John that he saw Bill} 

\begin{itemize}
  \item It is permissible for an SCC to be about only part of its subject.
\end{itemize}
\ex.[\textbf{Assumption: }] SCC subjects are only interpreted \textit{de dicto}.

\ex. (\textbf{Context:} \textit{David Bowie is my favourite singer.} is True )\\
My favourite singer is Ziggy Stardust
\a. Mary said my favourite singer is Ziggy Stardust (\textit{de re}/\textit{de dicto})
\b. Mary said about singers that my favourite is Ziggy Stardust (*\textit{de re}/\textit{de dicto})

\begin{itemize}
  \item There seems to be no requirement that SCCs be about their subject DPs.
\end{itemize}
\textbf{My Proposal}
\begin{itemize}
  \item The licit SCC subjects in \ref{ex:GoodSCCs} are most naturally pronounced with intonational stress on some component.
\end{itemize}
\ex.
\a. A NEWLY-minted doctor is Mary.
\b. A newly-MINTED doctor is Mary.
\b. A newly-minted DOCTOR is Mary.
\b.\# A newly-minted doctor is Mary.
\z.

\begin{itemize}
  \item Each of the intonation patterns in \Last[a--c] can be used in a distinct set of discourse contexts.
  \item The discourse contexts of SCCs are identical to canonical CT-Foc structures.
\end{itemize}

\ex. Who is a veteran doctor?
\a. A NEWLY-minted doctor is Mary.
\b.\# A newly-minted DOCTOR is Mary.
\z.

\begin{itemize}
  \item The intonation pattern in \Last[b] is infelicitous because it incorrectly marks \textit{doctor} as novel and \textit{newly-minted} as given.
\end{itemize}
\ex.\# A newly-minted DOCTOR is Mary.\\
\begin{forest}
  tree defaults
  [Who is a newly-minted what?
    [\ldots]
    [Who is a newly minted doctor?
      [\ldots]
      [Is Mary a newly minted doctor?
	[\textbf{A newly-minted DOCTOR is Mary.}]
      ]
    ]
  ]
\end{forest}

\ex. A NEWLY-minted doctor is Mary.\\
\begin{forest}
  tree defaults
  [Who is a doctor of what level of experience?
    [\textbf{Who is a veteran doctor?}
      [\dots]
    ]
    [\dots]
    [Who is a newly minted doctor?
      [\ldots]
      [Is Mary a newly minted doctor?
	[\textbf{A newly-minted DOCTOR is Mary.}]
      ]
    ]
  ]
\end{forest}

\ex. \textbf{CT condition on SCC subjects (part one):}\\
Some constituent of the subject of an SCC must be CT-marked.

\begin{itemize}
  \item Ruling out simple indefinite SCCs can be done with a simple extension of \Last.
\end{itemize}
\ex. \textbf{CT condition on SCC subjects (part two):}\\
The entire SCC subject must not be CT-marked

\begin{itemize}
  \item Consider the CT-Foc structure of \Next
\end{itemize}
\ex.* A doctor is Mary.

\begin{itemize}
  \item DP2 is focused. \parencite{mikkelsen2004specifying}
  \item Some constituent of DP1 is CT-marked.
  \item te indefinite article is not CT marked.
  \item \textit{doctor} must be CT-marked.
  \item Assuming the indefinite article is semantically vacuous, \textit{a [doctor]}$_{CT}$ is equivalent to \textit{[a doctor]}$_{CT}$.
  \item So, \Last requires that the entire SCC subject be CT-marked, violating the proposed CT condition on SCCs. 
\end{itemize}
\subsection{Apparent counter-examples}
\subsubsection{Stressed \textit{a(n)}}
\begin{itemize}
  \item A version of \Last with a stressed indefinite article ([ej] in my dialect) is more acceptable than \Last as is.
\end{itemize}
\ex.? A$_{CT}$ doctor is Mary.

\begin{itemize}
  \item The stressed indefinite article generally marks a contrast with the definite article.
\end{itemize}
\ex.
\a.[A: ] You must be the professor.
\b.[B: ] I'm A$_F$ professor.

\begin{itemize}
  \item This is predicted by the proposed CT condition on SCC subjects.
\end{itemize}
\ex. \textbf{CT condition on SCC subjects:}
\a.[(i)] Some constituent of the subject of an SCC must be CT-marked.
\b.[(ii)] The entire SCC subject must not be CT-marked.
\z.

\begin{itemize}
  \item \textit{a}$_{CT}$\textit{ doctor} $\neq$ \textit{[a doctor]}$_{CT}$
  \item a constituent of the SCC subject rather than the entire subject is CT-marked.
\end{itemize}
\subsubsection{\textit{One} and \textit{another}}
\begin{itemize}
  \item ``Simple'' indefinite DPs headed by \textit{one} or \textit{another} can be SCC subjects.
\end{itemize}
\ex. 
\a. One doctor is Mary.
\b. Another doctor is Molly.

\begin{itemize}
  \item The term ``simple indefinite'' for these examples is a misnomer.
  \item \textit{One} and \textit{another} are complex determiners.
  \item \textcite{kayne2015one} argues that \textit{one} is composed of the indefinite determiner and a ``singular classifier.''
    \begin{itemize}
      \item \textit{one} = \textit{w-}+\textit{an}
    \end{itemize}
  \item \textcite{heim1991reciprocity} provides an analysis of \textit{another} as \textit{an}+\textit{other}
  \item In both cases the determiner-like element bears a contentful element that can be CT-marked
\end{itemize}
\ex.
\a. w-$_{CT}$-an doctor is Mary.
\b. an- -other$_{CT}$ doctor is Molly.

\section{Syntactic Instantiation (How?)}
\begin{itemize}
  \item If the CT condition is a condition on licensing specificational subject, it should be expressible in syntactic terms
  \item Mikkelsen proposes that SCCs arise because of the presence of an uninterpretable Topic feature on T
  \item Movement of a topical DP to spec-T satisfies [\textit{u}Top]
\end{itemize}
\ex.[\textbf{Hypothesis: }] SCCs arise when T bears a [\textit{u}CT] feature that is satisfied by movement to its specifier. 

\begin{itemize}
  \item Either movement makes the [CT] feature local to the [\textit{u}CT] feature, or movement is triggered by long distance Agree
  \item Neither of these strategies work.
  \item Consider a particularly difficult SCC in \Next.
\end{itemize}
\ex. $[_{DP}$A figure $[_{PP}$in the history $[_{PP}$of generative$_{CT}$ grammar $]]]$ is Eric Lenneberg.

\subsection{CT Projection?}
\begin{itemize}
  \item A [CT] feature on an adjective embedded in an adjunct licenses the the indefinite subject.
  \item The [CT] feature can be made local to T if it projects to the DP it is contained in.
  \item Assuming CT projects like focus, we don't expect it to project sufficiently in \Last.
\end{itemize}
\ex. \textbf{Focus Projection} \parencite{selkirk1996sentence}\\
\a. F-marking of the \textit{head} of a phrase licenses the F-marking of the phrase.
\b. F-marking of an \textit{internal argument} of a head licenses the F-marking of the head.

\begin{itemize}
  \item Selkirk explicitly argues that F-marking on adjuncts does not project.
  \item We cannot rely on CT projection to license indefinite SCC subjects.
\end{itemize}
\subsection{Agree + Pied-piping?}
\begin{itemize}
  \item Using Agree to license indefinite SCC subjects such as \LLast is also problematic.
  \item Assuming the Agree probe is seeking a CT diacritic/feature, it shouldn't find it in \LLast.
  \item The CT-Marked constituent is in at least one island.
  \item Agree cannot reliably license indefinite SCC subjects.
\end{itemize}
\subsection{Free Merge + Interface constraints}
\begin{itemize}
  \item A promising direction
  \item Requires us to be more explicit about our theory of the interfaces.
  \item Eventually we have to investigate our theory of discourse pragmatics to ask why such a constraint should hold.
\end{itemize}
\section{Conclusion}
\begin{itemize}
  \item The indefinite restriction is pragmatic in nature.
  \item Following \posscite{mikkelsen2004specifying} observation, I analysed the SCC subject as topical (in some sense).
  \item SCC subjects must contain but not be CT-Marked constituents.
  \item This generalization predicts which indefinite DPs are acceptable SCC subjects in what discourse contexts, and rules out simple indefinite SCC subjects.
  \item It's an odd generalization, though.
\end{itemize}
\printbibliography
\appendix
\section{\textcite{heycock2012specification}}\label{sec:Heycock}
\begin{itemize}
  \item Heycock argues that the indefinite restriction is a ban on weak-quantified DPs as SCC subjects.
  \item The argument is based on two parallels 
\end{itemize}
\subsection{Parallel 1}
\begin{itemize}
  \item English SCC subjects cannot be focused (see \ref{ex:SCCIS}, above), and neither can German scrambled objects. \parencite{lenerz1977zur}
\end{itemize}
\ex.
\ag.Wem hat Peter das Futter gegeben?\\
who.\textsc{dat} has Peter the.\textsc{acc} food given\\
``Who has Peter given the food?''
\ag. Peter hat der Katze das Futter gegeben.\\
Peter has the.\textsc{dat} cat the.\textsc{acc} food given\\
``Peter has given the cat the food''\hfill[Default order]
\bg. Peter hat das Futter der Katze gegeben.\\
Peter has the.\textsc{acc} food the.\textsc{dat} cat given\\
``Peter has given the food to the cat''\hfill[Scrambled order]
\z.
\bg. Was hat Peter der Katze gegeben?\\
what.\textsc{acc} has Peter the.\textsc{dat} cat given\\
``What has Peter given (to) the cat?''
\ag. Peter hat der Katze das Futter gegeben.\\
Peter has the.\textsc{dat} cat the.\textsc{acc} food given\\
``Peter has given the cat the food''\hfill[Default order]
\bg.\# Peter hat das Futter der Katze gegeben.\\
Peter has the.\textsc{acc} food the.\textsc{dat} cat given\\
``Peter has given the food to the cat''\hfill[Scrambled order]
\z.
\z.

\begin{itemize}
  \item According to \textcite{dehoop1992case} and \textcite{diesing1992indefinites} scrambled objects are obligatorily interpreted strong.
\end{itemize}
\subsection{Parallel 2}
\begin{itemize}
  \item \textcite{milsark1974existential} argues that only strong-quantified DPs can serve as subjects of Individual-Level predicate
\end{itemize}
\ex. I had been struggling with a complicated set of data \ldots
\a.?* A problem was particularly hard.
\b. One problem was particularly hard.
\b. \{?A/one\} problem that I came across was particularly hard.
\b. One of the problems was particularly hard.\hfill\parencite{heycock2012specification}

\begin{itemize}
  \item Heycock argues that the same pattern holds for indefinite SCC subjects as shown in \Next.
\end{itemize}
\ex.
\a.?* A problem was that we didn't understand all the parameters.
\b. One problem was that we didn't understand all the parameters.
\b. \{A/one\} problem that I came across was that we didn't understand all the parameters.
\b. One of the problems was that we didn't understand all the parameters.\hfill\parencite{heycock2012specification}

\subsection{Issues}
\begin{itemize}
  \item Heycock's proposal that the indefinite restriction is a restriction on weak-quantified SCC subjects fails to capture the data
  \item Some weak-quantified DPs can be SCC subjects \Next and some strong-quantified DPs cannot \NNext. 
\end{itemize}
\ex.
\a.
\a. There is \textbf{a building no-one likes} on St George Street.
\b. \textbf{a building no-one likes} is Robarts.
\z.
\b.
\a. There are \textbf{sm side-effects}.
\b. \textbf{Sm side-effects} are headaches and dizziness. 
\z.

\ex.
\a. Each doctor is Mary, Bill, Sue, and John. (*Specificational)
\b.? Most early generative grammarians are Chomsky and Halle. (*Specificational)
\b.? SOME side-effects are drowsiness and blurred vision. (*Specificational)

\end{document}
