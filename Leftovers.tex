%        File: Leftovers.tex
%     Created: Fri Sep 25 01:00 PM 2015 E
% Last Change: Fri Sep 25 01:00 PM 2015 E
%
% arara: pdflatex: {options: "-draftmode"}
% arara: biber
% arara: pdflatex: {options: "-draftmode"}
% arara: pdflatex: {options: "-file-line-error-style"}
\documentclass[GPFinal]{subfiles}

\begin{document}
\subsection{Predicative Clauses}
As \textcite{mikkelsen2004specifying} notes, predicational copular clauses have a flexible information structure.
\ex.
\a. [Mary]$_F$ is a [doctor]$_{CT}$
\b. [Mary]$_{CT}$ is a [doctor]$_F$

In \Last[a] the noun \textit{doctor}
\subsection{Apparent SCs}
\textcite{heycock2012specification} suggests that those indefinites that can be SC subjects are strong indefinites.
This claim both under- and over-generates.
Although most weak quantifiers are ambiguous between weak and strong, \textit{a(n)} and \textit{sm} (the reduced form of \textit{some}) are not.
Despite not being strong though, \textit{a(n)} and \textit{sm} can head SC subjects.
\ex.
\a. An UNDERrated figure in the history of generative grammar is Eric Lenneberg.
\b. Sm SIDE-effects are headache, blurred vision and sore throat.
\z.

It is often assumed that stressed \textit{one} is actually a covert partitive (\textit{one of the \ldots}), though it does not behave quite as such.
For instance \textit{one of the Xs} presupposes that the set of X's is non-singleton, while \textit{one X} merely implies it.
\ex.
\a. ONE ugly building on campus is Robarts. In fact, it's the only one.
\b. One of the ugly buildings on campus is Robarts. \# In fact, it's the only one.

\end{document}


