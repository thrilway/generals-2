%        File: Leftovers.tex
%     Created: Fri Sep 25 01:00 PM 2015 E
% Last Change: Fri Sep 25 01:00 PM 2015 E
%
% arara: pdflatex: {options: "-draftmode"}
% arara: biber
% arara: pdflatex: {options: "-draftmode"}
% arara: pdflatex: {options: "-file-line-error-style"}
\documentclass[GPFinal]{subfiles}

\begin{document}
\subsection{Simple Definite SC Subjects}

\textcite{heycock2010variability} and \textcite{bejarkahnemuyipour2013agreement} discuss a particular reading of SCs with simple definite subjects, called ``the Poirot reading'' which is shown below in \Next.
\ex.And Poirot pointed at the Major and said ``For a long time now we have been trying to establish the identity of the murderer. But now I know\ldots\\
\ldots The murderer is you''

At first blush, this seems to be a counterexample to my proposal.
In this context, the existence and relevance \textit{the murderer} is entirely given/presupposed, while the fact that the identity of the murderer is Poirot's addressee seems to be new/contrastive. 
This would mean that no part of the subject is CT-marked, which should render the clause unacceptable.

If we consider the context carefully, we can see that this is not the entire story.
The sentence \textit{The murderer is you} would occur at the culmination of a murder mystery at which point many properties of \textit{the murderer} have been gleaned from the evidence.
The only relevant ``property'' left is \textit{the murderer}'s identity.
So, what is given is \textit{the murderer} and several of \textit{the murderer}'s properties.
What is new/contrastive is the identity of \textit{the murderer}, and that that identity is Poirot's addressee.

\end{document}


