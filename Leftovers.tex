%        File: Leftovers.tex
%     Created: Fri Sep 25 01:00 PM 2015 E
% Last Change: Fri Sep 25 01:00 PM 2015 E
%
% arara: pdflatex: {options: "-draftmode"}
% arara: biber
% arara: pdflatex: {options: "-draftmode"}
% arara: pdflatex: {options: "-file-line-error-style"}
\documentclass[GPFinal]{subfiles}

\begin{document}
\subsection{Predicational Clauses}
Although this paper is concerned with specificational clauses, the nature of the claim made means that it makes predictions beyond SCs.
Consider again the rationale for ruling out \Next.
\ex.* A doctor is Mary.

The string in \Last is associated with the ill-formed d-tree in \Next.
\ex. 
\begin{forest}
	  tree defaults
	  [Who is a what?
	    [Who is a doctor
	      [A doctor is Derek]
	      [A doctor is Mary]
	      [\dots]
	    ]
	    [Who is a lawyer?]
	    [Who is a socialist
	      [A socialist is Mary]
	    ]
	    [\dots]
	  ]
	\end{forest}

Since the ill-formedness of \Last renders the SC associated with it unacceptable we would expect \textit{any} linguistic expression associated with it to be unacceptable.
Consider the hypothetical predicative clause below in \Next, which is predicted to be unacceptable.
\ex. Mary$_F$ is a [doctor]$_{CT}$.

While the predicational clause is, generally speaking, acceptable, its status with the specified CT-Foc structure is unclear.
An investigation of status of \Last and similar predicational clauses that carefully controls for their discourse context and intonational contour would provide empirical support for or evidence against my proposal.
Such an investigation, however, is beyond the scope of this paper.

\subsection{\posscite{heycock2012specification} claim revisited}\label{sec:HeycockRedux}
Recall from section \ref{sec:Heycock}, that \textcite{heycock2012specification} claimed that those indefinites that can be SC subjects are strong indefinites.
In light of the above discussion, we are able to test the validity of Heycock's claim empirically.
Although most weak quantifiers are ambiguous between weak and strong, \textit{a(n)} and \textit{sm} (the reduced form of \textit{some}) are not.
Despite not being strong though, \textit{a(n)} and \textit{sm} can head SC subjects.
\ex.
\a. An UNDERrated figure in the history of generative grammar is Eric Lenneberg.
\b. Sm SIDE-effects are headache, blurred vision and sore throat.
\z.

Strong indefinites don't seem to be able to function as SC subjects, as demonstrated below in \Next.
\ex.
\a. Each doctor is Mary, Bill, Sue, and John. (*Specificational)
\b.? Most early generative grammarians are Chomsky and Halle. (*Specificational)
\b.? SOME side-effects are drowsiness and blurred vision. (*Specificational)

Copular clauses with strong indefinite subjects, instead are most naturally interpreted as identificational.
Consider also, the minimal pair in \Next, with only strong/weak varying between the two.
\ex.
\a. SOME side-effects are drowsiness and blurred vision. (*Specificational)
\b. sm side-effects are drowsiness and blurred vision.

The subject in \Last[b] is a weak indefinite because it and others like it can be used in existential constructions.
\ex. 
\a. There is \textbf{a building no-one likes} on St George Street.
\b. \textbf{a building no-one likes} is Robarts.

\ex.
\a. There are \textbf{sm side-effects}.
\b. \textbf{Sm side-effects} are headaches and dizziness. 

Contrary to \posscite{heycock2012specification} proposal, it is the weak counterpart that can be the subject of an SC.
It seems, then, that the proposal that weak indefinites are barred from being SC subjects cannot stand.

\subsection{Simple Definite SC Subjects}

\textcite{heycock2010variability} and \textcite{bejarkahnemuyipour2013agreement} discuss a particular reading of SCs with simple definite subjects, called ``the Poirot reading'' which is shown below in \Next.
\ex.And Poirot pointed at the Major and said ``For a long time now we have been trying to establish the identity of the murderer. But now I know\ldots\\
\ldots The murderer is you''

At first blush, this seems to be a counterexample to my proposal.
In this context, the existence and relevance \textit{the murderer} is entirely given/presupposed, while the fact that the identity of the murderer is Poirot's addressee seems to be new/contrastive. 
This would mean that no part of the subject is CT-marked, which should render the clause unacceptable.

If we consider the context carefully, we can see that this is not the entire story.
The sentence \textit{The murderer is you} would occur at the culmination of a murder mystery at which point many properties of \textit{the murderer} have been gleaned from the evidence.
The only relevant ``property'' left is \textit{the murderer}'s identity.
So, what is given is \textit{the murderer} and several of \textit{the murderer}'s properties.
What is new/contrastive is the identity of \textit{the murderer}, and that that identity is Poirot's addressee.

\end{document}


