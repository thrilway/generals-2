%        File: Leftovers.tex
%     Created: Fri Sep 25 01:00 PM 2015 E
% Last Change: Fri Sep 25 01:00 PM 2015 E
%
% arara: pdflatex: {options: "-draftmode"}
% arara: biber
% arara: pdflatex: {options: "-draftmode"}
% arara: pdflatex: {options: "-file-line-error-style"}
\documentclass[GPFinal]{subfiles}

\begin{document}
\subsection{Predicational Clauses}
Although this paper is concerned with specificational clauses, the nature of the claim made means that it makes predictions beyond SCs.
Consider again the rationale for ruling out \Next.
\ex.* A doctor is Mary.

The string in \Last is associated with the ill-formed d-tree in \Next.
\ex. 
\begin{forest}
	  tree defaults
	  [Who is a what?
	    [Who is a doctor
	      [A doctor is Derek]
	      [A doctor is Mary]
	      [\dots]
	    ]
	    [Who is a lawyer?]
	    [Who is a socialist
	      [A socialist is Mary]
	    ]
	    [\dots]
	  ]
	\end{forest}

Since the ill-formedness of \Last renders the SC associated with it unacceptable we would expect \textit{any} linguistic expression associated with it to be unacceptable.
Consider the hypothetical predicative clause below in \Next, which is predicted to be unacceptable.
\ex. Mary$_F$ is a [doctor]$_{CT}$.

While the predicational clause is, generally speaking, acceptable, its status with the specified CT-Foc structure is unclear.
An investigation of status of \Last and similar predicational clauses that carefully controls for their discourse context and intonational contour would provide empirical support for or evidence against my proposal.
Such an investigation, however, is beyond the scope of this paper.

\subsection{\posscite{heycock2012specification} claim revisited}\label{sec:HeycockRedux}
Recall from section \ref{sec:Heycock}, that \textcite{heycock2012specification} claimed that those indefinites that can be SC subjects are strong indefinites.
In light of the above discussion, we are able to test the validity of Heycock's claim empirically.
Although most weak quantifiers are ambiguous between weak and strong, \textit{a(n)} and \textit{sm} (the reduced form of \textit{some}) are not.
Despite not being strong though, \textit{a(n)} and \textit{sm} can head SC subjects.
\ex.
\a. An UNDERrated figure in the history of generative grammar is Eric Lenneberg.
\b. Sm SIDE-effects are headache, blurred vision and sore throat.
\z.

Strong indefinites don't seem to be able to function as SC subjects, as demonstrated below in \Next.
\ex.
\a. Each doctor is Mary, Bill, Sue, and John. (*Specificational)
\b.? Most early generative grammarians are Chomsky and Halle. (*Specificational)
\b.? SOME side-effects are drowsiness and blurred vision. (*Specificational)

Copular clauses with strong indefinite subjects, instead are most naturally interpreted as identificational.
Consider also, the minimal pair in \Next, with only strong/weak varying between the two.
\ex.
\a. SOME side-effects are drowsiness and blurred vision. (*Specificational)
\b. sm side-effects are drowsiness and blurred vision.

The subject in \Last[b] is a weak indefinite because it and others like it can be used in existential constructions.
\ex. 
\a. There is \textbf{a building no-one likes} on St George Street.
\b. \textbf{a building no-one likes} is Robarts.

\ex.
\a. There are \textbf{sm side-effects}.
\b. \textbf{Sm side-effects} are headaches and dizziness. 

Contrary to \posscite{heycock2012specification} proposal, it is the weak counterpart that can be the subject of an SC.
It seems, then, that the proposal that weak indefinites are barred from being SC subjects cannot stand.

\subsection{\textit{One} and \textit{another}}
As mentioned in section \ref{sec:MainArgument} The determiner-like elements \textit{one} and \textit{another} can serve as CTs in SC subjects.
Given the account argued for above, this means that alternatives generated by \Next[b] and \Next[c] are licit while those generated \Next[a] are illicit.
\ex.
\a.* A doctor$_{CT}$ is Mary.
\b.\label{ex:OneCT} One$_{CT}$ doctor is Mary.
\b. Another$_{CT}$ doctor is Mary.

This raises the question: What are the alternatives generated by \textit{one} and \textit{another}?

Since it is the most straightforward, let's consider \textit{another} first.
The sentence in \Last[c] can be felicitously uttered only in discourse contexts in which other doctors have been discussed and identified
\ex.
\a. Let me tell you about doctors.\\
\#Another doctor is Mary.
\b. Molly is a doctor.\\
Another doctor is Mary.

Given these judgement the CT-alternatives generated by \textit{Another doctor is Mary} seem to be equivalent to the question \textit{Who is a doctor?} as in the d-tree below in \Next.

\ex. 
\begin{forest}
  tree defaults
  [Who is a doctor?
    [\textbf{Molly is a doctor.}]
    [Who is another doctor?
      [\textbf{Another doctor is Mary.}]
      [\ldots]
    ]
  ]
\end{forest}

The SC in \ref{ex:OneCT} shows the inverse felicity conditions, it requires that doctors have been discussed but none have been named.
\ex.
\a. Let me tell you about doctors.\\
One doctor is Mary.
\b. Molly is a doctor.\\
\#One doctor is Mary.

If \textit{one} is merely the stressed pronunciation of \textit{a/an}, then the account I have proposed woul likely require serious revision.
Fortunately, there are good reasons doubt that \textit{one} and \textit{a/an} are distinct lexical items.
First, it is unlikely that \textit{one} is the stressed version of \textit{a/an}, since \textit{a/an} has another stressed version pronounced [ej]/[\ae{}n], which usually marks a contrast of definiteness.
\ex.
\a.[A:] Are you the professor?
\b.[B:] I'm [ej] professor.

Also, \textcite{kayne2015one} presents several pieces of evidence that \textit{one} is lexically distinct from \textit{a/an}.
While \textit{a/an NP}  can be interpreted as generic, \textit{one NP} cannot
\ex.
\a. A spider has eight legs and many eyes. (generic/specific)
\b. One spider has eight legs and many eyes. (*generic/specific)\hfill\parencite{kayne2015one}

He also notes that the syntactic distribution of \textit{a/an} differs from \textit{one} as shown below.
\ex.
	\a. 
		\a. too long a book
		\b.* too long one book
		\z.
	\b.
		\a. a few books
		\b.* one few books
		\z.
	\b.
		\a.* They're selling a-drawer desks in the back of the store.
		\b. They're selling one-drawer desks in the back of the store.
		\z.\hfill\parencite{kayne2015one}

While Kayne argues that \textit{one} is a complex determiner that is composed of \textit{a/an} and a singular classifier, the specific syntax/semantics of \textit{one} is beyond the scope of this paper.
What is important is that \textit{one} is semantically richer than \textit{a/an} so that it can be CT-marked.

For the purpose of illustration, I will assume \posscite{kayne2015one} analysis of \textit{one} is correct.
What does this predict about the CT interpretation of \textit{One doctor is Mary}, then?
The focus interpretation is the question \textit{Who is one doctor?}, and the CT interpretation would be \textit{Who is a doctor?}, as shown below in \Next.
\ex.
\begin{forest}
  tree defaults
  [Who is a doctor?
    [\(\cdots\)]
    [Who is one doctor?
      [\textbf{One doctor is Mary}]
      [\(\cdots\)]
    ]
  ]
\end{forest}


According to Kayne's analysis, the question \textit{Who is one doctor?} is only answerable with singular and therefore, its alternative questions ask for plural answers.

\ex.
\begin{forest}
  tree defaults
  [$\left\{ x \in D_e | doctor(x) \right\}$
    [$\left\{ x \in D_e | doctor(x) \wedge \neg \textsc{sing}(x) \right\}$]
    [$\left\{ x \in D_e | doctor(x) \wedge \textsc{sing}(x) \right\}$]
  ]
\end{forest}

%\ex.
%\begin{forest}
%  tree defaults
%  [{$
%    \begin{Bmatrix}
%      \text{Mary}\oplus\text{Molly}\oplus\text{John,}\\
%      \text{Mary}\oplus\text{Molly,}\\
%      \text{Molly}\oplus\text{John,}\\
%      \text{Mary}\oplus\text{Molly,}\\
%      \text{Mary, Molly, John}
%    \end{Bmatrix}
%  $}
%  [{$
%    \begin{Bmatrix}
%      \text{Mary}\oplus\text{Molly}\oplus\text{John,}\\
%      \text{Mary}\oplus\text{Molly,}\\
%      \text{Molly}\oplus\text{John,}\\
%      \text{Mary}\oplus\text{Molly}
%    \end{Bmatrix}
%  $}]
%  [{$
%    \begin{Bmatrix}
%      \text{Mary,}\\
%      \text{Molly,}\\
%      \text{John}\\
%    \end{Bmatrix}
%  $}]
%]
%\end{forest}
%
Again, whether this particular d-tree is correct is not vital for my proposal, just that some d-tree consistent with the one in \LLast is correct.

%It is often assumed that stressed \textit{one} is actually a covert partitive (\textit{one of the \ldots}), though it does not behave quite as such.
%For instance \textit{one of the Xs} presupposes that the set of X's is non-singleton, while \textit{one X} merely implies it.
%\ex.
%\a. ONE ugly building on campus is Robarts. In fact, it's the only one.
%\b. One of the ugly buildings on campus is Robarts. \# In fact, it's the only one.

\end{document}


